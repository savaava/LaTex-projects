\subsection{Requisiti Funzionali}
\newcounter{IFcounter}
\newcommand{\IFitem}{\addtocounter{IFcounter}{1}IF-\theIFcounter}
	\begin{tcolorbox}[colback=white,colframe=black!80!white,title=\textbf{Funzionalità individuali IF}]
	\begin{itemize}[itemsep=2pt, topsep=0pt, label=\textbf{\IFitem}]
		\item \textbf{Visualizzazione in ordine alfabetico}
		\\I contatti sono sempre visualizzati in ordine alfabetico, ordinando per cognome e nome.
		\\Si offre
		\\Anche usufruendo della visualizzazione per tag l’ordine dei contatti è sempre alfabetico (vedi IF-9).
		
		\item \textbf{Ricerca del contatto}
		\\L’utente può cercare il contatto inserendo nella barra di ricerca una sottostringa del nome o del cognome del contatto desiderato.
		\\Verranno visualizzati i contatti che rispettano il requisito di ricerca.
		
		\item \textbf{Ricerca per numero di telefono o mail}
		\\L’utente può cercare il contatto inserendo nella barra di ricerca un suffisso dei numeri di telefono o una sottostringa della mail associati al contatto desiderato. 
		\\Verranno visualizzati i contatti che rispettano il requisito di ricerca. 
		
		\item \textbf{Aggiunta del contatto}
		\\Cliccando sul simbolo “+” compare nella sezione a destra una
		schermata in cui è possibile inserire i campi del contatto (definiti in 
		DF-1).
		\\Successivamente si può salvare il nuovo contatto o annullare  
		l’operazione con i rispettivi tasti “Salva” o “Annulla”.
		////DOPPIONI
		
		\item \textbf{Modifica di un contatto}
		\\Cliccando su un contatto e premendo sul bottone “Modifica” nella sezione a destra è possibile modificare i suoi campi (definiti in DF-1). Dopo aver effettuato modifiche è possibile annullare o salvare l’operazione con i tasti “Salva” o “Annulla”.
		
		\item \textbf{Eliminazione di un contatto}
		\\Cliccando su un contatto e premendo sul bottone “Elimina” nella sezione a destra si rimuove dalla rubrica.
		
		\item \textbf{Importazione contatto/i da file}
		\\L’utente può scegliere, tramite la sezione del menù “File”, di importare nella rubrica i contatti da un file \texttt{.csv} o \texttt{.vCard}.
		
		\item \textbf{Esportazione contatto/i da file}
		\\L’utente può aprire il menù a tendina “File” e decidere di esportare 
		tutti i contatti oppure quelli associati ad un particolare tag selezionato 
		da una lista che uscirà al momento dell’esportazione 
		\\L’utente può scegliere di esportare i contatti in un file .csv oppure 
		.vCard e specificare il nome del file. 
		
		\item \textbf{Visualizzazione per tag}
		\\L’utente cliccando sull’icona dell’imbuto \includegraphics[height=0.4cm]{images/WhatsApp Image 2024-11-22 at 16.28.56.jpeg} sceglie uno dei tag che comporta la visualizzazione dei soli contatti associati al tag selezionato.
		\\La visualizzazione è sempre in ordine alfabetico (vedi IF-1).
		
	\end{itemize}
\end{tcolorbox}

\newcounter{DFcounter}
\newcommand{\DFitem}{\addtocounter{DFcounter}{1}DF-\theDFcounter}
\begin{tcolorbox}[colback=white,colframe=black!80!white,title=\textbf{Esigenze dei dati e informazioni DF}]
	\begin{itemize}[itemsep=2pt, topsep=0pt, label=\textbf{\DFitem}]
		\item \textbf{Campi del contatto}
		\\Per ogni contatto bisogna conservare i seguenti dati:
		\begin{itemize}[noitemsep, topsep=0pt, label=$\bullet$]
			\item cognome e/o nome
			\item da 0 a 3 numeri di telefono
			\item da 0 a 3 mail 
			\item 0 o più tag.
		\end{itemize}
		Tali dati possono essere aggiunti in fase di creazione del contatto e modificati in qualsiasi momento.
		
		\item \textbf{Categorizzazione contatto}
		\\L’utente può aggiungere facoltativamente un tag a piacere (ad esempio preferiti, famiglia, lavoro, …) ad ogni contatto in fase di aggiunta/modifica del contatto.
		\\Un tag è una particolare proprietà che si può associare ad 1 o più contatti.
		C’è un menù a tendina che mostra tutti i tag inseriti dall’utente, e consente di aggiungerne o di rimuoverne altri. 
		\\Per la visualizzazione dei contatti appartenenti ai tag vedi IF-9.
		
		\item \textbf{Immagine contatto}
		\\Ad ogni contatto è associata come immagine una sagoma grigia che vedrà nella visione dettagliata del contatto. Ma in fase di aggiunta/modifica di un contatto, l’utente può aggiungere un’immagine personalizzata, la quale può essere selezionata da quelle suggerite dall’applicazione stessa oppure importarne una dall’esterno.
		
		\item \textbf{Salvataggio in locale}
		\\Salvataggio in locale dei contatti inseriti in rubrica. Tale salvataggio è quello di default se non viene specificata la preferenza di usare un database (vedi IS-1).
		
	\end{itemize}
\end{tcolorbox}

\newcounter{UIcounter}
\newcommand{\UIitem}{\addtocounter{UIcounter}{1}UI-\theUIcounter}
\begin{tcolorbox}[colback=white,colframe=black!80!white,title=\textbf{Interfaccia Utente UI}]
	\begin{itemize}[itemsep=2pt, topsep=0pt, label=\textbf{\UIitem}]
		\item \textbf{Visualizzazione specifica del contatto}
		\\Dopo aver selezionato dalla rubrica un contatto, l’utente vedrà nella sezione a destra la visualizzazione dettagliata del contatto scelto. In questa sezione è visibile, oltre al nome e/o il cognome, anche i numeri di telefono, le email e i tag associati al contatto.
		
		\item \textbf{Avere interfaccia utente di tipo grafico} 
		\\Tramite l’utilizzo di JavaFX l’utente interagisce con il programma tramite interfaccia grafica, permettendone un utilizzo facilitato e maggiormente intuitivo.
		
		\item \textbf{Barra laterale di navigazione}
		\\Nella schermata di visualizzazione dei contatti, vi è sul bordo sinistro, disposto verticalmente, una barra di navigazione che contiene lettere in ordine alfabetico. L’utente cliccando su una lettera, l’elenco salta ai contatti con questa lettera iniziale, velocizzando così lo scorrimento della rubrica.
		\\Le lettere contenute in tale barra sono le iniziali degli unici contatti presenti nell’elenco (sia nella visualizzazione complessiva che per tag).
		
	\end{itemize}
\end{tcolorbox}

\newcounter{IScounter}
\newcommand{\ISitem}{\addtocounter{IScounter}{1}IS-\theIScounter}
\begin{tcolorbox}[colback=white,colframe=black!80!white,title=\textbf{Interfacce con sistemi esterni IS}]
	\begin{itemize}[itemsep=2pt, topsep=0pt, label=\textbf{\ISitem}]
		\item \textbf{Sincronizzazione con database}
		\\Nelle impostazioni dell’applicazione (cliccando su "File" e poi su "Configurazione") è possibile esprimere la propria preferenza riguardo la possibilità di salvare la propria rubrica su un database esterno, fornendo il link. 
		\\In questo caso, il salvataggio dati non avverrà più in locale, come avviene di default, (tramite file \texttt{.bin}) ma sul database (vedi DF-4).
	\end{itemize}
\end{tcolorbox}


\subsection{Requisiti Non Funzionali}
\newcounter{FCcounter}
\newcommand{\FCitem}{\addtocounter{FCcounter}{1}FC-\theFCcounter}
\begin{tcolorbox}[colback=white,colframe=black!80!white,title=\textbf{Vincoli FC}]
	\begin{itemize}[itemsep=2pt, topsep=0pt, label=\textbf{\FCitem}]
		\item \textbf{Usabilità}
		\\L’utente deve poter utilizzare la rubrica in maniera intuitiva,
		fornendogli un’interfaccia grafica invitante.
		
		\item \textbf{Velocità di risposta}
		\\L’utente deve poter effettuare le operazioni fornite dalla rubrica in tempi ragionevolmente brevi.		
	\end{itemize}
\end{tcolorbox}


\subsection{Tabella di categorizzazione dei requisiti}

\newpage
\subsection{Casi d'Uso}
\begin{tcolorbox}[colback=white,colframe=black!80!white,title=\textbf{C0 - Aggiungere contatto}]
\textbf{Attore partecipante}:
\\\textbf{Precondizioni}: 
\\\textbf{Postcondizioni}:
\\\textbf{Flusso di eventi normale}:
\begin{enumerate}[noitemsep, topsep=0pt]
	\item Utente clicca il pulsante “+”;
	\item Utente inserisce il nome e/o il cognome;
	\item Utente inserisce i numeri di telefono e/o le email ;
	\item Utente clicca il pulsante “Salva”;
	\item Il sistema aggiunge il contatto;
\end{enumerate}
\textbf{Flusso di eventi alternativo}:
\begin{itemize}[noitemsep, topsep=0pt]
	\item[3.a] Utente clicca il pulsante “Annulla”.
\end{itemize}
\end{tcolorbox}

\begin{tcolorbox}[colback=white,colframe=black!80!white,title=\textbf{C1 - Rimuovere contatto}]
	\textbf{Attore partecipante}: Utente
	\\\textbf{Precondizioni}: L’utente ha deciso il contatto da eliminare
	\\\textbf{Postcondizioni}: Viene rimosso dalla rubrica il contatto selezionato
	\\\textbf{Flusso di eventi normale}:
	\begin{enumerate}[noitemsep, topsep=0pt]
		\item Utente seleziona il contatto;
		\item Utente clicca il tasto “Elimina”;
		\item Il sistema rimuove il contatto;		
	\end{enumerate}
	\textbf{Flusso di eventi alternativo}: /
\end{tcolorbox}

\begin{tcolorbox}[colback=white,colframe=black!80!white,title=\textbf{C2}]
	\textbf{Attore partecipante}:
	\\\textbf{Precondizioni}: 
	\\\textbf{Postcondizioni}:
	\\\textbf{Flusso di eventi normale}:
	\begin{enumerate}[noitemsep, topsep=0pt]
		\item 
	\end{enumerate}
	\textbf{Flusso di eventi alternativo}:
	\begin{itemize}[noitemsep, topsep=0pt]
		\item[]
	\end{itemize}
\end{tcolorbox}

\begin{tcolorbox}[colback=white,colframe=black!80!white,title=\textbf{C3}]
	\textbf{Attore partecipante}:
	\\\textbf{Precondizioni}: 
	\\\textbf{Postcondizioni}:
	\\\textbf{Flusso di eventi normale}:
	\begin{enumerate}[noitemsep, topsep=0pt]
		\item 
	\end{enumerate}
	\textbf{Flusso di eventi alternativo}:
	\begin{itemize}[noitemsep, topsep=0pt]
		\item[]
	\end{itemize}
\end{tcolorbox}