\documentclass[12pt, a4paper]{article}
\usepackage[top=1.6cm, bottom=1.6cm, left=2cm, right=2cm]{geometry}
\usepackage{listings}
\usepackage{xcolor} % Per personalizzare i colori
\usepackage{graphicx}
\usepackage{caption}
\usepackage{multicol}
\setlength{\columnsep}{20pt} % Distanza tra le colonne

\lstset{
	language=Java,
    basicstyle=\ttfamily\small,   % Stile base (monospaziato)
	keywordstyle=\color{blue},    % Colore delle parole chiave
	commentstyle=\color{gray},    % Colore dei commenti
	stringstyle=\color{red},      % Colore delle stringhe
	numbers=left,                 % Numerazione delle righe
	numberstyle=\tiny,            % Stile dei numeri di riga
	frame=single,                 % Cornice intorno al codice
	breaklines=true,              % Divisione automatica delle righe lunghe
	tabsize=4,                    % Dimensione del tabulatore
	showspaces=false,             % Non mostra spazi visibili
	showstringspaces=false        % Non mostra spazi nelle stringhe
}


\title{Prove}
\author{Sava}
\date{\today}


\begin{document}
	\maketitle
	\tableofcontents

	\newpage
	\section{listing di codice java}
	\begin{lstlisting}[language=Java]
public class HelloWorld {
	public static void main(String[] args) {
		System.out.println("Hello,  World!");
	}
}
	\end{lstlisting}	
	
\begin{multicols}{2}
	\centering
	\fbox{\includegraphics[width=0.3\textwidth]{images/example-image}}
	\captionof{figure}{immagine con una cornice semplice.}
	\label{fig:example1}
	
	testo casuale	
	\vspace{10cm}\\testo casuale
	
	\columnbreak
	
	testo casuale
	
	\fcolorbox{red}{green}{\includegraphics[width=0.3\textwidth]{images/example-image}}
	\captionof{figure}{immagine con una cornice verde e rossa}
	\label{fig:example2}
\end{multicols}
	testo casuale \ref{fig:example1}
\end{document}






