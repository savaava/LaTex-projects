\documentclass[12pt, a4paper]{article}

% ******************************** PACKAGE ********************************
\usepackage{hyperref}
\usepackage[top=1.6cm, bottom=1.6cm, left=2cm, right=2cm]{geometry}
\usepackage{listings}
\usepackage{xcolor}
\usepackage{lipsum}
\usepackage{tcolorbox}
\usepackage{enumitem} % per \begin{itemize}[noitemsep,topsep=0pt]
\usepackage{multirow}

% ******************************** SETUP ********************************
\hypersetup{
	linktocpage,
	colorlinks=false, %set true if you want colored links
	linktoc=all,     %set to all if you want both sections and subsections linked
	linkcolor=blue,  %choose some color if you want links to stand out
}

\lstset{
	language=Java,
	basicstyle=\ttfamily\small,   % Stile base (monospaziato)
	keywordstyle=\color{blue},    % Colore delle parole chiave
	commentstyle=\color{gray},    % Colore dei commenti
	stringstyle=\color{red},      % Colore delle stringhe
	numbers=left,                 % Numerazione delle righe
	numberstyle=\tiny,            % Stile dei numeri di riga
	frame=single,                 % Cornice intorno al codice
	breaklines=true,              % Divisione automatica delle righe lunghe
	tabsize=4,                    % Dimensione del tabulatore
	showspaces=false,             % Non mostra spazi visibili
	showstringspaces=false        % Non mostra spazi nelle stringhe
}

% ******************************** MAIN PAGE INFO ********************************
\title{Sintassi e Funzionalità}
\author{Sava}
\date{\today}

% ******************************** DOCUMENTO ********************************
\begin{document}
	\maketitle
	\tableofcontents 
	\listoffigures
	\listoftables
	
	
	\newpage	
	\section{Listing di codici}
	Si deve importare usepackage {listings}
	e lstset{} per modificare il linguaggio
	\subsection{Matlab}
	
	\subsection{java}
	\hypertarget{Sava target}{Sono l'hyper target}
	\begin{lstlisting}[language=Java]
public class HelloWorld {
	public static void main(String[] args) {
		System.out.println("Hello,  World!");
	}
}
	\end{lstlisting}	
	
	
	
	\newpage
	\section{Layout - geometria - box}
	\subsection{Multicolonna}	
	
	
	\subsection{Tcolorbox - Bullets}
	\begin{tcolorbox}
		Semplice tcolorbox.
	\end{tcolorbox}
	\begin{tcolorbox}[colback=red!5!white,colframe=red!50!black,title=Tcolorbox speciale]
		tcolorbox
		\begin{itemize}[noitemsep,topsep=0pt]
			\item con sfondo del box 5\% rosso e 95\% bianco
			\item con sfondo del frame 50\% rosso e 50\% nero
		\end{itemize}
	\end{tcolorbox}
	
	\subsection{tabelle}
	Utilizza il sito web \textit{\url{https://www.tablesgenerator.com/}} per costruire la tabella manualmente e ottenrne il rispettivo codice \LaTeX
	\begin{table}[h]
		\begin{tabular}{|l|l|l|}
			\hline
			a\_00 & a\_01 & a\_02 \\ \hline
			a\_10 & a\_11 & a\_12 \\ \hline
			a\_20 & a\_21 & a\_22 \\ \hline
		\end{tabular}
	\end{table}
	\begin{table}[h]
		\begin{tabular}{|c|cc|c|}
			\hline
			a\_00                        & \multicolumn{2}{c|}{doppio testo}  & a\_02 \\ \hline
			\multirow{2}{*}{doppia riga} & \multicolumn{1}{c|}{a\_11} & a\_12 & a\_12 \\ \cline{2-4} 
			& \multicolumn{1}{c|}{a\_21} & a\_22 & a\_22 \\ \hline
			a\_20                        & \multicolumn{1}{c|}{a\_21} & a\_22 & a\_22 \\ \hline
		\end{tabular}
	\end{table}
	\begin{table}[!ht]
		\centering
		\begin{tabular}{|c|c|c|}
			\hline
			\textbf{cella} & cella & cella \\ \hline
			\textbf{cella} & cella & cella \\ \hline
			\textbf{cella} & cella & cella \\ \hline
		\end{tabular}
		\caption{ciao sono la caption}
		\label{lbl_table3}
	\end{table}
	
	
	
	\newpage
	\section{Riferimenti}
	\subsection{hypertarget - hyperlink}
	\hyperlink{Sava target}{link text all'indice (puntiamo a hypertarget)}
	\subsection{url - href}
	\begin{itemize}[itemsep=1pt,topsep=0pt]
		\item \url{https://www.tablesgenerator.com/}
		\item \textit{\href{https://www.tablesgenerator.com/}{Link alla pagina web del generatore delle tabelle}}
	\end{itemize}
\end{document}