\documentclass[12pt, a4paper]{article}
\usepackage[top=1.6cm, bottom=1.6cm, left=2cm, right=2cm]{geometry}
\usepackage{listings}
\usepackage{xcolor}

\lstset{
	language=Java,
	basicstyle=\ttfamily\small,   % Stile base (monospaziato)
	keywordstyle=\color{blue},    % Colore delle parole chiave
	commentstyle=\color{gray},    % Colore dei commenti
	stringstyle=\color{red},      % Colore delle stringhe
	numbers=left,                 % Numerazione delle righe
	numberstyle=\tiny,            % Stile dei numeri di riga
	frame=single,                 % Cornice intorno al codice
	breaklines=true,              % Divisione automatica delle righe lunghe
	tabsize=4,                    % Dimensione del tabulatore
	showspaces=false,             % Non mostra spazi visibili
	showstringspaces=false        % Non mostra spazi nelle stringhe
}


\title{Sintassi e Funzionalità}
\author{Sava}
\date{\today}

\begin{document}
	\maketitle
	\tableofcontents
	
	\newpage
	\section{Listing di codici}
	Si deve importare usepackage {listings}
	e lstset{} per modificare il linguaggio
	\subsection{Matlab}
	
	\subsection{java}
	\begin{lstlisting}[language=Java]
public class HelloWorld {
	public static void main(String[] args) {
		System.out.println("Hello,  World!");
	}
}
	\end{lstlisting}	
	
	\newpage
	\section{Layout e geometria}
	\subsection{multicolonna}
	\subsection{}
	
	\newpage
	\section{tabelle}
	\begin{tabbing}
		
	\end{tabbing}
	
\end{document}