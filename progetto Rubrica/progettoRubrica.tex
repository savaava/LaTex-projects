\documentclass[12pt, a4paper]{article}

% ******************************** PACKAGE ********************************
\usepackage[top=1.5cm, bottom=1.5cm, left=1.5cm, right=1.5cm]{geometry}
\usepackage{adjustbox}
\usepackage{caption} % Necessario per \captionof
\usepackage{xcolor}
\usepackage{enumitem}
\usepackage[most]{tcolorbox}
\usepackage{fancyhdr}
\pagestyle{fancy}
\usepackage[colorlinks=true]{hyperref}

\usepackage{tikz}
\usepackage{aeguill}
\usepackage{mdframed}

% ******************************** SETUP ********************************
% Comando per disattivare il colore dei link
\newcommand{\disablelinkcolor}{%
	\hypersetup{linkcolor=black}%
}

% Comando per riattivare il colore dei link
\newcommand{\enablelinkcolor}{
	\hypersetup{
	    %hidelinks,
	    linkcolor=blue
		%linkbordercolor=blue,
		%pdfborderstyle={/S/U/W 1} % Bordo sottolineato (stile underline)
	}
}

% ******************************** MAIN PAGE INFO ********************************
\title{\Huge Applicazione Software di una Rubrica}
\author{Andrea Savastano 
	\\ Annamaria Scermino
	\\ Alessandro Monte 
	\\ Francesco Musto}
\date{\today}

% ******************************** DOCUMENTO ********************************
\begin{document}
	\fancyhf{}
	\fancyhead[R]{page \thepage}
	\thispagestyle{fancy}
	
	\maketitle
	\newpage
\disablelinkcolor
	\tableofcontents 
	\listoffigures
	
\enablelinkcolor % Riattiva i link sottolineati
	
	\newpage
	\section*{Descrizione introduttiva}
	\addcontentsline{toc}{section}{Descrizione introduttiva}
	Si vuole realizzare un software destinato a gestire un insieme di contatti, associando ad ognuno di essi le informazioni personali quali nome, cognome, numero di telefono, e-mail.
	\vspace{.2cm}\\Si rende disponibile un'interfaccia grafica che permette all'utente di usufruire delle funzionalità offerte dal sistema.
	
	\section*{Mockup}
	\addcontentsline{toc}{section}{Mockup}
	Si inseriscono alcuni mockup dell'interfaccia grafica del software per scopo illustrativo, finalizzati ad una stesura dei requisiti più dettagliata e ad un supporto alla comprensione dei requisiti.
\begin{figure}[h]
	\centering
	\includegraphics[width=\linewidth]{images/mockup.png}
	\caption{Mockup}
	\label{mockup}			
\end{figure}
	
	
	\newpage
	% Configurazione dell'intestazione
	\fancyhf{} % Resetta intestazioni e piè di pagina
	\fancyhead[L]{\nouppercase{\leftmark}} % Testata sinistra: sezione corrente
	\fancyhead[R]{page \thepage} % Testata destra: numero di pagina
	
	\section{Ingegneria dei Requisiti - SRS}
	\subsection{Requisiti Funzionali}
	\begin{tcolorbox}[breakable, colback=white,colframe=black!80!white,title=\textbf{Funzionalità individuali IF}]
	\begin{itemize}[itemsep=2pt, topsep=0pt]
		\hypertarget{IF-1}{\item[\textbf{IF-1}]}
		\textbf{Visualizzazione in ordine alfabetico}
		\\I contatti sono sempre visualizzati in ordine alfabetico, ordinandoli per cognome e nome.
		\\Si offre la possibilità di poter scegliere, cliccando sull’icona dell’imbuto \includegraphics[height=0.4cm]{images/imbuto_icona.jpeg}, tra l’ordinamento A-Z oppure Z-A.
		\\Anche usufruendo della visualizzazione per Tag l’ordine dei contatti è sempre alfabetico (vedi IF-9).
		
		\item[\textbf{IF-2}] \textbf{Ricerca del contatto}
		\\L’utente può cercare il contatto desiderato inserendo nella casella di ricerca una sottostringa del nome o del cognome del contatto desiderato.
		\\Verranno visualizzati, secondo l’ordine alfabetico stabilito in IF-1, i contatti che rispettano il requisito di ricerca.
		
		\item[\textbf{IF-3}] \textbf{Ricerca per numero di telefono o mail}
		\\L’utente può cercare il contatto desiderato inserendo nella casella di ricerca un suffisso del numero di telefono o una sottostringa della mail associati al contatto. 
		\\Verranno visualizzati, secondo l’ordine alfabetico stabilito in IF-1, i contatti che rispettano il requisito di ricerca.		
		
		\item[\textbf{IF-4}] \textbf{Aggiunta del contatto}
		\\L’utente ha la possibilità di aggiungere un contatto alla sua rubrica.
		In particolare cliccando sul simbolo “+” oppure attraverso il menù a tendina “Rubrica”, compare nella sezione a destra una
		schermata in cui è possibile inserire i campi del contatto (definiti in 
		DF-1, DF-2, DF-3).
		\\Successivamente si può salvare il nuovo contatto o annullare  
		l’operazione con i rispettivi pulsanti “Salva” o “Annulla”.
		\\E’ possibile aggiungere un nuovo contatto contenente gli stessi valori dei campi di 1 o più contatti già esistenti; in particolare il nuovo contatto può avere stesso nome e/o cognome e/o numero di telefono e/o mail di un contatto già presente.		
		
		\item[\textbf{IF-5}] \textbf{Modifica di un contatto}
		\\Cliccando su un contatto e premendo sul bottone “Modifica” nella sezione a destra è possibile modificare i suoi campi (definiti in DF-1). Dopo aver effettuato modifiche è possibile annullare o salvare l’operazione con i tasti “Salva” o “Annulla”.
		
		\item[\textbf{IF-6}] \textbf{Eliminazione di un contatto}
		\\Cliccando su un contatto e premendo sul bottone “Elimina” nella sezione a destra si rimuove dalla rubrica.
		
		\item[\textbf{IF-7}] \textbf{Importazione contatto/i da file}
		\\L’utente può scegliere, tramite la sezione del menù “File”, di importare nella rubrica i contatti da un file \texttt{.csv} o \texttt{.vCard}.
	
		\item[\textbf{IF-8}] \textbf{Esportazione contatto/i da file}
		\\L’utente può aprire il menù a tendina “File” e decidere di esportare 
		tutti i contatti oppure quelli associati ad un particolare tag selezionato 
		da una lista che uscirà al momento dell’esportazione.
		\\L’utente può scegliere di esportare i contatti in un file .csv oppure 
		.vCard e specificare il nome del file. 
		
		\item[\textbf{IF-9}] \textbf{Visualizzazione per tag}
		\\L’utente cliccando sull’icona dell’imbuto \includegraphics[height=0.4cm]{images/imbuto_icona.jpeg} sceglie uno dei tag che comporta la visualizzazione dei soli contatti associati al tag selezionato.
		\\La visualizzazione è sempre in ordine alfabetico (vedi \hyperlink{IF-1}{IF-1}).
		
	\end{itemize}
\end{tcolorbox}

\begin{tcolorbox}[colback=white,colframe=black!80!white,title=\textbf{Esigenze dei dati e informazioni DF}]
	\begin{itemize}[itemsep=2pt, topsep=0pt]
		\item[\textbf{DF-1}] \textbf{Campi del contatto}
		\\Per ogni contatto bisogna conservare i seguenti dati:
		\begin{itemize}[noitemsep, topsep=0pt, label=$\bullet$]
			\item cognome e/o nome
			\item da 0 a 3 numeri di telefono
			\item da 0 a 3 mail 
			\item 0 o più tag.
		\end{itemize}
		Tali dati possono essere aggiunti in fase di creazione del contatto e modificati in qualsiasi momento.
		
		\item[\textbf{DF-2}] \textbf{Categorizzazione contatto}
		\\L’utente può aggiungere facoltativamente un tag a piacere (ad esempio preferiti, famiglia, lavoro, …) ad ogni contatto in fase di aggiunta/modifica del contatto.
		\\Un tag è una particolare proprietà che si può associare ad 1 o più contatti.
		C’è un menù a tendina che mostra tutti i tag inseriti dall’utente, e consente di aggiungerne o di rimuoverne altri. 
		\\Per la visualizzazione dei contatti appartenenti ai tag vedi IF-9.
		
		\item[\textbf{DF-3}] \textbf{Immagine contatto}
		\\Ad ogni contatto è associata come immagine una sagoma grigia che vedrà nella visione dettagliata del contatto. Ma in fase di aggiunta/modifica di un contatto, l’utente può aggiungere un’immagine personalizzata, la quale può essere selezionata da quelle suggerite dall’applicazione stessa oppure importarne una dall’esterno.
		
		\item[\textbf{DF-4}] \textbf{Salvataggio in locale}
		\\Salvataggio in locale dei contatti inseriti in rubrica. Tale salvataggio è quello di default se non viene specificata la preferenza di usare un database (vedi IS-1).
		
	\end{itemize}
\end{tcolorbox}

\begin{tcolorbox}[colback=white,colframe=black!80!white,title=\textbf{Interfaccia Utente UI}]
	\begin{itemize}[itemsep=2pt, topsep=0pt]
		\item[\textbf{UI-1}] \textbf{Visualizzazione specifica del contatto}
		\\Dopo aver selezionato dalla rubrica un contatto, l’utente vedrà nella sezione a destra la visualizzazione dettagliata del contatto scelto. In questa sezione è visibile, oltre al nome e/o il cognome, anche i numeri di telefono, le email e i tag associati al contatto.
		
		\item[\textbf{UI-2}] \textbf{Avere interfaccia utente di tipo grafico} 
		\\Tramite l’utilizzo di JavaFX l’utente interagisce con il programma tramite interfaccia grafica, permettendone un utilizzo facilitato e maggiormente intuitivo.
		
		\item[\textbf{UI-3}] \textbf{Barra laterale di navigazione}
		\\Nella schermata di visualizzazione dei contatti, vi è sul bordo sinistro, disposto verticalmente, una barra di navigazione che contiene lettere in ordine alfabetico. L’utente cliccando su una lettera, l’elenco salta ai contatti con questa lettera iniziale, velocizzando così lo scorrimento della rubrica.
		\\Le lettere contenute in tale barra sono le iniziali degli unici contatti presenti nell’elenco (sia nella visualizzazione complessiva che per tag).
		
	\end{itemize}
\end{tcolorbox}

\begin{tcolorbox}[colback=white,colframe=black!80!white,title=\textbf{Interfacce con sistemi esterni IS}]
	\begin{itemize}[itemsep=2pt, topsep=0pt]
		\item[\textbf{IS-1}] \textbf{Sincronizzazione con database}
		\\Nelle impostazioni dell’applicazione (cliccando su "File" e poi su "Configurazione") è possibile esprimere la propria preferenza riguardo la possibilità di salvare la propria rubrica su un database esterno, fornendo il link. 
		\\In questo caso, il salvataggio dati non avverrà più in locale, come avviene di default, (tramite file \texttt{.bin}) ma sul database (vedi DF-4).
	\end{itemize}
\end{tcolorbox}


\subsection{Requisiti Non Funzionali}
\begin{tcolorbox}[colback=white,colframe=black!80!white,title=\textbf{Vincoli FC}]
	\begin{itemize}[itemsep=2pt, topsep=0pt]
		\item[\textbf{FC-1}] \textbf{Usabilità}
		\\L’utente deve poter utilizzare la rubrica in maniera intuitiva,
		fornendogli un’interfaccia grafica invitante.
		
		\item[\textbf{FC-2}] \textbf{Velocità di risposta}
		\\L’utente deve poter effettuare le operazioni fornite dalla rubrica in tempi ragionevolmente brevi.		
	\end{itemize}
\end{tcolorbox}

\subsection{Tabella di categorizzazione dei requisiti}

\newpage
\subsection{Casi d'Uso formato testuale}
\begin{tcolorbox}[colback=white,colframe=black!80!white,title=\textbf{C0 - Aggiungere contatto}]
\textbf{Attore partecipante}: Utente
\\\textbf{Precondizioni}: Utente ha la schermata della rubrica aperta.
\\\textbf{Postcondizioni}: Nella rubrica viene aggiunto un contatto.
\\\textbf{Flusso di eventi normale}:
\begin{enumerate}[noitemsep, topsep=0pt]
\item Utente clicca il pulsante “+”;
\item Utente inserisce il nome e/o il cognome;
\item Utente inserisce da 0 a 3 numeri di telefono;
\item 	Utente inserisce da 0 a 3 mail ;
\item 	Utente inserisce da 0 a 3 Tag;
\item 	Utente inserisce un’immagine;
\item 	Utente salva l’operazione;
\item 	Il sistema aggiunge il contatto.
\end{enumerate}
\textbf{Flusso di eventi alternativo}:
\begin{itemize}[noitemsep, topsep=0pt]
	\item[1a. ] Utente clicca sul pulsante “Rubrica”;
	\item[1a.1] Utente crea il contatto;
	\item[6a. ] Utente non inserisce alcuna immagine, quindi rimane quella di default;
	\item[7a. ] Utente annulla l’operazione;
	\item[7a.1] L’esecuzione riprende dal passo 1;
\end{itemize}
\end{tcolorbox}

\begin{tcolorbox}[colback=white,colframe=black!80!white,title=\textbf{C1 - Eliminare contatto}]
	\textbf{Attore partecipante}: Utente
	\\\textbf{Precondizioni}: Esiste almeno un contatto nella rubrica.
	\\\textbf{Postcondizioni}: Viene rimosso dalla rubrica il contatto selezionato.
	\\\textbf{Flusso di eventi normale}:
	\begin{enumerate}[noitemsep, topsep=0pt]
\item	Utente seleziona il contatto da eliminare;
\item	Utente clicca il pulsante “Elimina”;
\item	Utente conferma l’operazione dalla schermata di conferma;
\item	Il sistema rimuove il contatto.
	\end{enumerate}
	\textbf{Flusso di eventi alternativo}:
	\begin{itemize}[noitemsep, topsep=0pt]
		\item[3a. ] Utente annulla l’operazione dalla schermata di conferma;
		\item[3a.1] L’esecuzione riprende dal passo 1;
	\end{itemize}
\end{tcolorbox}

\begin{tcolorbox}[colback=white,colframe=black!80!white,title=\textbf{C2 - Modificare contatto}]
	\textbf{Attore partecipante}: Utente
	\\\textbf{Precondizioni}: Esiste almeno un contatto nella rubrica.
	\\\textbf{Postcondizioni}: Viene modificato il contatto selezionato.
	\\\textbf{Flusso di eventi normale}:
	\begin{enumerate}[noitemsep, topsep=0pt]
\item Utente seleziona il contatto da modificare;
\item Utente clicca il pulsante “Modifica”; 
\item Utente modifica uno o più campi del contatto;
\item Utente salva l’operazione;
\item Il sistema modifica il contatto.		
	\end{enumerate}
	\textbf{Flusso di eventi alternativo}:
	\begin{itemize}[noitemsep, topsep=0pt]
		\item[4a. ] Utente annulla l’operazione;
		\item[4a.1] L’esecuzione riprende dal passo 1;
	\end{itemize}
\end{tcolorbox}

\begin{tcolorbox}[colback=white,colframe=black!80!white,title=\textbf{C3 - Cercare contatto}]
	\textbf{Attore partecipante}: Utente
	\\\textbf{Precondizioni}: Utente ha la schermata della rubrica aperta.
	\\\textbf{Postcondizioni}: I contatti che rispettano il criterio di ricerca vengono visualizzati.
	\\\textbf{Flusso di eventi normale}:
	\begin{enumerate}[noitemsep, topsep=0pt]
		\item Utente scrive nella casella di ricerca una sottostringa del nome o del cognome del contatto da cercare;
		\item Il sistema visualizza l’insieme dei contatti che rispetta il criterio di ricerca.		
	\end{enumerate}
	\textbf{Flusso di eventi alternativo}:
	\begin{itemize}[noitemsep, topsep=0pt]
		\item[1a.] Utente scrive nella casella di ricerca un prefisso del numero di telefono del contatto da cercare;
		\item[1b.] Utente scrive nella casella di ricerca una sottostringa della mail del contatto da cercare;
	\end{itemize}
\end{tcolorbox}

\begin{tcolorbox}[colback=white,colframe=black!80!white,title=\textbf{C4 - Importare rubrica}]
	\textbf{Attore partecipante}: Utente
	\\\textbf{Precondizioni}: Utente possiede un file \texttt{.csv} o \texttt{.vCard}
	\\\textbf{Postcondizioni}: Vengono caricati nella rubrica dell’utente i contatti contenuti nel file fornito.
	\\\textbf{Flusso di eventi normale}:
	\begin{enumerate}[noitemsep, topsep=0pt]
		\item Utente clicca il menù a tendina “File”;
		\item Utente clicca il pulsante “Importa”;
		\item Utente fornisce il file con estensione \texttt{.csv} o \texttt{.vCard};
		\item Utente seleziona file;
		\item Utente importa il file;
		\item Gli eventuali contatti presenti nel file vengono aggiunti alla rubrica dell’utente.		
	\end{enumerate}
	\textbf{Flusso di eventi alternativo}:
	\begin{itemize}[noitemsep, topsep=0pt]
		\item[3a. ] Utente chiude il pop-up;
		\item[3a.1] L’esecuzione riprende dal passo 1;
		\item[4a. ] Utente fornisce un file con contenuto non interpretabile dalla rubrica;
		\item[4a.1] L’esecuzione riprende dal passo 3;
		\item[4b. ] Utente decide di non fornire un file;
		\item[4b.1] L’esecuzione riprende dal passo 3;
		\item[5a. ] Utente chiude il pop-up;
		\item[5a.1] L’esecuzione riprende dal passo 1;
		
	\end{itemize}
\end{tcolorbox}

\begin{tcolorbox}[colback=white,colframe=black!80!white,title=\textbf{C5 - Esportare rubrica}]
	\textbf{Attore partecipante}: Utente
	\\\textbf{Precondizioni}: Utente ha la schermata della rubrica aperta.
	\\\textbf{Postcondizioni}: Viene prodotto un file \texttt{.csv} o \texttt{.vCard} con i contatti selezionati.
	\\\textbf{Flusso di eventi normale}:
	\begin{enumerate}[noitemsep, topsep=0pt]
		\item Utente clicca il menù a tendina “File”;
		\item Utente clicca il pulsante “Esporta”;
		\item Utente sceglie la categoria dei contatti da esportare
		\item Utente sceglie l’estensione del file tra l’opzione \texttt{.csv} e \texttt{.vCard};
		\item Utente sceglie il percorso;
 		\item Utente sceglie il percorso dove salvare il file;
		\item Utente inserisce il nome del file da lui desiderato;
		\item Utente salva l’operazione;
		\item Il sistema produce il file con l’estensione scelta.		
	\end{enumerate}
	\textbf{Flusso di eventi alternativo}:
	\begin{itemize}[noitemsep, topsep=0pt]
		\item[3a. ] Utente chiude il pop-up;
		\item[3a.1] L’esecuzione riprende al passo 1;
		\item[4a. ] Utente chiude il pop-up;
		\item[4a.1] L’esecuzione riprende al passo 1;
		\item[5a. ] Utente chiude il pop-up;
		\item[5a.1] L’esecuzione riprende al passo 1;
		\item[6a. ] Utente decide di non fornire il percorso;
		\item[6a.1] L’esecuzione riprende al passo 3;
		\item[7a. ] Utente chiude il pop-up;
		\item[7a.1] L’esecuzione riprende al passo 1;
		\item[8a. ] Utente chiude il pop-up;
		\item[8a.1] L’esecuzione riprende al passo 1;		
	\end{itemize}
\end{tcolorbox}

\begin{tcolorbox}[colback=white,colframe=black!80!white,title=\textbf{C6 - Visualizzare rubrica}]
	\textbf{Attore partecipante}: Utente
	\\\textbf{Precondizioni}: Utente ha la schermata della rubrica aperta.
	\\\textbf{Postcondizioni}: Visualizzare i contatti della rubrica.
	\\\textbf{Flusso di eventi normale}:
	\begin{enumerate}[noitemsep, topsep=0pt]
		\item Utente visualizza tutti gli eventuali contatti per cognome in ordine alfabetico.
	\end{enumerate}
	\textbf{Flusso di eventi alternativo}:
	\begin{itemize}[noitemsep, topsep=0pt]
		\item[1a. ] Utente clicca il pulsante dell’imbuto \includegraphics[height=0.4cm]{images/imbuto_icona.jpeg};
		\item[1a.1]	Utente sceglie 1 o più Tag tra quelli presenti;
		\item[1a.2]	Utente visualizza i contatti associati al/ai Tag selezionato/i.
		\item[1b. ] Utente clicca il pulsante dell’imbuto \includegraphics[height=0.4cm]{images/imbuto_icona.jpeg};
		\item[1b.1]	Utente sceglie l’ordine alfabetico inverso;
		\item[1b.2]	Utente visualizza i contatti in ordine alfabetico inverso.
		\item[1c. ] Utente clicca il pulsante dell’imbuto \includegraphics[height=0.4cm]{images/imbuto_icona.jpeg};
		\item[1c.1]	Utente sceglie l’ordine per nome;
		\item[1c.2]	Utente visualizza i contatti in ordine per nome.
	\end{itemize}
\end{tcolorbox}

\begin{tcolorbox}[colback=white,colframe=black!80!white,title=\textbf{C7 - Salvare rubrica}]
	\textbf{Attore partecipante}: Sistema
	\\\textbf{Precondizioni}: Utente ha effettuato una modifica. 
	\\\textbf{Postcondizioni}: Il Sistema salva la modifica.
	\\\textbf{Flusso di eventi normale}:
	\begin{enumerate}[noitemsep, topsep=0pt]
		\item Utente effettua una qualsiasi modifica alla sua rubrica (aggiunge contatto, elimina Tag, importa un file di una rubrica, …); 
		\item La modifica viene salvata in locale in un file binario.
	\end{enumerate}
	\textbf{Flusso di eventi alternativo}:
	\begin{itemize}[noitemsep, topsep=0pt]
		\item[2a.] La modifica viene salvata sul database.
	\end{itemize}
\end{tcolorbox}

\begin{tcolorbox}[colback=white,colframe=black!80!white,title=\textbf{C8 - Aggiungere Tag}]
	\textbf{Attore partecipante}: Utente
	\\\textbf{Precondizioni}: Utente ha la schermata della rubrica aperta.
	\\\textbf{Postcondizioni}: 1 o più Tag sono stati aggiunti.
	\\\textbf{Flusso di eventi normale}:
	\begin{enumerate}[noitemsep, topsep=0pt]
		\item Utente clicca il menù a tendina “Rubrica”;
	\item	Utente clicca il pulsante “Gestisci Tag”;
	\item	Utente aggiunge un nuovo Tag;
	\item	Utente salva l’operazione.		
	\end{enumerate}
	\textbf{Flusso di eventi alternativo}:
	\begin{itemize}[noitemsep, topsep=0pt]
		\item[4a. ] Utente annulla l’operazione;
		\item[4a.1] L’esecuzione riprende al passo 1;		
	\end{itemize}
\end{tcolorbox}

\begin{tcolorbox}[colback=white,colframe=black!80!white,title=\textbf{C9 - Modificare Tag}]
	\textbf{Attore partecipante}: Utente
	\\\textbf{Precondizioni}: Utente ha la schermata della rubrica aperta. 
	\\\textbf{Postcondizioni}: 1 o più Tag sono stati modificati.
	\\\textbf{Flusso di eventi normale}:
	\begin{enumerate}[noitemsep, topsep=0pt]
		\item Utente clicca il menù a tendina “Rubrica”;
		\item Utente clicca il pulsante “Gestisci Tag”;
	\item 	Utente modifica un Tag precedentemente aggiunto;
	\item 	Utente salva l’operazione.		
	\end{enumerate}
	\textbf{Flusso di eventi alternativo}:
	\begin{itemize}[noitemsep, topsep=0pt]
		\item[4a. ] Utente annulla l’operazione;
		\item[4a.1] L’esecuzione riprende al passo 1;		
	\end{itemize}
\end{tcolorbox}

\begin{tcolorbox}[colback=white,colframe=black!80!white,title=\textbf{C10 - Eliminare Tag}]
	\textbf{Attore partecipante}: Utente
	\\\textbf{Precondizioni}: Utente ha la schermata della rubrica aperta.
	\\\textbf{Postcondizioni}: 1 o più Tag sono stati eliminati.
	\\\textbf{Flusso di eventi normale}:
	\begin{enumerate}[noitemsep, topsep=0pt]
		\item Utente clicca il menù a tendina “Rubrica”;
	\item	Utente clicca il pulsante “Gestisci Tag”;
	\item	Utente seleziona il Tag che vuole eliminare;
	\item	Utente conferma l’operazione dalla schermata di conferma;
	\item	Il sistema elimina il Tag.
	\end{enumerate}
	\textbf{Flusso di eventi alternativo}:
	\begin{itemize}[noitemsep, topsep=0pt]
		\item[4a. ] Utente annulla l’operazione dalla schermata di conferma;
		\item[4a.1] L’esecuzione riprende dal passo 1;		
	\end{itemize}
\end{tcolorbox}

\subsection{Diagramma dei Casi d'Uso}
	
	\newpage
	\section{Design}
	\subsection{Diagrammi di classi}
I diagrammi di classi rappresentano il sistema della rubrica con le funzionalità di gestione dei contatti, dei tag, importazione/esportazione di dati, configurazioni e salvataggio dei dati in locale o su DB.
\begin{itemize}[noitemsep, topsep=5pt]
	\item \textbf{Contact}: Modella un contatto con attributi come nome, cognome, numeri di telefono, e-mail, immagine del profilo, e tag associati. Ha metodi per gestire i contatti, aggiungere e rimuovere numeri e e-mail, e per gestire i tag associati.
	\item \textbf{Tag}: Modella un'etichetta associata ai contatti, con attributi come id, descrizione, e un indice statico. Include metodi per ottenere e impostare descrizioni e id, e per confrontare gli oggetti.
	\item \textbf{AddressBook}: Contiene contatti e tag. È una classe singleton, quindi ha un'istanza unica, e gestisce il salvataggio dei dati tramite un database o la serializzazione. Ha metodi per aggiungere, rimuovere e ottenere contatti e tag, e per caricare e salvare i dati.
	\item \textbf{MainController}: Gestisce l'interfaccia utente principale, con metodi per inizializzare i componenti dell'interfaccia, aggiungere, modificare, rimuovere contatti, per visualizzare pop-up per l'importazione/esportazione e la gestione dei tag.
	\item \textbf{ImportPopupController}, \textbf{ExportPopupController}, \textbf{ManageTagsPopupController}, \textbf{ConfigPopupController}, \textbf{ImagePopupController}: Gestiscono i pop-up per importare/esportare contatti, gestire i tag, gestire la configurazione (link del DB) e gestire le immagini di profilo.
	\item \textbf{ConfirmPopupController}: Gestisce anch'essa un pop-up di conferma di un'eventuale operazione di eliminazione di un contatto o di un tag richiesta dall'utente.
	\item \textbf{Database}: Gestisce l'interazione con un database MongoDB per l'archiviazione di contatti e tag. Ha metodi per inserire, aggiornare, rimuovere, recuperare contatti e tag dal database.
	\item \textbf{Converter}: Classe di utilità finalizzata a determinate conversioni per Import ed Export.
\end{itemize}

\subsubsection{Livello di dettaglio basso}
\definecolor{plantucolor0000}{RGB}{241,241,241}
\definecolor{plantucolor0001}{RGB}{24,24,24}
\definecolor{plantucolor0002}{RGB}{173,209,178}
\definecolor{plantucolor0003}{RGB}{0,0,0}
\definecolor{plantucolor0004}{RGB}{200,41,48}
\definecolor{plantucolor0005}{RGB}{132,190,132}
\definecolor{plantucolor0006}{RGB}{3,128,72}
\definecolor{plantucolor0007}{RGB}{242,77,92}
\definecolor{plantucolor0008}{RGB}{180,167,229}

\begin{adjustbox}{width=.95\paperwidth, center}
	\resizebox{\textwidth}{!}{
\begin{tikzpicture}[yscale=-1
,pstyle0/.style={color=plantucolor0001,fill=plantucolor0000,line width=0.5pt}
,pstyle1/.style={color=plantucolor0001,fill=plantucolor0002,line width=1.0pt}
,pstyle2/.style={color=plantucolor0001,line width=0.5pt}
,pstyle3/.style={color=plantucolor0004,line width=1.0pt}
,pstyle4/.style={color=plantucolor0006,fill=plantucolor0005,line width=1.0pt}
,pstyle5/.style={color=plantucolor0004,fill=plantucolor0007,line width=1.0pt}
,pstyle6/.style={color=plantucolor0001,fill=plantucolor0008,line width=1.0pt}
,pstyle7/.style={color=plantucolor0001,line width=1.0pt,dash pattern=on 7.0pt off 7.0pt}
,pstyle8/.style={color=plantucolor0001,line width=1.0pt}
,pstyle9/.style={color=plantucolor0001,fill=plantucolor0001,line width=1.0pt}
]
\draw[pstyle0] (142.5pt,1012pt) arc (180:270:5pt) -- (147.5pt,1007pt) -- (326.4151pt,1007pt) arc (270:360:5pt) -- (331.4151pt,1012pt) -- (331.4151pt,1174.2227pt) arc (0:90:5pt) -- (326.4151pt,1179.2227pt) -- (147.5pt,1179.2227pt) arc (90:180:5pt) -- (142.5pt,1174.2227pt) -- cycle;
\draw[pstyle1] (206.6712pt,1023pt) ellipse (11pt and 11pt);
\node at (206.6712pt,1023pt)[]{\textbf{\Large C}};
\node at (227.1712pt,1014.127pt)[below right,color=black]{Contact};
\draw[pstyle2] (143.5pt,1039pt) -- (330.4151pt,1039pt);
\draw[pstyle3] (150.5pt,1050.373pt) rectangle (156.5pt,1056.373pt);
\node at (162.5pt,1043pt)[below right,color=black]{String name};
\draw[pstyle3] (150.5pt,1068.1191pt) rectangle (156.5pt,1074.1191pt);
\node at (162.5pt,1060.7461pt)[below right,color=black]{String surname};
\draw[pstyle3] (150.5pt,1085.8652pt) rectangle (156.5pt,1091.8652pt);
\node at (162.5pt,1078.4922pt)[below right,color=black]{String[] numbers};
\draw[pstyle3] (150.5pt,1103.6113pt) rectangle (156.5pt,1109.6113pt);
\node at (162.5pt,1096.2383pt)[below right,color=black]{String[] emails};
\draw[pstyle3] (150.5pt,1121.3574pt) rectangle (156.5pt,1127.3574pt);
\node at (162.5pt,1113.9844pt)[below right,color=black]{Byte[] profilePicture};
\draw[pstyle3] (150.5pt,1139.1035pt) rectangle (156.5pt,1145.1035pt);
\node at (162.5pt,1131.7305pt)[below right,color=black]{Set\textless Integer\textgreater  tagIndexes};
\draw[pstyle3] (150.5pt,1156.8496pt) rectangle (156.5pt,1162.8496pt);
\node at (162.5pt,1149.4766pt)[below right,color=black]{int id};
\draw[pstyle2] (143.5pt,1171.2227pt) -- (330.4151pt,1171.2227pt);
\draw[pstyle0] (7pt,333pt) arc (180:270:5pt) -- (12pt,328pt) -- (259.5265pt,328pt) arc (270:360:5pt) -- (264.5265pt,333pt) -- (264.5265pt,566.207pt) arc (0:90:5pt) -- (259.5265pt,571.207pt) -- (12pt,571.207pt) arc (90:180:5pt) -- (7pt,566.207pt) -- cycle;
\draw[pstyle1] (86.6733pt,344pt) ellipse (11pt and 11pt);
\node at (86.6733pt,344pt)[]{\textbf{\Large C}};
\node at (107.1733pt,335.127pt)[below right,color=black]{AddressBook};
\draw[pstyle2] (8pt,360pt) -- (263.5265pt,360pt);
\draw[pstyle3] (15pt,371.373pt) rectangle (21pt,377.373pt);
\node at (27pt,364pt)[below right,color=black]{\underline{AddressBook instance}};
\draw[pstyle3] (15pt,389.1191pt) rectangle (21pt,395.1191pt);
\node at (27pt,381.7461pt)[below right,color=black]{ObservableList\textless Contact\textgreater  contacts};
\draw[pstyle3] (15pt,406.8652pt) rectangle (21pt,412.8652pt);
\node at (27pt,399.4922pt)[below right,color=black]{ObservableSet\textless String\textgreater  tags};
\draw[pstyle3] (15pt,424.6113pt) rectangle (21pt,430.6113pt);
\node at (27pt,417.2383pt)[below right,color=black]{String dbUrl};
\draw[pstyle3] (15pt,442.3574pt) rectangle (21pt,448.3574pt);
\node at (27pt,434.9844pt)[below right,color=black]{Database db};
\draw[pstyle2] (8pt,456.7305pt) -- (263.5265pt,456.7305pt);
\draw[pstyle4] (18pt,471.1035pt) ellipse (3pt and 3pt);
\node at (27pt,460.7305pt)[below right,color=black]{void saveConfig()};
\draw[pstyle4] (18pt,488.8496pt) ellipse (3pt and 3pt);
\node at (27pt,478.4766pt)[below right,color=black]{void loadConfig()};
\draw[pstyle4] (18pt,506.5957pt) ellipse (3pt and 3pt);
\node at (27pt,496.2227pt)[below right,color=black]{void saveOBJ()};
\draw[pstyle4] (18pt,524.3418pt) ellipse (3pt and 3pt);
\node at (27pt,513.9688pt)[below right,color=black]{AddressBook loadOBJ()};
\draw[pstyle4] (18pt,542.0879pt) ellipse (3pt and 3pt);
\node at (27pt,531.7148pt)[below right,color=black]{void saveToDB()};
\draw[pstyle4] (18pt,559.834pt) ellipse (3pt and 3pt);
\node at (27pt,549.4609pt)[below right,color=black]{AddressBook loadFromDB()};
\draw[pstyle0] (938pt,12pt) arc (180:270:5pt) -- (943pt,7pt) -- (1169.3435pt,7pt) arc (270:360:5pt) -- (1174.3435pt,12pt) -- (1174.3435pt,262.9531pt) arc (0:90:5pt) -- (1169.3435pt,267.9531pt) -- (943pt,267.9531pt) arc (90:180:5pt) -- (938pt,262.9531pt) -- cycle;
\draw[pstyle1] (1002.0218pt,23pt) ellipse (11pt and 11pt);
\node at (1002.0218pt,23pt)[]{\textbf{\Large C}};
\node at (1022.5218pt,14.127pt)[below right,color=black]{MainController};
\draw[pstyle2] (939pt,39pt) -- (1173.3435pt,39pt);
\draw[pstyle3] (946pt,50.373pt) rectangle (952pt,56.373pt);
\node at (958pt,43pt)[below right,color=black]{AddressBook addressBook};
\draw[pstyle2] (939pt,64.7461pt) -- (1173.3435pt,64.7461pt);
\draw[pstyle5] (946pt,76.1191pt) rectangle (952pt,82.1191pt);
\node at (958pt,68.7461pt)[below right,color=black]{void onAddContact()};
\draw[pstyle5] (946pt,93.8652pt) rectangle (952pt,99.8652pt);
\node at (958pt,86.4922pt)[below right,color=black]{void onContactClicked()};
\draw[pstyle5] (946pt,111.6113pt) rectangle (952pt,117.6113pt);
\node at (958pt,104.2383pt)[below right,color=black]{void onSaveContact()};
\draw[pstyle5] (946pt,129.3574pt) rectangle (952pt,135.3574pt);
\node at (958pt,121.9844pt)[below right,color=black]{void onModifyContact()};
\draw[pstyle5] (946pt,147.1035pt) rectangle (952pt,153.1035pt);
\node at (958pt,139.7305pt)[below right,color=black]{void onRemoveContact()};
\draw[pstyle5] (946pt,164.8496pt) rectangle (952pt,170.8496pt);
\node at (958pt,157.4766pt)[below right,color=black]{void onCancel()};
\draw[pstyle5] (946pt,182.5957pt) rectangle (952pt,188.5957pt);
\node at (958pt,175.2227pt)[below right,color=black]{void showImportPopup()};
\draw[pstyle5] (946pt,200.3418pt) rectangle (952pt,206.3418pt);
\node at (958pt,192.9688pt)[below right,color=black]{void showExportPopup()};
\draw[pstyle5] (946pt,218.0879pt) rectangle (952pt,224.0879pt);
\node at (958pt,210.7148pt)[below right,color=black]{void showConfigPopup()};
\draw[pstyle5] (946pt,235.834pt) rectangle (952pt,241.834pt);
\node at (958pt,228.4609pt)[below right,color=black]{void showImagePopup()};
\draw[pstyle5] (946pt,253.5801pt) rectangle (952pt,259.5801pt);
\node at (958pt,246.207pt)[below right,color=black]{void showManageTagsPopup()};
\draw[pstyle0] (860.5pt,833pt) arc (180:270:5pt) -- (865.5pt,828pt) -- (1110.598pt,828pt) arc (270:360:5pt) -- (1115.598pt,833pt) -- (1115.598pt,941.9844pt) arc (0:90:5pt) -- (1110.598pt,946.9844pt) -- (865.5pt,946.9844pt) arc (90:180:5pt) -- (860.5pt,941.9844pt) -- cycle;
\draw[pstyle1] (950.8418pt,844pt) ellipse (11pt and 11pt);
\node at (950.8418pt,844pt)[]{\textbf{\Large C}};
\node at (971.3418pt,835.127pt)[below right,color=black]{Converter};
\draw[pstyle2] (861.5pt,860pt) -- (1114.598pt,860pt);
\draw[pstyle2] (861.5pt,868pt) -- (1114.598pt,868pt);
\draw[pstyle4] (871.5pt,882.373pt) ellipse (3pt and 3pt);
\node at (880.5pt,872pt)[below right,color=black]{\underline{Collection\textless Contact\textgreater  parseCSV()}};
\draw[pstyle4] (871.5pt,900.1191pt) ellipse (3pt and 3pt);
\node at (880.5pt,889.7461pt)[below right,color=black]{\underline{Collection\textless Contact\textgreater  parseVCard()}};
\draw[pstyle4] (871.5pt,917.8652pt) ellipse (3pt and 3pt);
\node at (880.5pt,907.4922pt)[below right,color=black]{\underline{void onExportCSV()}};
\draw[pstyle4] (871.5pt,935.6113pt) ellipse (3pt and 3pt);
\node at (880.5pt,925.2383pt)[below right,color=black]{\underline{void onExportVCard()}};
\draw[pstyle0] (1076.5pt,413pt) arc (180:270:5pt) -- (1081.5pt,408pt) -- (1320.4191pt,408pt) arc (270:360:5pt) -- (1325.4191pt,413pt) -- (1325.4191pt,486.4922pt) arc (0:90:5pt) -- (1320.4191pt,491.4922pt) -- (1081.5pt,491.4922pt) arc (90:180:5pt) -- (1076.5pt,486.4922pt) -- cycle;
\draw[pstyle1] (1119.5828pt,424pt) ellipse (11pt and 11pt);
\node at (1119.5828pt,424pt)[]{\textbf{\Large C}};
\node at (1139.8234pt,415.127pt)[below right,color=black]{ImportPopupController};
\draw[pstyle2] (1077.5pt,440pt) -- (1324.4191pt,440pt);
\draw[pstyle3] (1084.5pt,451.373pt) rectangle (1090.5pt,457.373pt);
\node at (1096.5pt,444pt)[below right,color=black]{ContactManager contactManager};
\draw[pstyle2] (1077.5pt,465.7461pt) -- (1324.4191pt,465.7461pt);
\draw[pstyle5] (1084.5pt,477.1191pt) rectangle (1090.5pt,483.1191pt);
\node at (1096.5pt,469.7461pt)[below right,color=black]{void onImport()};
\draw[pstyle0] (792.5pt,404pt) arc (180:270:5pt) -- (797.5pt,399pt) -- (1036.4191pt,399pt) arc (270:360:5pt) -- (1041.4191pt,404pt) -- (1041.4191pt,495.2383pt) arc (0:90:5pt) -- (1036.4191pt,500.2383pt) -- (797.5pt,500.2383pt) arc (90:180:5pt) -- (792.5pt,495.2383pt) -- cycle;
\draw[pstyle1] (836.2428pt,415pt) ellipse (11pt and 11pt);
\node at (836.2428pt,415pt)[]{\textbf{\Large C}};
\node at (856.6301pt,406.127pt)[below right,color=black]{ExportPopupController};
\draw[pstyle2] (793.5pt,431pt) -- (1040.4191pt,431pt);
\draw[pstyle3] (800.5pt,442.373pt) rectangle (806.5pt,448.373pt);
\node at (812.5pt,435pt)[below right,color=black]{ContactManager contactManager};
\draw[pstyle3] (800.5pt,460.1191pt) rectangle (806.5pt,466.1191pt);
\node at (812.5pt,452.7461pt)[below right,color=black]{TagManager tagManager};
\draw[pstyle2] (793.5pt,474.4922pt) -- (1040.4191pt,474.4922pt);
\draw[pstyle5] (800.5pt,485.8652pt) rectangle (806.5pt,491.8652pt);
\node at (812.5pt,478.4922pt)[below right,color=black]{void onExport()};
\draw[pstyle0] (300.5pt,395pt) arc (180:270:5pt) -- (305.5pt,390pt) -- (522.7146pt,390pt) arc (270:360:5pt) -- (527.7146pt,395pt) -- (527.7146pt,503.9844pt) arc (0:90:5pt) -- (522.7146pt,508.9844pt) -- (305.5pt,508.9844pt) arc (90:180:5pt) -- (300.5pt,503.9844pt) -- cycle;
\draw[pstyle1] (315.5pt,406pt) ellipse (11pt and 11pt);
\node at (315.5pt,406pt)[]{\textbf{\Large C}};
\node at (329.5pt,397.127pt)[below right,color=black]{ManageTagsPopupController};
\draw[pstyle2] (301.5pt,422pt) -- (526.7146pt,422pt);
\draw[pstyle3] (308.5pt,433.373pt) rectangle (314.5pt,439.373pt);
\node at (320.5pt,426pt)[below right,color=black]{TagManager tagManager};
\draw[pstyle2] (301.5pt,447.7461pt) -- (526.7146pt,447.7461pt);
\draw[pstyle5] (308.5pt,459.1191pt) rectangle (314.5pt,465.1191pt);
\node at (320.5pt,451.7461pt)[below right,color=black]{void onAdd()};
\draw[pstyle5] (308.5pt,476.8652pt) rectangle (314.5pt,482.8652pt);
\node at (320.5pt,469.4922pt)[below right,color=black]{void onUpdate()};
\draw[pstyle5] (308.5pt,494.6113pt) rectangle (314.5pt,500.6113pt);
\node at (320.5pt,487.2383pt)[below right,color=black]{void onDelete()};
\draw[pstyle0] (1361pt,421.5pt) arc (180:270:5pt) -- (1366pt,416.5pt) -- (1537.7543pt,416.5pt) arc (270:360:5pt) -- (1542.7543pt,421.5pt) -- (1542.7543pt,477.2461pt) arc (0:90:5pt) -- (1537.7543pt,482.2461pt) -- (1366pt,482.2461pt) arc (90:180:5pt) -- (1361pt,477.2461pt) -- cycle;
\draw[pstyle1] (1376pt,432.5pt) ellipse (11pt and 11pt);
\node at (1376pt,432.5pt)[]{\textbf{\Large C}};
\node at (1390pt,423.627pt)[below right,color=black]{ImagePopupController};
\draw[pstyle2] (1362pt,448.5pt) -- (1541.7543pt,448.5pt);
\draw[pstyle2] (1362pt,456.5pt) -- (1541.7543pt,456.5pt);
\draw[pstyle4] (1372pt,470.873pt) ellipse (3pt and 3pt);
\node at (1381pt,460.5pt)[below right,color=black]{File getSelectedImage()};
\draw[pstyle0] (562.5pt,413pt) arc (180:270:5pt) -- (567.5pt,408pt) -- (752.9331pt,408pt) arc (270:360:5pt) -- (757.9331pt,413pt) -- (757.9331pt,486.4922pt) arc (0:90:5pt) -- (752.9331pt,491.4922pt) -- (567.5pt,491.4922pt) arc (90:180:5pt) -- (562.5pt,486.4922pt) -- cycle;
\draw[pstyle1] (577.5pt,424pt) ellipse (11pt and 11pt);
\node at (577.5pt,424pt)[]{\textbf{\Large C}};
\node at (591.5pt,415.127pt)[below right,color=black]{ConfirmPopupController};
\draw[pstyle2] (563.5pt,440pt) -- (756.9331pt,440pt);
\draw[pstyle2] (563.5pt,448pt) -- (756.9331pt,448pt);
\draw[pstyle4] (573.5pt,462.373pt) ellipse (3pt and 3pt);
\node at (582.5pt,452pt)[below right,color=black]{onConfirm()};
\draw[pstyle4] (573.5pt,480.1191pt) ellipse (3pt and 3pt);
\node at (582.5pt,469.7461pt)[below right,color=black]{onCancel()};
\draw[pstyle0] (33pt,92pt) arc (180:270:5pt) -- (38pt,87pt) -- (233.8225pt,87pt) arc (270:360:5pt) -- (238.8225pt,92pt) -- (238.8225pt,183.2383pt) arc (0:90:5pt) -- (233.8225pt,188.2383pt) -- (38pt,188.2383pt) arc (90:180:5pt) -- (33pt,183.2383pt) -- cycle;
\draw[pstyle1] (56.8509pt,103pt) ellipse (11pt and 11pt);
\node at (56.8509pt,103pt)[]{\textbf{\Large C}};
\node at (72.8178pt,94.127pt)[below right,color=black]{ConfigPopupController};
\draw[pstyle2] (34pt,119pt) -- (237.8225pt,119pt);
\draw[pstyle3] (41pt,130.373pt) rectangle (47pt,136.373pt);
\node at (53pt,123pt)[below right,color=black]{AddressBook addressBook};
\draw[pstyle2] (34pt,144.7461pt) -- (237.8225pt,144.7461pt);
\draw[pstyle5] (41pt,156.1191pt) rectangle (47pt,162.1191pt);
\node at (53pt,148.7461pt)[below right,color=black]{void onVerify()};
\draw[pstyle5] (41pt,173.8652pt) rectangle (47pt,179.8652pt);
\node at (53pt,166.4922pt)[below right,color=black]{void onConfirm()};
\draw[pstyle0] (16.5pt,851pt) arc (180:270:5pt) -- (21.5pt,846pt) -- (156.799pt,846pt) arc (270:360:5pt) -- (161.799pt,851pt) -- (161.799pt,924.4922pt) arc (0:90:5pt) -- (156.799pt,929.4922pt) -- (21.5pt,929.4922pt) arc (90:180:5pt) -- (16.5pt,924.4922pt) -- cycle;
\draw[pstyle1] (71.8813pt,862pt) ellipse (11pt and 11pt);
\node at (71.8813pt,862pt)[]{\textbf{\Large C}};
\node at (92.3813pt,853.127pt)[below right,color=black]{Tag};
\draw[pstyle2] (17.5pt,878pt) -- (160.799pt,878pt);
\draw[pstyle3] (24.5pt,889.373pt) rectangle (30.5pt,895.373pt);
\node at (36.5pt,882pt)[below right,color=black]{final int id};
\draw[pstyle3] (24.5pt,907.1191pt) rectangle (30.5pt,913.1191pt);
\node at (36.5pt,899.7461pt)[below right,color=black]{String descrizione};
\draw[pstyle2] (17.5pt,921.4922pt) -- (160.799pt,921.4922pt);
\draw[pstyle0] (268.5pt,636pt) arc (180:270:5pt) -- (273.5pt,631pt) -- (540.3151pt,631pt) arc (270:360:5pt) -- (545.3151pt,636pt) -- (545.3151pt,762.7305pt) arc (0:90:5pt) -- (540.3151pt,767.7305pt) -- (273.5pt,767.7305pt) arc (90:180:5pt) -- (268.5pt,762.7305pt) -- cycle;
\draw[pstyle1] (372.2928pt,647pt) ellipse (11pt and 11pt);
\node at (372.2928pt,647pt)[]{\textbf{\Large C}};
\node at (392.7928pt,638.127pt)[below right,color=black]{Database};
\draw[pstyle2] (269.5pt,663pt) -- (544.3151pt,663pt);
\draw[pstyle2] (269.5pt,671pt) -- (544.3151pt,671pt);
\draw[pstyle4] (279.5pt,685.373pt) ellipse (3pt and 3pt);
\node at (288.5pt,675pt)[below right,color=black]{\underline{boolean verifyDBUrl(String url)}};
\draw[pstyle4] (279.5pt,703.1191pt) ellipse (3pt and 3pt);
\node at (288.5pt,692.7461pt)[below right,color=black]{void insertContact(Contact c)};
\draw[pstyle4] (279.5pt,720.8652pt) ellipse (3pt and 3pt);
\node at (288.5pt,710.4922pt)[below right,color=black]{void removeTag(Tag tag)};
\draw[pstyle4] (279.5pt,738.6113pt) ellipse (3pt and 3pt);
\node at (288.5pt,728.2383pt)[below right,color=black]{Collection\textless Contact\textgreater  getAllContacts()};
\draw[pstyle4] (279.5pt,756.3574pt) ellipse (3pt and 3pt);
\node at (288.5pt,745.9844pt)[below right,color=black]{Collection\textless Tag\textgreater  getAllTags()};
\draw[pstyle0] (107.5pt,1244pt) arc (180:270:5pt) -- (112.5pt,1239pt) -- (213.7pt,1239pt) arc (270:360:5pt) -- (218.7pt,1244pt) -- (218.7pt,1282pt) arc (0:90:5pt) -- (213.7pt,1287pt) -- (112.5pt,1287pt) arc (90:180:5pt) -- (107.5pt,1282pt) -- cycle;
\draw[pstyle6] (122.5pt,1255pt) ellipse (11pt and 11pt);
\node at (122.5pt,1255pt)[]{\textbf{\Large I}};
\node at (136.5pt,1246.127pt)[below right,color=black]{\textit{Serializable}};
\draw[pstyle2] (108.5pt,1271pt) -- (217.7pt,1271pt);
\draw[pstyle2] (108.5pt,1279pt) -- (217.7pt,1279pt);
\draw[pstyle0] (581pt,680.5pt) arc (180:270:5pt) -- (586pt,675.5pt) -- (694.3945pt,675.5pt) arc (270:360:5pt) -- (699.3945pt,680.5pt) -- (699.3945pt,718.5pt) arc (0:90:5pt) -- (694.3945pt,723.5pt) -- (586pt,723.5pt) arc (90:180:5pt) -- (581pt,718.5pt) -- cycle;
\draw[pstyle6] (596pt,691.5pt) ellipse (11pt and 11pt);
\node at (596pt,691.5pt)[]{\textbf{\Large I}};
\node at (610pt,682.627pt)[below right,color=black]{\textit{TagManager}};
\draw[pstyle2] (582pt,707.5pt) -- (698.3945pt,707.5pt);
\draw[pstyle2] (582pt,715.5pt) -- (698.3945pt,715.5pt);
\draw[pstyle0] (789.5pt,680.5pt) arc (180:270:5pt) -- (794.5pt,675.5pt) -- (929.9526pt,675.5pt) arc (270:360:5pt) -- (934.9526pt,680.5pt) -- (934.9526pt,718.5pt) arc (0:90:5pt) -- (929.9526pt,723.5pt) -- (794.5pt,723.5pt) arc (90:180:5pt) -- (789.5pt,718.5pt) -- cycle;
\draw[pstyle6] (804.5pt,691.5pt) ellipse (11pt and 11pt);
\node at (804.5pt,691.5pt)[]{\textbf{\Large I}};
\node at (818.5pt,682.627pt)[below right,color=black]{\textit{ContactManager}};
\draw[pstyle2] (790.5pt,707.5pt) -- (933.9526pt,707.5pt);
\draw[pstyle2] (790.5pt,715.5pt) -- (933.9526pt,715.5pt);
\draw[pstyle7] (180.5pt,1179.49pt) ..controls (180.5pt,1201.31pt) and (180.5pt,1205pt) .. (180.5pt,1220.8pt);
\draw[pstyle8] (180.5pt,1238.8pt) -- (186.5pt,1220.8pt) -- (174.5pt,1220.8pt) -- (180.5pt,1238.8pt) -- cycle;
\draw[pstyle7] (125pt,929.12pt) ..controls (125pt,1007.79pt) and (125pt,1158.05pt) .. (125pt,1220.77pt);
\draw[pstyle8] (125pt,1238.77pt) -- (131pt,1220.77pt) -- (119pt,1220.77pt) -- (125pt,1238.77pt) -- cycle;
\draw[pstyle8] (213.25pt,583.29pt) ..controls (213.25pt,710.46pt) and (213.25pt,894.66pt) .. (213.25pt,1006.7pt);
\draw[pstyle8] (213.25pt,571.29pt) -- (209.25pt,577.29pt) -- (213.25pt,583.29pt) -- (217.25pt,577.29pt) -- (213.25pt,571.29pt) -- cycle;
\node at (210.7647pt,579.2073pt)[below right,color=black]{1};
\node at (189.3514pt,982.6071pt)[below right,color=black]{0..*};
\draw[pstyle8] (277.09pt,556pt) ..controls (288.88pt,556pt) and (284.5pt,556pt) .. (284.5pt,556pt) ..controls (284.5pt,556pt) and (284.5pt,593.86pt) .. (284.5pt,630.69pt);
\draw[pstyle9] (265.09pt,556pt) -- (271.09pt,560pt) -- (277.09pt,556pt) -- (271.09pt,552pt) -- (265.09pt,556pt) -- cycle;
\node at (272.3594pt,545.3941pt)[below right,color=black]{1};
\node at (248.996pt,606.3298pt)[below right,color=black]{0..1};
\draw[pstyle8] (89pt,583.22pt) ..controls (89pt,674.98pt) and (89pt,783.35pt) .. (89pt,845.82pt);
\draw[pstyle8] (89pt,571.22pt) -- (85pt,577.22pt) -- (89pt,583.22pt) -- (93pt,577.22pt) -- (89pt,571.22pt) -- cycle;
\node at (78.9432pt,579.1331pt)[below right,color=black]{1};
\node at (66.3155pt,821.5906pt)[below right,color=black]{0..*};
\draw[pstyle7] (265.03pt,540pt) ..controls (405.54pt,540pt) and (610.5pt,540pt) .. (610.5pt,540pt) ..controls (610.5pt,540pt) and (610.5pt,612.66pt) .. (610.5pt,657.28pt);
\draw[pstyle8] (610.5pt,675.28pt) -- (616.5pt,657.28pt) -- (604.5pt,657.28pt) -- (610.5pt,675.28pt) -- cycle;
\draw[pstyle7] (265.09pt,525pt) ..controls (468.28pt,525pt) and (839.83pt,525pt) .. (839.83pt,525pt) ..controls (839.83pt,525pt) and (839.83pt,609.29pt) .. (839.83pt,657.24pt);
\draw[pstyle8] (839.83pt,675.24pt) -- (845.83pt,657.24pt) -- (833.83pt,657.24pt) -- (839.83pt,675.24pt) -- cycle;
\draw[pstyle8] (925.8pt,215pt) ..controls (702.32pt,215pt) and (252pt,215pt) .. (252pt,215pt) ..controls (252pt,215pt) and (252pt,270.69pt) .. (252pt,327.79pt);
\draw[pstyle8] (937.8pt,215pt) -- (931.8pt,211pt) -- (925.8pt,215pt) -- (931.8pt,219pt) -- (937.8pt,215pt) -- cycle;
\node at (922.6185pt,203.6872pt)[below right,color=black]{1};
\node at (252.2286pt,303.4891pt)[below right,color=black]{1};
\draw[pstyle7] (1174.01pt,138pt) ..controls (1290.72pt,138pt) and (1452pt,138pt) .. (1452pt,138pt) ..controls (1452pt,138pt) and (1452pt,330.29pt) .. (1452pt,410.46pt);
\draw[pstyle9] (1452pt,416.46pt) -- (1456pt,407.46pt) -- (1452pt,411.46pt) -- (1448pt,407.46pt) -- (1452pt,416.46pt) -- cycle;
\draw[pstyle7] (937.99pt,242pt) ..controls (821.28pt,242pt) and (660pt,242pt) .. (660pt,242pt) ..controls (660pt,242pt) and (660pt,340.88pt) .. (660pt,401.51pt);
\draw[pstyle9] (660pt,407.51pt) -- (664pt,398.51pt) -- (660pt,402.51pt) -- (656pt,398.51pt) -- (660pt,407.51pt) -- cycle;
\draw[pstyle7] (1059pt,268.14pt) ..controls (1059pt,432.24pt) and (1059pt,700.91pt) .. (1059pt,821.57pt);
\draw[pstyle9] (1059pt,827.57pt) -- (1063pt,818.57pt) -- (1059pt,822.57pt) -- (1055pt,818.57pt) -- (1059pt,827.57pt) -- cycle;
\draw[pstyle7] (860.2pt,938pt) ..controls (661.24pt,938pt) and (300pt,938pt) .. (300pt,938pt) ..controls (300pt,938pt) and (300pt,965.21pt) .. (300pt,1000.89pt);
\draw[pstyle9] (300pt,1006.89pt) -- (304pt,997.89pt) -- (300pt,1001.89pt) -- (296pt,997.89pt) -- (300pt,1006.89pt) -- cycle;
\draw[pstyle8] (1089.5pt,503.15pt) ..controls (1089.5pt,574.38pt) and (1089.5pt,700pt) .. (1089.5pt,700pt) ..controls (1089.5pt,700pt) and (1001.19pt,700pt) .. (934.61pt,700pt);
\draw[pstyle8] (1089.5pt,491.15pt) -- (1085.5pt,497.15pt) -- (1089.5pt,503.15pt) -- (1093.5pt,497.15pt) -- (1089.5pt,491.15pt) -- cycle;
\node at (1063.4764pt,499.3721pt)[below right,color=black]{1};
\node at (942.9207pt,683.2059pt)[below right,color=black]{1};
\draw[pstyle7] (1102.5pt,491.1pt) ..controls (1102.5pt,569.15pt) and (1102.5pt,731.77pt) .. (1102.5pt,821.88pt);
\draw[pstyle9] (1102.5pt,827.88pt) -- (1106.5pt,818.88pt) -- (1102.5pt,822.88pt) -- (1098.5pt,818.88pt) -- (1102.5pt,827.88pt) -- cycle;
\draw[pstyle8] (887.17pt,512.09pt) ..controls (887.17pt,565.45pt) and (887.17pt,635.26pt) .. (887.17pt,675.48pt);
\draw[pstyle8] (887.17pt,500.09pt) -- (883.17pt,506.09pt) -- (887.17pt,512.09pt) -- (891.17pt,506.09pt) -- (887.17pt,500.09pt) -- cycle;
\node at (875.5027pt,508.3461pt)[below right,color=black]{1};
\node at (885.9599pt,650.9389pt)[below right,color=black]{1};
\draw[pstyle8] (780.16pt,496pt) ..controls (718.65pt,496pt) and (669.5pt,496pt) .. (669.5pt,496pt) ..controls (669.5pt,496pt) and (669.5pt,621.45pt) .. (669.5pt,675.36pt);
\draw[pstyle8] (792.16pt,496pt) -- (786.16pt,492pt) -- (780.16pt,496pt) -- (786.16pt,500pt) -- (792.16pt,496pt) -- cycle;
\node at (777.3478pt,482.3534pt)[below right,color=black]{1};
\node at (669.1376pt,651.1854pt)[below right,color=black]{1};
\draw[pstyle7] (988pt,500.17pt) ..controls (988pt,581.58pt) and (988pt,735.12pt) .. (988pt,821.88pt);
\draw[pstyle9] (988pt,827.88pt) -- (992pt,818.88pt) -- (988pt,822.88pt) -- (984pt,818.88pt) -- (988pt,827.88pt) -- cycle;
\draw[pstyle7] (527.64pt,450pt) ..controls (539.22pt,450pt) and (544.91pt,450pt) .. (556.3pt,450pt);
\draw[pstyle9] (562.3pt,450pt) -- (553.3pt,446pt) -- (557.3pt,450pt) -- (553.3pt,454pt) -- (562.3pt,450pt) -- cycle;
\draw[pstyle8] (539.66pt,505pt) ..controls (595.94pt,505pt) and (640pt,505pt) .. (640pt,505pt) ..controls (640pt,505pt) and (640pt,623.12pt) .. (640pt,675.26pt);
\draw[pstyle8] (527.66pt,505pt) -- (533.66pt,509pt) -- (539.66pt,505pt) -- (533.66pt,501pt) -- (527.66pt,505pt) -- cycle;
\node at (535.8973pt,492.3572pt)[below right,color=black]{1};
\node at (632.0782pt,651.0598pt)[below right,color=black]{1};
\draw[pstyle8] (136pt,200.32pt) ..controls (136pt,238.14pt) and (136pt,279.39pt) .. (136pt,327.95pt);
\draw[pstyle8] (136pt,188.32pt) -- (132pt,194.32pt) -- (136pt,200.32pt) -- (140pt,194.32pt) -- (136pt,188.32pt) -- cycle;
\node at (128.555pt,196.2401pt)[below right,color=black]{1};
\node at (128.7974pt,303.6594pt)[below right,color=black]{1};
\draw[pstyle7] (545.59pt,746pt) ..controls (690.64pt,746pt) and (897.5pt,746pt) .. (897.5pt,746pt) ..controls (897.5pt,746pt) and (897.5pt,783.23pt) .. (897.5pt,821.9pt);
\draw[pstyle9] (897.5pt,827.9pt) -- (901.5pt,818.9pt) -- (897.5pt,822.9pt) -- (893.5pt,818.9pt) -- (897.5pt,827.9pt) -- cycle;
\end{tikzpicture}
}
\end{adjustbox}
\begin{figure}[h]
	\caption{Diagramma classi essenziale}
	\label{fig:Diagramma classi essenziale}
\end{figure}
\newpage
\subsubsection{Livello di dettaglio alto}
\definecolor{plantucolor0000}{RGB}{241,241,241}
\definecolor{plantucolor0001}{RGB}{24,24,24}
\definecolor{plantucolor0002}{RGB}{173,209,178}
\definecolor{plantucolor0003}{RGB}{0,0,0}
\definecolor{plantucolor0004}{RGB}{200,41,48}
\definecolor{plantucolor0005}{RGB}{132,190,132}
\definecolor{plantucolor0006}{RGB}{3,128,72}
\definecolor{plantucolor0007}{RGB}{242,77,92}
\definecolor{plantucolor0008}{RGB}{180,167,229}

\begin{adjustbox}{width=.99\paperwidth, center}
	\resizebox{\textwidth}{!}{
		\begin{tikzpicture}[yscale=-1
,pstyle0/.style={color=plantucolor0001,fill=plantucolor0000,line width=0.5pt}
,pstyle1/.style={color=plantucolor0001,fill=plantucolor0002,line width=1.0pt}
,pstyle2/.style={color=plantucolor0001,line width=0.5pt}
,pstyle3/.style={color=plantucolor0004,line width=1.0pt}
,pstyle4/.style={color=plantucolor0006,fill=plantucolor0005,line width=1.0pt}
,pstyle5/.style={color=plantucolor0004,fill=plantucolor0007,line width=1.0pt}
,pstyle6/.style={color=plantucolor0001,fill=plantucolor0008,line width=1.0pt}
,pstyle7/.style={color=plantucolor0001,line width=1.0pt,dash pattern=on 7.0pt off 7.0pt}
,pstyle8/.style={color=plantucolor0001,line width=1.0pt}
,pstyle9/.style={color=plantucolor0001,fill=plantucolor0001,line width=1.0pt}
]
\draw[pstyle0] (1931pt,1534pt) arc (180:270:5pt) -- (1936pt,1529pt) -- (2212.7294pt,1529pt) arc (270:360:5pt) -- (2217.7294pt,1534pt) -- (2217.7294pt,1749.4609pt) arc (0:90:5pt) -- (2212.7294pt,1754.4609pt) -- (1936pt,1754.4609pt) arc (90:180:5pt) -- (1931pt,1749.4609pt) -- cycle;
\draw[pstyle1] (2057.0965pt,1545pt) ellipse (11pt and 11pt);
\node at (2057.0965pt,1545pt)[]{\textbf{\Large C}};
\node at (2077.5965pt,1536.127pt)[below right,color=black]{Tag};
\draw[pstyle2] (1932pt,1561pt) -- (2216.7294pt,1561pt);
\draw[pstyle3] (1939pt,1572.373pt) rectangle (1945pt,1578.373pt);
\node at (1951pt,1565pt)[below right,color=black]{final int id};
\draw[pstyle3] (1939pt,1590.1191pt) rectangle (1945pt,1596.1191pt);
\node at (1951pt,1582.7461pt)[below right,color=black]{String descrizione};
\draw[pstyle3] (1939pt,1607.8652pt) rectangle (1945pt,1613.8652pt);
\node at (1951pt,1600.4922pt)[below right,color=black]{\underline{int index}};
\draw[pstyle2] (1932pt,1622.2383pt) -- (2216.7294pt,1622.2383pt);
\draw[pstyle4] (1942pt,1636.6113pt) ellipse (3pt and 3pt);
\node at (1951pt,1626.2383pt)[below right,color=black]{Tag(String descrizione)};
\draw[pstyle4] (1942pt,1654.3574pt) ellipse (3pt and 3pt);
\node at (1951pt,1643.9844pt)[below right,color=black]{Tag(String description, int id)};
\draw[pstyle4] (1942pt,1672.1035pt) ellipse (3pt and 3pt);
\node at (1951pt,1661.7305pt)[below right,color=black]{int getId()};
\draw[pstyle4] (1942pt,1689.8496pt) ellipse (3pt and 3pt);
\node at (1951pt,1679.4766pt)[below right,color=black]{String getDescrizione()};
\draw[pstyle4] (1942pt,1707.5957pt) ellipse (3pt and 3pt);
\node at (1951pt,1697.2227pt)[below right,color=black]{void setDescrizione(String descrizione)};
\draw[pstyle4] (1942pt,1725.3418pt) ellipse (3pt and 3pt);
\node at (1951pt,1714.9688pt)[below right,color=black]{\underline{void setIndex(int index)}};
\draw[pstyle4] (1942pt,1743.0879pt) ellipse (3pt and 3pt);
\node at (1951pt,1732.7148pt)[below right,color=black]{boolean equals(Object o)};
\draw[pstyle0] (1671pt,1828pt) arc (180:270:5pt) -- (1676pt,1823pt) -- (1979.4602pt,1823pt) arc (270:360:5pt) -- (1984.4602pt,1828pt) -- (1984.4602pt,2309.6523pt) arc (0:90:5pt) -- (1979.4602pt,2314.6523pt) -- (1676pt,2314.6523pt) arc (90:180:5pt) -- (1671pt,2309.6523pt) -- cycle;
\draw[pstyle1] (1797.4437pt,1839pt) ellipse (11pt and 11pt);
\node at (1797.4437pt,1839pt)[]{\textbf{\Large C}};
\node at (1817.9437pt,1830.127pt)[below right,color=black]{Contact};
\draw[pstyle2] (1672pt,1855pt) -- (1983.4602pt,1855pt);
\draw[pstyle3] (1679pt,1866.373pt) rectangle (1685pt,1872.373pt);
\node at (1691pt,1859pt)[below right,color=black]{String name};
\draw[pstyle3] (1679pt,1884.1191pt) rectangle (1685pt,1890.1191pt);
\node at (1691pt,1876.7461pt)[below right,color=black]{String surname};
\draw[pstyle3] (1679pt,1901.8652pt) rectangle (1685pt,1907.8652pt);
\node at (1691pt,1894.4922pt)[below right,color=black]{String[] numbers};
\draw[pstyle3] (1679pt,1919.6113pt) rectangle (1685pt,1925.6113pt);
\node at (1691pt,1912.2383pt)[below right,color=black]{String[] emails};
\draw[pstyle3] (1679pt,1937.3574pt) rectangle (1685pt,1943.3574pt);
\node at (1691pt,1929.9844pt)[below right,color=black]{Byte[] profilePicture};
\draw[pstyle3] (1679pt,1955.1035pt) rectangle (1685pt,1961.1035pt);
\node at (1691pt,1947.7305pt)[below right,color=black]{Set\textless Integer\textgreater  tagIndexes};
\draw[pstyle3] (1679pt,1972.8496pt) rectangle (1685pt,1978.8496pt);
\node at (1691pt,1965.4766pt)[below right,color=black]{String id};
\draw[pstyle2] (1672pt,1987.2227pt) -- (1983.4602pt,1987.2227pt);
\draw[pstyle4] (1682pt,2001.5957pt) ellipse (3pt and 3pt);
\node at (1691pt,1991.2227pt)[below right,color=black]{Contact(String name, String surname)};
\draw[pstyle4] (1682pt,2019.3418pt) ellipse (3pt and 3pt);
\node at (1691pt,2008.9688pt)[below right,color=black]{String getId()};
\draw[pstyle4] (1682pt,2037.0879pt) ellipse (3pt and 3pt);
\node at (1691pt,2026.7148pt)[below right,color=black]{void setId()};
\draw[pstyle4] (1682pt,2054.834pt) ellipse (3pt and 3pt);
\node at (1691pt,2044.4609pt)[below right,color=black]{String getName()};
\draw[pstyle4] (1682pt,2072.5801pt) ellipse (3pt and 3pt);
\node at (1691pt,2062.207pt)[below right,color=black]{void setName(String name)};
\draw[pstyle4] (1682pt,2090.3262pt) ellipse (3pt and 3pt);
\node at (1691pt,2079.9531pt)[below right,color=black]{String getSurname()};
\draw[pstyle4] (1682pt,2108.0723pt) ellipse (3pt and 3pt);
\node at (1691pt,2097.6992pt)[below right,color=black]{void setSurname(String surname)};
\draw[pstyle4] (1682pt,2125.8184pt) ellipse (3pt and 3pt);
\node at (1691pt,2115.4453pt)[below right,color=black]{String[] getNumbers()};
\draw[pstyle4] (1682pt,2143.5645pt) ellipse (3pt and 3pt);
\node at (1691pt,2133.1914pt)[below right,color=black]{void setNumbers(String[] numbers)};
\draw[pstyle4] (1682pt,2161.3105pt) ellipse (3pt and 3pt);
\node at (1691pt,2150.9375pt)[below right,color=black]{String[] getEmails()};
\draw[pstyle4] (1682pt,2179.0566pt) ellipse (3pt and 3pt);
\node at (1691pt,2168.6836pt)[below right,color=black]{void setEmails(String[] emails)};
\draw[pstyle4] (1682pt,2196.8027pt) ellipse (3pt and 3pt);
\node at (1691pt,2186.4297pt)[below right,color=black]{Byte[] getProfilePicture()};
\draw[pstyle4] (1682pt,2214.5488pt) ellipse (3pt and 3pt);
\node at (1691pt,2204.1758pt)[below right,color=black]{void setProfilePicture(Byte[] profilePicture)};
\draw[pstyle4] (1682pt,2232.2949pt) ellipse (3pt and 3pt);
\node at (1691pt,2221.9219pt)[below right,color=black]{Set\textless Integer\textgreater  getAllTagIndexes()};
\draw[pstyle4] (1682pt,2250.041pt) ellipse (3pt and 3pt);
\node at (1691pt,2239.668pt)[below right,color=black]{void removeTagIndex(Integer tagIndex)};
\draw[pstyle4] (1682pt,2267.7871pt) ellipse (3pt and 3pt);
\node at (1691pt,2257.4141pt)[below right,color=black]{void addTagIndex(Integer tagIndex)};
\draw[pstyle4] (1682pt,2285.5332pt) ellipse (3pt and 3pt);
\node at (1691pt,2275.1602pt)[below right,color=black]{boolean equals(Object o)};
\draw[pstyle4] (1682pt,2303.2793pt) ellipse (3pt and 3pt);
\node at (1691pt,2292.9063pt)[below right,color=black]{String toString()};
\draw[pstyle0] (1409pt,404pt) arc (180:270:5pt) -- (1414pt,399pt) -- (1791.2787pt,399pt) arc (270:360:5pt) -- (1796.2787pt,404pt) -- (1796.2787pt,1045.3672pt) arc (0:90:5pt) -- (1791.2787pt,1050.3672pt) -- (1414pt,1050.3672pt) arc (90:180:5pt) -- (1409pt,1045.3672pt) -- cycle;
\draw[pstyle1] (1553.5493pt,415pt) ellipse (11pt and 11pt);
\node at (1553.5493pt,415pt)[]{\textbf{\Large C}};
\node at (1574.0493pt,406.127pt)[below right,color=black]{AddressBook};
\draw[pstyle2] (1410pt,431pt) -- (1795.2787pt,431pt);
\draw[pstyle3] (1417pt,442.373pt) rectangle (1423pt,448.373pt);
\node at (1429pt,435pt)[below right,color=black]{\underline{AddressBook instance}};
\draw[pstyle3] (1417pt,460.1191pt) rectangle (1423pt,466.1191pt);
\node at (1429pt,452.7461pt)[below right,color=black]{ObservableList\textless Contact\textgreater  contacts};
\draw[pstyle3] (1417pt,477.8652pt) rectangle (1423pt,483.8652pt);
\node at (1429pt,470.4922pt)[below right,color=black]{ObservableList\textless Tag\textgreater  tags};
\draw[pstyle3] (1417pt,495.6113pt) rectangle (1423pt,501.6113pt);
\node at (1429pt,488.2383pt)[below right,color=black]{String dbUrl};
\draw[pstyle3] (1417pt,513.3574pt) rectangle (1423pt,519.3574pt);
\node at (1429pt,505.9844pt)[below right,color=black]{Database db};
\draw[pstyle3] (1417pt,531.1035pt) rectangle (1423pt,537.1035pt);
\node at (1429pt,523.7305pt)[below right,color=black]{String pathData};
\draw[pstyle3] (1417pt,548.8496pt) rectangle (1423pt,554.8496pt);
\node at (1429pt,541.4766pt)[below right,color=black]{String pathConfig};
\draw[pstyle2] (1410pt,563.2227pt) -- (1795.2787pt,563.2227pt);
\draw[pstyle5] (1417pt,574.5957pt) rectangle (1423pt,580.5957pt);
\node at (1429pt,567.2227pt)[below right,color=black]{AddressBook()};
\draw[pstyle4] (1420pt,595.3418pt) ellipse (3pt and 3pt);
\node at (1429pt,584.9688pt)[below right,color=black]{ObservableList\textless Contact\textgreater  getAllContacts()};
\draw[pstyle4] (1420pt,613.0879pt) ellipse (3pt and 3pt);
\node at (1429pt,602.7148pt)[below right,color=black]{void addContact(Contact c)};
\draw[pstyle4] (1420pt,630.834pt) ellipse (3pt and 3pt);
\node at (1429pt,620.4609pt)[below right,color=black]{public void addManyContacts(Collection\textless Contact\textgreater  c)};
\draw[pstyle4] (1420pt,648.5801pt) ellipse (3pt and 3pt);
\node at (1429pt,638.207pt)[below right,color=black]{void removeContact(Contact c)};
\draw[pstyle4] (1420pt,666.3262pt) ellipse (3pt and 3pt);
\node at (1429pt,655.9531pt)[below right,color=black]{ObservableList\textless Tag\textgreater  getAllTags()};
\draw[pstyle4] (1420pt,684.0723pt) ellipse (3pt and 3pt);
\node at (1429pt,673.6992pt)[below right,color=black]{Tag getTag(int id)};
\draw[pstyle4] (1420pt,701.8184pt) ellipse (3pt and 3pt);
\node at (1429pt,691.4453pt)[below right,color=black]{Tag getTag(String description)};
\draw[pstyle4] (1420pt,719.5645pt) ellipse (3pt and 3pt);
\node at (1429pt,709.1914pt)[below right,color=black]{void addTag(Tag tag)};
\draw[pstyle4] (1420pt,737.3105pt) ellipse (3pt and 3pt);
\node at (1429pt,726.9375pt)[below right,color=black]{void addManyTags(Collection\textless Tag\textgreater  c)};
\draw[pstyle4] (1420pt,755.0566pt) ellipse (3pt and 3pt);
\node at (1429pt,744.6836pt)[below right,color=black]{void removeTag(Tag tag)};
\draw[pstyle4] (1420pt,772.8027pt) ellipse (3pt and 3pt);
\node at (1429pt,762.4297pt)[below right,color=black]{\underline{AddressBook getInstance()}};
\draw[pstyle4] (1420pt,790.5488pt) ellipse (3pt and 3pt);
\node at (1429pt,780.1758pt)[below right,color=black]{void setDBUrl(String dbUrl)};
\draw[pstyle4] (1420pt,808.2949pt) ellipse (3pt and 3pt);
\node at (1429pt,797.9219pt)[below right,color=black]{String getDBUrl()};
\draw[pstyle4] (1420pt,826.041pt) ellipse (3pt and 3pt);
\node at (1429pt,815.668pt)[below right,color=black]{void saveOBJ()};
\draw[pstyle4] (1420pt,843.7871pt) ellipse (3pt and 3pt);
\node at (1429pt,833.4141pt)[below right,color=black]{void loadOBJ()};
\draw[pstyle4] (1420pt,861.5332pt) ellipse (3pt and 3pt);
\node at (1429pt,851.1602pt)[below right,color=black]{void removeOBJ()};
\draw[pstyle4] (1420pt,879.2793pt) ellipse (3pt and 3pt);
\node at (1429pt,868.9063pt)[below right,color=black]{void removeConfig()};
\draw[pstyle4] (1420pt,897.0254pt) ellipse (3pt and 3pt);
\node at (1429pt,886.6523pt)[below right,color=black]{void saveToDB()};
\draw[pstyle4] (1420pt,914.7715pt) ellipse (3pt and 3pt);
\node at (1429pt,904.3984pt)[below right,color=black]{void dataToDB()};
\draw[pstyle4] (1420pt,932.5176pt) ellipse (3pt and 3pt);
\node at (1429pt,922.1445pt)[below right,color=black]{void loadFromDB()};
\draw[pstyle4] (1420pt,950.2637pt) ellipse (3pt and 3pt);
\node at (1429pt,939.8906pt)[below right,color=black]{void loadConfig()};
\draw[pstyle4] (1420pt,968.0098pt) ellipse (3pt and 3pt);
\node at (1429pt,957.6367pt)[below right,color=black]{void saveConfig()};
\draw[pstyle4] (1420pt,985.7559pt) ellipse (3pt and 3pt);
\node at (1429pt,975.3828pt)[below right,color=black]{void initDB()};
\draw[pstyle4] (1420pt,1003.502pt) ellipse (3pt and 3pt);
\node at (1429pt,993.1289pt)[below right,color=black]{Database getDB()};
\draw[pstyle4] (1420pt,1021.248pt) ellipse (3pt and 3pt);
\node at (1429pt,1010.875pt)[below right,color=black]{void removeDB()};
\draw[pstyle4] (1420pt,1038.9941pt) ellipse (3pt and 3pt);
\node at (1429pt,1028.6211pt)[below right,color=black]{String toString()};
\draw[pstyle0] (282.5pt,12pt) arc (180:270:5pt) -- (287.5pt,7pt) -- (513.8435pt,7pt) arc (270:360:5pt) -- (518.8435pt,12pt) -- (518.8435pt,333.9375pt) arc (0:90:5pt) -- (513.8435pt,338.9375pt) -- (287.5pt,338.9375pt) arc (90:180:5pt) -- (282.5pt,333.9375pt) -- cycle;
\draw[pstyle1] (346.5218pt,23pt) ellipse (11pt and 11pt);
\node at (346.5218pt,23pt)[]{\textbf{\Large C}};
\node at (367.0218pt,14.127pt)[below right,color=black]{MainController};
\draw[pstyle2] (283.5pt,39pt) -- (517.8435pt,39pt);
\draw[pstyle3] (290.5pt,50.373pt) rectangle (296.5pt,56.373pt);
\node at (302.5pt,43pt)[below right,color=black]{AddressBook addressBook};
\draw[pstyle2] (283.5pt,64.7461pt) -- (517.8435pt,64.7461pt);
\draw[pstyle4] (293.5pt,79.1191pt) ellipse (3pt and 3pt);
\node at (302.5pt,68.7461pt)[below right,color=black]{void initialize()};
\draw[pstyle5] (290.5pt,93.8652pt) rectangle (296.5pt,99.8652pt);
\node at (302.5pt,86.4922pt)[below right,color=black]{void setImageCircle()};
\draw[pstyle5] (290.5pt,111.6113pt) rectangle (296.5pt,117.6113pt);
\node at (302.5pt,104.2383pt)[below right,color=black]{void searchFieldBinding()};
\draw[pstyle5] (290.5pt,129.3574pt) rectangle (296.5pt,135.3574pt);
\node at (302.5pt,121.9844pt)[below right,color=black]{void onFilterIconClicked()};
\draw[pstyle5] (290.5pt,147.1035pt) rectangle (296.5pt,153.1035pt);
\node at (302.5pt,139.7305pt)[below right,color=black]{void onAddContact()};
\draw[pstyle5] (290.5pt,164.8496pt) rectangle (296.5pt,170.8496pt);
\node at (302.5pt,157.4766pt)[below right,color=black]{void onContactClicked()};
\draw[pstyle5] (290.5pt,182.5957pt) rectangle (296.5pt,188.5957pt);
\node at (302.5pt,175.2227pt)[below right,color=black]{void onModifyContact()};
\draw[pstyle5] (290.5pt,200.3418pt) rectangle (296.5pt,206.3418pt);
\node at (302.5pt,192.9688pt)[below right,color=black]{void onDeleteContact()};
\draw[pstyle5] (290.5pt,218.0879pt) rectangle (296.5pt,224.0879pt);
\node at (302.5pt,210.7148pt)[below right,color=black]{void onSaveContact()};
\draw[pstyle5] (290.5pt,235.834pt) rectangle (296.5pt,241.834pt);
\node at (302.5pt,228.4609pt)[below right,color=black]{void onCancel()};
\draw[pstyle5] (290.5pt,253.5801pt) rectangle (296.5pt,259.5801pt);
\node at (302.5pt,246.207pt)[below right,color=black]{void showImportPopup()};
\draw[pstyle5] (290.5pt,271.3262pt) rectangle (296.5pt,277.3262pt);
\node at (302.5pt,263.9531pt)[below right,color=black]{void showExportPopup()};
\draw[pstyle5] (290.5pt,289.0723pt) rectangle (296.5pt,295.0723pt);
\node at (302.5pt,281.6992pt)[below right,color=black]{void showConfigPopup()};
\draw[pstyle5] (290.5pt,306.8184pt) rectangle (296.5pt,312.8184pt);
\node at (302.5pt,299.4453pt)[below right,color=black]{void showManageTagsPopup()};
\draw[pstyle5] (290.5pt,324.5645pt) rectangle (296.5pt,330.5645pt);
\node at (302.5pt,317.1914pt)[below right,color=black]{void showImagePopup()};
\draw[pstyle0] (820.5pt,1525pt) arc (180:270:5pt) -- (825.5pt,1520pt) -- (1243.3041pt,1520pt) arc (270:360:5pt) -- (1248.3041pt,1525pt) -- (1248.3041pt,1758.207pt) arc (0:90:5pt) -- (1243.3041pt,1763.207pt) -- (825.5pt,1763.207pt) arc (90:180:5pt) -- (820.5pt,1758.207pt) -- cycle;
\draw[pstyle1] (997.1949pt,1536pt) ellipse (11pt and 11pt);
\node at (997.1949pt,1536pt)[]{\textbf{\Large C}};
\node at (1017.6949pt,1527.127pt)[below right,color=black]{Converter};
\draw[pstyle2] (821.5pt,1552pt) -- (1247.3041pt,1552pt);
\draw[pstyle2] (821.5pt,1560pt) -- (1247.3041pt,1560pt);
\draw[pstyle4] (831.5pt,1574.373pt) ellipse (3pt and 3pt);
\node at (840.5pt,1564pt)[below right,color=black]{\underline{String byteArrayToString(Byte[] byteObjectArray)}};
\draw[pstyle4] (831.5pt,1592.1191pt) ellipse (3pt and 3pt);
\node at (840.5pt,1581.7461pt)[below right,color=black]{\underline{boolean checkCSVFormat(File file)}};
\draw[pstyle4] (831.5pt,1609.8652pt) ellipse (3pt and 3pt);
\node at (840.5pt,1599.4922pt)[below right,color=black]{\underline{boolean checkVCardFormat(File file)}};
\draw[pstyle4] (831.5pt,1627.6113pt) ellipse (3pt and 3pt);
\node at (840.5pt,1617.2383pt)[below right,color=black]{\underline{Byte[] imageViewToByteArray(ImageView imageView)}};
\draw[pstyle4] (831.5pt,1645.3574pt) ellipse (3pt and 3pt);
\node at (840.5pt,1634.9844pt)[below right,color=black]{\underline{void onExportCSV(Collection\textless Contact\textgreater  contacts, File file)}};
\draw[pstyle4] (831.5pt,1663.1035pt) ellipse (3pt and 3pt);
\node at (840.5pt,1652.7305pt)[below right,color=black]{\underline{void onExportVCard(Collection\textless Contact\textgreater  contacts, File file)}};
\draw[pstyle4] (831.5pt,1680.8496pt) ellipse (3pt and 3pt);
\node at (840.5pt,1670.4766pt)[below right,color=black]{\underline{Collection\textless Contact\textgreater  parseCSV(File file)}};
\draw[pstyle4] (831.5pt,1698.5957pt) ellipse (3pt and 3pt);
\node at (840.5pt,1688.2227pt)[below right,color=black]{\underline{Collection\textless Contact\textgreater  parseVCard(File file)}};
\draw[pstyle4] (831.5pt,1716.3418pt) ellipse (3pt and 3pt);
\node at (840.5pt,1705.9688pt)[below right,color=black]{\underline{Byte[] stringToByteArray(String originalString)}};
\draw[pstyle4] (831.5pt,1734.0879pt) ellipse (3pt and 3pt);
\node at (840.5pt,1723.7148pt)[below right,color=black]{\underline{byte[] toPrimitive(Byte[] byteObjectArray)}};
\draw[pstyle4] (831.5pt,1751.834pt) ellipse (3pt and 3pt);
\node at (840.5pt,1741.4609pt)[below right,color=black]{\underline{Byte[] toWrapper(byte[] byteArray)}};
\draw[pstyle0] (529pt,652.5pt) arc (180:270:5pt) -- (534pt,647.5pt) -- (772.9191pt,647.5pt) arc (270:360:5pt) -- (777.9191pt,652.5pt) -- (777.9191pt,796.9766pt) arc (0:90:5pt) -- (772.9191pt,801.9766pt) -- (534pt,801.9766pt) arc (90:180:5pt) -- (529pt,796.9766pt) -- cycle;
\draw[pstyle1] (572.0828pt,663.5pt) ellipse (11pt and 11pt);
\node at (572.0828pt,663.5pt)[]{\textbf{\Large C}};
\node at (592.3234pt,654.627pt)[below right,color=black]{ImportPopupController};
\draw[pstyle2] (530pt,679.5pt) -- (776.9191pt,679.5pt);
\draw[pstyle3] (537pt,690.873pt) rectangle (543pt,696.873pt);
\node at (549pt,683.5pt)[below right,color=black]{ContactManager contactManager};
\draw[pstyle3] (537pt,708.6191pt) rectangle (543pt,714.6191pt);
\node at (549pt,701.2461pt)[below right,color=black]{File file};
\draw[pstyle3] (537pt,726.3652pt) rectangle (543pt,732.3652pt);
\node at (549pt,718.9922pt)[below right,color=black]{Button importButton};
\draw[pstyle2] (530pt,740.7383pt) -- (776.9191pt,740.7383pt);
\draw[pstyle4] (540pt,755.1113pt) ellipse (3pt and 3pt);
\node at (549pt,744.7383pt)[below right,color=black]{void initialize()};
\draw[pstyle5] (537pt,769.8574pt) rectangle (543pt,775.8574pt);
\node at (549pt,762.4844pt)[below right,color=black]{void selectFile()};
\draw[pstyle5] (537pt,787.6035pt) rectangle (543pt,793.6035pt);
\node at (549pt,780.2305pt)[below right,color=black]{void onImport()};
\draw[pstyle0] (813pt,634.5pt) arc (180:270:5pt) -- (818pt,629.5pt) -- (1080.8528pt,629.5pt) arc (270:360:5pt) -- (1085.8528pt,634.5pt) -- (1085.8528pt,814.4688pt) arc (0:90:5pt) -- (1080.8528pt,819.4688pt) -- (818pt,819.4688pt) arc (90:180:5pt) -- (813pt,814.4688pt) -- cycle;
\draw[pstyle1] (868.6533pt,645.5pt) ellipse (11pt and 11pt);
\node at (868.6533pt,645.5pt)[]{\textbf{\Large C}};
\node at (889.1533pt,636.627pt)[below right,color=black]{ExportPopupController};
\draw[pstyle2] (814pt,661.5pt) -- (1084.8528pt,661.5pt);
\draw[pstyle3] (821pt,672.873pt) rectangle (827pt,678.873pt);
\node at (833pt,665.5pt)[below right,color=black]{ContactManager contactManager};
\draw[pstyle3] (821pt,690.6191pt) rectangle (827pt,696.6191pt);
\node at (833pt,683.2461pt)[below right,color=black]{TagManager tagManager};
\draw[pstyle3] (821pt,708.3652pt) rectangle (827pt,714.3652pt);
\node at (833pt,700.9922pt)[below right,color=black]{File file};
\draw[pstyle3] (821pt,726.1113pt) rectangle (827pt,732.1113pt);
\node at (833pt,718.7383pt)[below right,color=black]{ChoiceBox\textless String\textgreater  exportChoiceBox};
\draw[pstyle3] (821pt,743.8574pt) rectangle (827pt,749.8574pt);
\node at (833pt,736.4844pt)[below right,color=black]{Button saveButton};
\draw[pstyle2] (814pt,758.2305pt) -- (1084.8528pt,758.2305pt);
\draw[pstyle4] (824pt,772.6035pt) ellipse (3pt and 3pt);
\node at (833pt,762.2305pt)[below right,color=black]{void initialize()};
\draw[pstyle5] (821pt,787.3496pt) rectangle (827pt,793.3496pt);
\node at (833pt,779.9766pt)[below right,color=black]{void choosePath()};
\draw[pstyle5] (821pt,805.0957pt) rectangle (827pt,811.0957pt);
\node at (833pt,797.7227pt)[below right,color=black]{void onExport()};
\draw[pstyle0] (1121.5pt,608pt) arc (180:270:5pt) -- (1126.5pt,603pt) -- (1348.4367pt,603pt) arc (270:360:5pt) -- (1353.4367pt,608pt) -- (1353.4367pt,841.207pt) arc (0:90:5pt) -- (1348.4367pt,846.207pt) -- (1126.5pt,846.207pt) arc (90:180:5pt) -- (1121.5pt,841.207pt) -- cycle;
\draw[pstyle1] (1138.6249pt,619pt) ellipse (11pt and 11pt);
\node at (1138.6249pt,619pt)[]{\textbf{\Large C}};
\node at (1153.0971pt,610.127pt)[below right,color=black]{ManageTagsPopupController};
\draw[pstyle2] (1122.5pt,635pt) -- (1352.4367pt,635pt);
\draw[pstyle3] (1129.5pt,646.373pt) rectangle (1135.5pt,652.373pt);
\node at (1141.5pt,639pt)[below right,color=black]{TagManager tagManager};
\draw[pstyle3] (1129.5pt,664.1191pt) rectangle (1135.5pt,670.1191pt);
\node at (1141.5pt,656.7461pt)[below right,color=black]{ListView\textless String\textgreater  tagsListView};
\draw[pstyle3] (1129.5pt,681.8652pt) rectangle (1135.5pt,687.8652pt);
\node at (1141.5pt,674.4922pt)[below right,color=black]{Button addButton};
\draw[pstyle3] (1129.5pt,699.6113pt) rectangle (1135.5pt,705.6113pt);
\node at (1141.5pt,692.2383pt)[below right,color=black]{Button updateButton};
\draw[pstyle3] (1129.5pt,717.3574pt) rectangle (1135.5pt,723.3574pt);
\node at (1141.5pt,709.9844pt)[below right,color=black]{Button deleteButton};
\draw[pstyle3] (1129.5pt,735.1035pt) rectangle (1135.5pt,741.1035pt);
\node at (1141.5pt,727.7305pt)[below right,color=black]{TextField nameField};
\draw[pstyle2] (1122.5pt,749.4766pt) -- (1352.4367pt,749.4766pt);
\draw[pstyle4] (1132.5pt,763.8496pt) ellipse (3pt and 3pt);
\node at (1141.5pt,753.4766pt)[below right,color=black]{void initialize()};
\draw[pstyle5] (1129.5pt,778.5957pt) rectangle (1135.5pt,784.5957pt);
\node at (1141.5pt,771.2227pt)[below right,color=black]{void onAdd()};
\draw[pstyle5] (1129.5pt,796.3418pt) rectangle (1135.5pt,802.3418pt);
\node at (1141.5pt,788.9688pt)[below right,color=black]{void onUpdate()};
\draw[pstyle5] (1129.5pt,814.0879pt) rectangle (1135.5pt,820.0879pt);
\node at (1141.5pt,806.7148pt)[below right,color=black]{void onDelete()};
\draw[pstyle5] (1129.5pt,831.834pt) rectangle (1135.5pt,837.834pt);
\node at (1141.5pt,824.4609pt)[below right,color=black]{void onTagClicked()};
\draw[pstyle0] (7pt,590pt) arc (180:270:5pt) -- (12pt,585pt) -- (187.0824pt,585pt) arc (270:360:5pt) -- (192.0824pt,590pt) -- (192.0824pt,858.6992pt) arc (0:90:5pt) -- (187.0824pt,863.6992pt) -- (12pt,863.6992pt) arc (90:180:5pt) -- (7pt,858.6992pt) -- cycle;
\draw[pstyle1] (23.4976pt,601pt) ellipse (11pt and 11pt);
\node at (23.4976pt,601pt)[]{\textbf{\Large C}};
\node at (37.8304pt,592.127pt)[below right,color=black]{ImagePopupController};
\draw[pstyle2] (8pt,617pt) -- (191.0824pt,617pt);
\draw[pstyle3] (15pt,628.373pt) rectangle (21pt,634.373pt);
\node at (27pt,621pt)[below right,color=black]{ImageView imgAdd};
\draw[pstyle3] (15pt,646.1191pt) rectangle (21pt,652.1191pt);
\node at (27pt,638.7461pt)[below right,color=black]{HBox defaultImgHBox};
\draw[pstyle3] (15pt,663.8652pt) rectangle (21pt,669.8652pt);
\node at (27pt,656.4922pt)[below right,color=black]{File selectedImage};
\draw[pstyle3] (15pt,681.6113pt) rectangle (21pt,687.6113pt);
\node at (27pt,674.2383pt)[below right,color=black]{int imageIndex};
\draw[pstyle2] (8pt,695.9844pt) -- (191.0824pt,695.9844pt);
\draw[pstyle4] (18pt,710.3574pt) ellipse (3pt and 3pt);
\node at (27pt,699.9844pt)[below right,color=black]{void initialize()};
\draw[pstyle4] (18pt,728.1035pt) ellipse (3pt and 3pt);
\node at (27pt,717.7305pt)[below right,color=black]{File getSelectedImage()};
\draw[pstyle4] (18pt,745.8496pt) ellipse (3pt and 3pt);
\node at (27pt,735.4766pt)[below right,color=black]{int getImageIndex()};
\draw[pstyle5] (15pt,760.5957pt) rectangle (21pt,766.5957pt);
\node at (27pt,753.2227pt)[below right,color=black]{void onImageClicked1()};
\draw[pstyle5] (15pt,778.3418pt) rectangle (21pt,784.3418pt);
\node at (27pt,770.9688pt)[below right,color=black]{void onImageClicked2()};
\draw[pstyle5] (15pt,796.0879pt) rectangle (21pt,802.0879pt);
\node at (27pt,788.7148pt)[below right,color=black]{void onImageClicked3()};
\draw[pstyle5] (15pt,813.834pt) rectangle (21pt,819.834pt);
\node at (27pt,806.4609pt)[below right,color=black]{void onImageClicked4()};
\draw[pstyle5] (15pt,831.5801pt) rectangle (21pt,837.5801pt);
\node at (27pt,824.207pt)[below right,color=black]{void onImageClicked5()};
\draw[pstyle5] (15pt,849.3262pt) rectangle (21pt,855.3262pt);
\node at (27pt,841.9531pt)[below right,color=black]{void onRestoreButton()};
\draw[pstyle0] (227pt,670pt) arc (180:270:5pt) -- (232pt,665pt) -- (417.4331pt,665pt) arc (270:360:5pt) -- (422.4331pt,670pt) -- (422.4331pt,778.9844pt) arc (0:90:5pt) -- (417.4331pt,783.9844pt) -- (232pt,783.9844pt) arc (90:180:5pt) -- (227pt,778.9844pt) -- cycle;
\draw[pstyle1] (242pt,681pt) ellipse (11pt and 11pt);
\node at (242pt,681pt)[]{\textbf{\Large C}};
\node at (256pt,672.127pt)[below right,color=black]{ConfirmPopupController};
\draw[pstyle2] (228pt,697pt) -- (421.4331pt,697pt);
\draw[pstyle3] (235pt,708.373pt) rectangle (241pt,714.373pt);
\node at (247pt,701pt)[below right,color=black]{boolean choice};
\draw[pstyle2] (228pt,722.7461pt) -- (421.4331pt,722.7461pt);
\draw[pstyle4] (238pt,737.1191pt) ellipse (3pt and 3pt);
\node at (247pt,726.7461pt)[below right,color=black]{boolean getChoice()};
\draw[pstyle5] (235pt,751.8652pt) rectangle (241pt,757.8652pt);
\node at (247pt,744.4922pt)[below right,color=black]{onConfirm()};
\draw[pstyle5] (235pt,769.6113pt) rectangle (241pt,775.6113pt);
\node at (247pt,762.2383pt)[below right,color=black]{onCancel()};
\draw[pstyle0] (1499.5pt,83pt) arc (180:270:5pt) -- (1504.5pt,78pt) -- (1700.3225pt,78pt) arc (270:360:5pt) -- (1705.3225pt,83pt) -- (1705.3225pt,262.9688pt) arc (0:90:5pt) -- (1700.3225pt,267.9688pt) -- (1504.5pt,267.9688pt) arc (90:180:5pt) -- (1499.5pt,262.9688pt) -- cycle;
\draw[pstyle1] (1523.3509pt,94pt) ellipse (11pt and 11pt);
\node at (1523.3509pt,94pt)[]{\textbf{\Large C}};
\node at (1539.3178pt,85.127pt)[below right,color=black]{ConfigPopupController};
\draw[pstyle2] (1500.5pt,110pt) -- (1704.3225pt,110pt);
\draw[pstyle3] (1507.5pt,121.373pt) rectangle (1513.5pt,127.373pt);
\node at (1519.5pt,114pt)[below right,color=black]{AddressBook addressBook};
\draw[pstyle3] (1507.5pt,139.1191pt) rectangle (1513.5pt,145.1191pt);
\node at (1519.5pt,131.7461pt)[below right,color=black]{Label resultLabel};
\draw[pstyle3] (1507.5pt,156.8652pt) rectangle (1513.5pt,162.8652pt);
\node at (1519.5pt,149.4922pt)[below right,color=black]{Button verifyButton};
\draw[pstyle3] (1507.5pt,174.6113pt) rectangle (1513.5pt,180.6113pt);
\node at (1519.5pt,167.2383pt)[below right,color=black]{Button confirmButton};
\draw[pstyle3] (1507.5pt,192.3574pt) rectangle (1513.5pt,198.3574pt);
\node at (1519.5pt,184.9844pt)[below right,color=black]{TextField textField};
\draw[pstyle2] (1500.5pt,206.7305pt) -- (1704.3225pt,206.7305pt);
\draw[pstyle4] (1510.5pt,221.1035pt) ellipse (3pt and 3pt);
\node at (1519.5pt,210.7305pt)[below right,color=black]{void initialize()};
\draw[pstyle5] (1507.5pt,235.8496pt) rectangle (1513.5pt,241.8496pt);
\node at (1519.5pt,228.4766pt)[below right,color=black]{void onVerify()};
\draw[pstyle5] (1507.5pt,253.5957pt) rectangle (1513.5pt,259.5957pt);
\node at (1519.5pt,246.2227pt)[below right,color=black]{void onConfirm()};
\draw[pstyle0] (1420.5pt,1115pt) arc (180:270:5pt) -- (1425.5pt,1110pt) -- (1819.5978pt,1110pt) arc (270:360:5pt) -- (1824.5978pt,1115pt) -- (1824.5978pt,1454.6836pt) arc (0:90:5pt) -- (1819.5978pt,1459.6836pt) -- (1425.5pt,1459.6836pt) arc (90:180:5pt) -- (1420.5pt,1454.6836pt) -- cycle;
\draw[pstyle1] (1587.9342pt,1126pt) ellipse (11pt and 11pt);
\node at (1587.9342pt,1126pt)[]{\textbf{\Large C}};
\node at (1608.4342pt,1117.127pt)[below right,color=black]{Database};
\draw[pstyle2] (1421.5pt,1142pt) -- (1823.5978pt,1142pt);
\draw[pstyle3] (1428.5pt,1153.373pt) rectangle (1434.5pt,1159.373pt);
\node at (1440.5pt,1146pt)[below right,color=black]{MongoDatabase mongoDb};
\draw[pstyle2] (1421.5pt,1167.7461pt) -- (1823.5978pt,1167.7461pt);
\draw[pstyle4] (1431.5pt,1182.1191pt) ellipse (3pt and 3pt);
\node at (1440.5pt,1171.7461pt)[below right,color=black]{Database(String url)};
\draw[pstyle4] (1431.5pt,1199.8652pt) ellipse (3pt and 3pt);
\node at (1440.5pt,1189.4922pt)[below right,color=black]{\underline{boolean verifyDBUrl(String url)}};
\draw[pstyle4] (1431.5pt,1217.6113pt) ellipse (3pt and 3pt);
\node at (1440.5pt,1207.2383pt)[below right,color=black]{void insert(Contact c)};
\draw[pstyle4] (1431.5pt,1235.3574pt) ellipse (3pt and 3pt);
\node at (1440.5pt,1224.9844pt)[below right,color=black]{void removeContact(Contact c)};
\draw[pstyle4] (1431.5pt,1253.1035pt) ellipse (3pt and 3pt);
\node at (1440.5pt,1242.7305pt)[below right,color=black]{void upsert(Tag tag)};
\draw[pstyle4] (1431.5pt,1270.8496pt) ellipse (3pt and 3pt);
\node at (1440.5pt,1260.4766pt)[below right,color=black]{void removeTag(Tag tag)};
\draw[pstyle4] (1431.5pt,1288.5957pt) ellipse (3pt and 3pt);
\node at (1440.5pt,1278.2227pt)[below right,color=black]{Collection\textless Contact\textgreater  getAllContacts()};
\draw[pstyle4] (1431.5pt,1306.3418pt) ellipse (3pt and 3pt);
\node at (1440.5pt,1295.9688pt)[below right,color=black]{Collection\textless Tag\textgreater  getAllTags()};
\draw[pstyle4] (1431.5pt,1324.0879pt) ellipse (3pt and 3pt);
\node at (1440.5pt,1313.7148pt)[below right,color=black]{void insertManyContacts(Collection\textless Contact\textgreater  contacts)};
\draw[pstyle4] (1431.5pt,1341.834pt) ellipse (3pt and 3pt);
\node at (1440.5pt,1331.4609pt)[below right,color=black]{void insertManyTags(Collection\textless Tag\textgreater  tags)};
\draw[pstyle4] (1431.5pt,1359.5801pt) ellipse (3pt and 3pt);
\node at (1440.5pt,1349.207pt)[below right,color=black]{void deleteAllContacts()};
\draw[pstyle4] (1431.5pt,1377.3262pt) ellipse (3pt and 3pt);
\node at (1440.5pt,1366.9531pt)[below right,color=black]{void deleteAllTags()};
\draw[pstyle4] (1431.5pt,1395.0723pt) ellipse (3pt and 3pt);
\node at (1440.5pt,1384.6992pt)[below right,color=black]{Document contactToDocument(Contact c)};
\draw[pstyle4] (1431.5pt,1412.8184pt) ellipse (3pt and 3pt);
\node at (1440.5pt,1402.4453pt)[below right,color=black]{Contact documentToContact(Document d)};
\draw[pstyle4] (1431.5pt,1430.5645pt) ellipse (3pt and 3pt);
\node at (1440.5pt,1420.1914pt)[below right,color=black]{Document tagToDocument(Tag tag)};
\draw[pstyle4] (1431.5pt,1448.3105pt) ellipse (3pt and 3pt);
\node at (1440.5pt,1437.9375pt)[below right,color=black]{Tag documentToTag (Document d)};
\draw[pstyle0] (2019pt,2050pt) arc (180:270:5pt) -- (2024pt,2045pt) -- (2125.2pt,2045pt) arc (270:360:5pt) -- (2130.2pt,2050pt) -- (2130.2pt,2088pt) arc (0:90:5pt) -- (2125.2pt,2093pt) -- (2024pt,2093pt) arc (90:180:5pt) -- (2019pt,2088pt) -- cycle;
\draw[pstyle6] (2034pt,2061pt) ellipse (11pt and 11pt);
\node at (2034pt,2061pt)[]{\textbf{\Large I}};
\node at (2048pt,2052.127pt)[below right,color=black]{\textit{Serializable}};
\draw[pstyle2] (2020pt,2077pt) -- (2129.2pt,2077pt);
\draw[pstyle2] (2020pt,2085pt) -- (2129.2pt,2085pt);
\draw[pstyle0] (600pt,1221.5pt) arc (180:270:5pt) -- (605pt,1216.5pt) -- (976.0542pt,1216.5pt) arc (270:360:5pt) -- (981.0542pt,1221.5pt) -- (981.0542pt,1348.2305pt) arc (0:90:5pt) -- (976.0542pt,1353.2305pt) -- (605pt,1353.2305pt) arc (90:180:5pt) -- (600pt,1348.2305pt) -- cycle;
\draw[pstyle6] (729.5508pt,1232.5pt) ellipse (11pt and 11pt);
\node at (729.5508pt,1232.5pt)[]{\textbf{\Large I}};
\node at (750.0508pt,1223.627pt)[below right,color=black]{\textit{ContactManager}};
\draw[pstyle2] (601pt,1248.5pt) -- (980.0542pt,1248.5pt);
\draw[pstyle2] (601pt,1256.5pt) -- (980.0542pt,1256.5pt);
\draw[pstyle4] (611pt,1270.873pt) ellipse (3pt and 3pt);
\node at (620pt,1260.5pt)[below right,color=black]{\textit{ObservableList\textless Contact\textgreater  getAllContacts()}};
\draw[pstyle4] (611pt,1288.6191pt) ellipse (3pt and 3pt);
\node at (620pt,1278.2461pt)[below right,color=black]{\textit{void addContact(Contact c)}};
\draw[pstyle4] (611pt,1306.3652pt) ellipse (3pt and 3pt);
\node at (620pt,1295.9922pt)[below right,color=black]{\textit{void addManyContacts(Collection\textless Contact\textgreater  c)}};
\draw[pstyle4] (611pt,1324.1113pt) ellipse (3pt and 3pt);
\node at (620pt,1313.7383pt)[below right,color=black]{\textit{void removeContact(Contact c)}};
\draw[pstyle4] (611pt,1341.8574pt) ellipse (3pt and 3pt);
\node at (620pt,1331.4844pt)[below right,color=black]{\textit{Collection\textless Contact\textgreater  getContactsFromTag(Tag tag)}};
\draw[pstyle0] (1088pt,1213pt) arc (180:270:5pt) -- (1093pt,1208pt) -- (1379.7476pt,1208pt) arc (270:360:5pt) -- (1384.7476pt,1213pt) -- (1384.7476pt,1357.4766pt) arc (0:90:5pt) -- (1379.7476pt,1362.4766pt) -- (1093pt,1362.4766pt) arc (90:180:5pt) -- (1088pt,1357.4766pt) -- cycle;
\draw[pstyle6] (1188.9265pt,1224pt) ellipse (11pt and 11pt);
\node at (1188.9265pt,1224pt)[]{\textbf{\Large I}};
\node at (1209.4265pt,1215.127pt)[below right,color=black]{\textit{TagManager}};
\draw[pstyle2] (1089pt,1240pt) -- (1383.7476pt,1240pt);
\draw[pstyle2] (1089pt,1248pt) -- (1383.7476pt,1248pt);
\draw[pstyle4] (1099pt,1262.373pt) ellipse (3pt and 3pt);
\node at (1108pt,1252pt)[below right,color=black]{\textit{void addTag(Tag tag)}};
\draw[pstyle4] (1099pt,1280.1191pt) ellipse (3pt and 3pt);
\node at (1108pt,1269.7461pt)[below right,color=black]{\textit{void addManyTags(Collection\textless Tag\textgreater  c)}};
\draw[pstyle4] (1099pt,1297.8652pt) ellipse (3pt and 3pt);
\node at (1108pt,1287.4922pt)[below right,color=black]{\textit{void removeTag(Tag tag)}};
\draw[pstyle4] (1099pt,1315.6113pt) ellipse (3pt and 3pt);
\node at (1108pt,1305.2383pt)[below right,color=black]{\textit{Tag getTag(int id)}};
\draw[pstyle4] (1099pt,1333.3574pt) ellipse (3pt and 3pt);
\node at (1108pt,1322.9844pt)[below right,color=black]{\textit{Tag getTag(String descrizione)}};
\draw[pstyle4] (1099pt,1351.1035pt) ellipse (3pt and 3pt);
\node at (1108pt,1340.7305pt)[below right,color=black]{\textit{ObservableList\textless Tag\textgreater  getAllTags()}};
\draw[pstyle7] (1984.28pt,2069pt) ..controls (1996.5pt,2069pt) and (1990.24pt,2069pt) .. (2000.99pt,2069pt);
\draw[pstyle8] (2018.99pt,2069pt) -- (2000.99pt,2063pt) -- (2000.99pt,2075pt) -- (2018.99pt,2069pt) -- cycle;
\draw[pstyle7] (2074.5pt,1754.1pt) ..controls (2074.5pt,1853.05pt) and (2074.5pt,1971.59pt) .. (2074.5pt,2026.59pt);
\draw[pstyle8] (2074.5pt,2044.59pt) -- (2080.5pt,2026.59pt) -- (2068.5pt,2026.59pt) -- (2074.5pt,2044.59pt) -- cycle;
\draw[pstyle8] (1808.28pt,833pt) ..controls (1854.01pt,833pt) and (1877.75pt,833pt) .. (1877.75pt,833pt) ..controls (1877.75pt,833pt) and (1877.75pt,1462.44pt) .. (1877.75pt,1822.96pt);
\draw[pstyle8] (1796.28pt,833pt) -- (1802.28pt,837pt) -- (1808.28pt,833pt) -- (1802.28pt,829pt) -- (1796.28pt,833pt) -- cycle;
\node at (1803.9847pt,820.928pt)[below right,color=black]{1};
\node at (1858.0625pt,1798.6379pt)[below right,color=black]{0..*};
\draw[pstyle8] (1608.25pt,1062.14pt) ..controls (1608.25pt,1082.52pt) and (1608.25pt,1090.56pt) .. (1608.25pt,1109.84pt);
\draw[pstyle9] (1608.25pt,1050.14pt) -- (1604.25pt,1056.14pt) -- (1608.25pt,1062.14pt) -- (1612.25pt,1056.14pt) -- (1608.25pt,1050.14pt) -- cycle;
\node at (1601.3018pt,1057.961pt)[below right,color=black]{1};
\node at (1586.1399pt,1085.5586pt)[below right,color=black]{0..1};
\draw[pstyle8] (1808.18pt,616pt) ..controls (1938.45pt,616pt) and (2074.5pt,616pt) .. (2074.5pt,616pt) ..controls (2074.5pt,616pt) and (2074.5pt,1262.42pt) .. (2074.5pt,1528.82pt);
\draw[pstyle8] (1796.18pt,616pt) -- (1802.18pt,620pt) -- (1808.18pt,616pt) -- (1802.18pt,612pt) -- (1796.18pt,616pt) -- cycle;
\node at (1804.4969pt,595.0875pt)[below right,color=black]{1};
\node at (2052.8049pt,1504.3353pt)[below right,color=black]{0..*};
\draw[pstyle7] (1414.75pt,1050.14pt) ..controls (1414.75pt,1173.61pt) and (1414.75pt,1285pt) .. (1414.75pt,1285pt) ..controls (1414.75pt,1285pt) and (1420.9pt,1285pt) .. (1403.32pt,1285pt);
\draw[pstyle8] (1385.32pt,1285pt) -- (1403.32pt,1291pt) -- (1403.32pt,1279pt) -- (1385.32pt,1285pt) -- cycle;
\draw[pstyle7] (1408.81pt,957pt) ..controls (1208.58pt,957pt) and (925pt,957pt) .. (925pt,957pt) ..controls (925pt,957pt) and (925pt,1102.06pt) .. (925pt,1198.24pt);
\draw[pstyle8] (925pt,1216.24pt) -- (931pt,1198.24pt) -- (919pt,1198.24pt) -- (925pt,1216.24pt) -- cycle;
\draw[pstyle8] (530.57pt,304pt) ..controls (805.69pt,304pt) and (1454.25pt,304pt) .. (1454.25pt,304pt) ..controls (1454.25pt,304pt) and (1454.25pt,343.79pt) .. (1454.25pt,398.89pt);
\draw[pstyle8] (518.57pt,304pt) -- (524.57pt,308pt) -- (530.57pt,304pt) -- (524.57pt,300pt) -- (518.57pt,304pt) -- cycle;
\node at (526.7332pt,296.2822pt)[below right,color=black]{1};
\node at (1443.2181pt,374.03pt)[below right,color=black]{1};
\draw[pstyle7] (317.38pt,339.09pt) ..controls (317.38pt,456.83pt) and (317.38pt,594pt) .. (317.38pt,594pt) ..controls (317.38pt,594pt) and (257.51pt,594pt) .. (198.21pt,594pt);
\draw[pstyle9] (192.21pt,594pt) -- (201.21pt,598pt) -- (197.21pt,594pt) -- (201.21pt,590pt) -- (192.21pt,594pt) -- cycle;
\draw[pstyle7] (352.25pt,339.22pt) ..controls (352.25pt,449.43pt) and (352.25pt,581.52pt) .. (352.25pt,658.87pt);
\draw[pstyle9] (352.25pt,664.87pt) -- (356.25pt,655.87pt) -- (352.25pt,659.87pt) -- (348.25pt,655.87pt) -- (352.25pt,664.87pt) -- cycle;
\draw[pstyle7] (470.25pt,339.17pt) ..controls (470.25pt,731.39pt) and (470.25pt,1682pt) .. (470.25pt,1682pt) ..controls (470.25pt,1682pt) and (654.75pt,1682pt) .. (814.38pt,1682pt);
\draw[pstyle9] (820.38pt,1682pt) -- (811.38pt,1678pt) -- (815.38pt,1682pt) -- (811.38pt,1686pt) -- (820.38pt,1682pt) -- cycle;
\draw[pstyle7] (1248.98pt,1759pt) ..controls (1459.03pt,1759pt) and (1747.75pt,1759pt) .. (1747.75pt,1759pt) ..controls (1747.75pt,1759pt) and (1747.75pt,1779.43pt) .. (1747.75pt,1816.86pt);
\draw[pstyle9] (1747.75pt,1822.86pt) -- (1751.75pt,1813.86pt) -- (1747.75pt,1817.86pt) -- (1743.75pt,1813.86pt) -- (1747.75pt,1822.86pt) -- cycle;
\draw[pstyle8] (689pt,813.55pt) ..controls (689pt,923.38pt) and (689pt,1111.23pt) .. (689pt,1216.28pt);
\draw[pstyle8] (689pt,801.55pt) -- (685pt,807.55pt) -- (689pt,813.55pt) -- (693pt,807.55pt) -- (689pt,801.55pt) -- cycle;
\node at (685.1638pt,809.9222pt)[below right,color=black]{1};
\node at (670.5236pt,1191.9297pt)[below right,color=black]{1};
\draw[pstyle7] (564.5pt,801.52pt) ..controls (564.5pt,1015.4pt) and (564.5pt,1601pt) .. (564.5pt,1601pt) ..controls (564.5pt,1601pt) and (690.42pt,1601pt) .. (814.47pt,1601pt);
\draw[pstyle9] (820.47pt,1601pt) -- (811.47pt,1597pt) -- (815.47pt,1601pt) -- (811.47pt,1605pt) -- (820.47pt,1601pt) -- cycle;
\draw[pstyle8] (869pt,831.58pt) ..controls (869pt,944.18pt) and (869pt,1116.95pt) .. (869pt,1216.44pt);
\draw[pstyle8] (869pt,819.58pt) -- (865pt,825.58pt) -- (869pt,831.58pt) -- (873pt,825.58pt) -- (869pt,819.58pt) -- cycle;
\node at (855.582pt,827.4916pt)[below right,color=black]{1};
\node at (870.2111pt,1192.1072pt)[below right,color=black]{1};
\draw[pstyle8] (1051pt,831.56pt) ..controls (1051pt,973.67pt) and (1051pt,1213pt) .. (1051pt,1213pt) ..controls (1051pt,1213pt) and (1066.08pt,1213pt) .. (1087.65pt,1213pt);
\draw[pstyle8] (1051pt,819.56pt) -- (1047pt,825.56pt) -- (1051pt,831.56pt) -- (1055pt,825.56pt) -- (1051pt,819.56pt) -- cycle;
\node at (1051.5164pt,827.47pt)[below right,color=black]{1};
\node at (1072.8061pt,1192.8751pt)[below right,color=black]{1};
\draw[pstyle7] (1016pt,819.61pt) ..controls (1016pt,987.18pt) and (1016pt,1328.76pt) .. (1016pt,1513.65pt);
\draw[pstyle9] (1016pt,1519.65pt) -- (1020pt,1510.65pt) -- (1016pt,1514.65pt) -- (1012pt,1510.65pt) -- (1016pt,1519.65pt) -- cycle;
\draw[pstyle7] (1121.49pt,617pt) ..controls (888.78pt,617pt) and (387.12pt,617pt) .. (387.12pt,617pt) ..controls (387.12pt,617pt) and (387.12pt,634.03pt) .. (387.12pt,658.78pt);
\draw[pstyle9] (387.12pt,664.78pt) -- (391.12pt,655.78pt) -- (387.12pt,659.78pt) -- (383.12pt,655.78pt) -- (387.12pt,664.78pt) -- cycle;
\draw[pstyle8] (1237.5pt,858.12pt) ..controls (1237.5pt,968.07pt) and (1237.5pt,1114.6pt) .. (1237.5pt,1207.99pt);
\draw[pstyle8] (1237.5pt,846.12pt) -- (1233.5pt,852.12pt) -- (1237.5pt,858.12pt) -- (1241.5pt,852.12pt) -- (1237.5pt,846.12pt) -- cycle;
\node at (1230.2947pt,854.027pt)[below right,color=black]{1};
\node at (1230.2762pt,1183.6219pt)[below right,color=black]{1};
\draw[pstyle8] (1602.5pt,280.12pt) ..controls (1602.5pt,317.71pt) and (1602.5pt,351.3pt) .. (1602.5pt,398.94pt);
\draw[pstyle8] (1602.5pt,268.12pt) -- (1598.5pt,274.12pt) -- (1602.5pt,280.12pt) -- (1606.5pt,274.12pt) -- (1602.5pt,268.12pt) -- cycle;
\node at (1595.2465pt,276.0346pt)[below right,color=black]{1};
\node at (1595.406pt,374.0813pt)[below right,color=black]{1};
\draw[pstyle7] (1545.75pt,1460.05pt) ..controls (1545.75pt,1496.96pt) and (1545.75pt,1525pt) .. (1545.75pt,1525pt) ..controls (1545.75pt,1525pt) and (1394.73pt,1525pt) .. (1254.72pt,1525pt);
\draw[pstyle9] (1248.72pt,1525pt) -- (1257.72pt,1529pt) -- (1253.72pt,1525pt) -- (1257.72pt,1521pt) -- (1248.72pt,1525pt) -- cycle;
\end{tikzpicture}
}
\end{adjustbox}

\begin{figure}[h]
	\caption{Diagramma classi completo}
	\label{fig:Diagramma classi completo}
\end{figure}

\newpage
\subsection{Diagrammi di sequenza}
Si presentano una serie di diagrammi di sequenza che descrivono alcuni casi d'uso definiti nel sistema, descrivendo le interazioni tra l'attore \textit{Utente} e gli oggetti del sistema coinvolti, al fine di soddisfare i requisiti specificati. \\
Oltre ai casi d'uso, vengono inclusi 2 diagrammi che rappresentano flussi operativi rilevanti per la comprensione del funzionamento complessivo del sistema.
\subsubsection{C1 - Aggiungere Contatto}
Il seguente diagramma di sequenza illustra l'esecuzione del caso d'uso C1:
% generated by Plantuml 1.2024.3       
\definecolor{plantucolor0000}{RGB}{255,255,255}
\definecolor{plantucolor0001}{RGB}{24,24,24}
\definecolor{plantucolor0002}{RGB}{0,0,0}
\definecolor{plantucolor0003}{RGB}{226,226,240}
\definecolor{plantucolor0004}{RGB}{238,238,238}
\definecolor{plantucolor0005}{RGB}{168,0,54}

\begin{adjustbox}{width=.88\paperwidth, center}
	\resizebox{\textwidth}{!}{
\begin{tikzpicture}[yscale=-1
,pstyle0/.style={color=plantucolor0001,fill=white,line width=1.0pt}
,pstyle1/.style={color=black,line width=1.5pt}
,pstyle2/.style={color=plantucolor0001,line width=0.5pt,dash pattern=on 5.0pt off 5.0pt}
,pstyle3/.style={color=plantucolor0001,fill=plantucolor0003,line width=0.5pt}
,pstyle4/.style={color=plantucolor0001,line width=0.5pt}
,pstyle5/.style={color=plantucolor0001,fill=plantucolor0001,line width=1.0pt}
,pstyle6/.style={color=plantucolor0001,line width=1.0pt}
,pstyle9/.style={color=plantucolor0005,line width=2.0pt}
]
\draw[pstyle0] (297.6157pt,165.9707pt) rectangle (307.6157pt,201.4492pt);
\draw[pstyle0] (297.6157pt,282.6738pt) rectangle (307.6157pt,318.1523pt);
\draw[pstyle0] (297.6157pt,404.5879pt) rectangle (307.6157pt,435.0664pt);
\draw[pstyle0] (297.6157pt,627.7266pt) rectangle (307.6157pt,735.9512pt);
\draw[pstyle0] (467.0763pt,705.9512pt) rectangle (477.0763pt,735.9512pt);
\draw[pstyle0] (620.8057pt,516.291pt) rectangle (630.8057pt,551.7695pt);
\draw[pstyle1] (10pt,363.6309pt) rectangle (761.0057pt,559.7695pt);
\draw[pstyle2] (44pt,82.7461pt) -- (44pt,753.9512pt);
\draw[pstyle2] (301.9816pt,124.1191pt) -- (301.9816pt,753.9512pt);
\draw[pstyle2] (471.2241pt,210.3438pt) -- (471.2241pt,753.9512pt);
\draw[pstyle2] (624.9285pt,443.9609pt) -- (624.9285pt,753.9512pt);
\draw[pstyle2] (755.6828pt,672.0996pt) -- (755.6828pt,753.9512pt);
\node at (20pt,65pt)[below right,color=black]{Utente};
\draw[pstyle3] (44.6pt,13.5pt) ellipse (8pt and 8pt);
\draw[pstyle4] (44.6pt,21.5pt) -- (44.6pt,48.5pt)(31.6pt,29.5pt) -- (57.6pt,29.5pt)(44.6pt,48.5pt) -- (31.6pt,63.5pt)(44.6pt,48.5pt) -- (57.6pt,63.5pt);
\node at (20pt,752.9512pt)[below right,color=black]{Utente};
\draw[pstyle3] (44.6pt,779.1973pt) ellipse (8pt and 8pt);
\draw[pstyle4] (44.6pt,787.1973pt) -- (44.6pt,814.1973pt)(31.6pt,795.1973pt) -- (57.6pt,795.1973pt)(44.6pt,814.1973pt) -- (31.6pt,829.1973pt)(44.6pt,814.1973pt) -- (57.6pt,829.1973pt);
\draw[pstyle3] (239.9816pt,757.9512pt) arc (180:270:5pt) -- (244.9816pt,752.9512pt) -- (360.2497pt,752.9512pt) arc (270:360:5pt) -- (365.2497pt,757.9512pt) -- (365.2497pt,779.6973pt) arc (0:90:5pt) -- (360.2497pt,784.6973pt) -- (244.9816pt,784.6973pt) arc (90:180:5pt) -- (239.9816pt,779.6973pt) -- cycle;
\node at (246.9816pt,759.9512pt)[below right,color=black]{a MainController};
\draw[pstyle3] (410.2241pt,757.9512pt) arc (180:270:5pt) -- (415.2241pt,752.9512pt) -- (528.9285pt,752.9512pt) arc (270:360:5pt) -- (533.9285pt,757.9512pt) -- (533.9285pt,779.6973pt) arc (0:90:5pt) -- (528.9285pt,784.6973pt) -- (415.2241pt,784.6973pt) arc (90:180:5pt) -- (410.2241pt,779.6973pt) -- cycle;
\node at (417.2241pt,759.9512pt)[below right,color=black]{an AddressBook};
\draw[pstyle3] (543.9285pt,757.9512pt) arc (180:270:5pt) -- (548.9285pt,752.9512pt) -- (702.6828pt,752.9512pt) arc (270:360:5pt) -- (707.6828pt,757.9512pt) -- (707.6828pt,779.6973pt) arc (0:90:5pt) -- (702.6828pt,784.6973pt) -- (548.9285pt,784.6973pt) arc (90:180:5pt) -- (543.9285pt,779.6973pt) -- cycle;
\node at (550.9285pt,759.9512pt)[below right,color=black]{ImagePopupController};
\draw[pstyle3] (717.6828pt,757.9512pt) arc (180:270:5pt) -- (722.6828pt,752.9512pt) -- (790.1495pt,752.9512pt) arc (270:360:5pt) -- (795.1495pt,757.9512pt) -- (795.1495pt,779.6973pt) arc (0:90:5pt) -- (790.1495pt,784.6973pt) -- (722.6828pt,784.6973pt) arc (90:180:5pt) -- (717.6828pt,779.6973pt) -- cycle;
\node at (724.6828pt,759.9512pt)[below right,color=black]{a Contact};
\draw[pstyle0] (297.6157pt,165.9707pt) rectangle (307.6157pt,201.4492pt);
\draw[pstyle0] (297.6157pt,282.6738pt) rectangle (307.6157pt,318.1523pt);
\draw[pstyle0] (297.6157pt,404.5879pt) rectangle (307.6157pt,435.0664pt);
\draw[pstyle0] (297.6157pt,627.7266pt) rectangle (307.6157pt,735.9512pt);
\draw[pstyle0] (467.0763pt,705.9512pt) rectangle (477.0763pt,735.9512pt);
\draw[pstyle0] (620.8057pt,516.291pt) rectangle (630.8057pt,551.7695pt);
\draw[pstyle5] (227.9816pt,111.2246pt) -- (237.9816pt,115.2246pt) -- (227.9816pt,119.2246pt) -- (231.9816pt,115.2246pt) -- cycle;
\draw[pstyle6] (44.6pt,115.2246pt) -- (233.9816pt,115.2246pt);
\node at (51.6pt,96.7461pt)[below right,color=black]{Apre rubrica};
\draw[pstyle3] (239.9816pt,97.7461pt) arc (180:270:5pt) -- (244.9816pt,92.7461pt) -- (360.2497pt,92.7461pt) arc (270:360:5pt) -- (365.2497pt,97.7461pt) -- (365.2497pt,119.4922pt) arc (0:90:5pt) -- (360.2497pt,124.4922pt) -- (244.9816pt,124.4922pt) arc (90:180:5pt) -- (239.9816pt,119.4922pt) -- cycle;
\node at (246.9816pt,99.7461pt)[below right,color=black]{a MainController};
\draw[pstyle6] (302.6157pt,152.9707pt) -- (349.6157pt,152.9707pt);
\draw[pstyle6] (349.6157pt,152.9707pt) -- (349.6157pt,165.9707pt);
\draw[pstyle6] (308.6157pt,165.9707pt) -- (349.6157pt,165.9707pt);
\draw[pstyle5] (318.6157pt,161.9707pt) -- (308.6157pt,165.9707pt) -- (318.6157pt,169.9707pt) -- (314.6157pt,165.9707pt) -- cycle;
\node at (314.6157pt,134.4922pt)[below right,color=black]{initialize(...)};
\draw[pstyle5] (398.2241pt,197.4492pt) -- (408.2241pt,201.4492pt) -- (398.2241pt,205.4492pt) -- (402.2241pt,201.4492pt) -- cycle;
\draw[pstyle6] (302.6157pt,201.4492pt) -- (404.2241pt,201.4492pt);
\node at (309.6157pt,182.9707pt)[below right,color=black]{getInstance()};
\draw[pstyle3] (410.2241pt,183.9707pt) arc (180:270:5pt) -- (415.2241pt,178.9707pt) -- (528.9285pt,178.9707pt) arc (270:360:5pt) -- (533.9285pt,183.9707pt) -- (533.9285pt,205.7168pt) arc (0:90:5pt) -- (528.9285pt,210.7168pt) -- (415.2241pt,210.7168pt) arc (90:180:5pt) -- (410.2241pt,205.7168pt) -- cycle;
\node at (417.2241pt,185.9707pt)[below right,color=black]{an AddressBook};
\draw[pstyle5] (290.6157pt,240.1953pt) -- (300.6157pt,244.1953pt) -- (290.6157pt,248.1953pt) -- (294.6157pt,244.1953pt) -- cycle;
\draw[pstyle6] (44.6pt,244.1953pt) -- (296.6157pt,244.1953pt);
\node at (51.6pt,225.7168pt)[below right,color=black]{clicca "+" o "Aggiungi contatto"};
\draw[pstyle6] (302.6157pt,269.6738pt) -- (349.6157pt,269.6738pt);
\draw[pstyle6] (349.6157pt,269.6738pt) -- (349.6157pt,282.6738pt);
\draw[pstyle6] (308.6157pt,282.6738pt) -- (349.6157pt,282.6738pt);
\draw[pstyle5] (318.6157pt,278.6738pt) -- (308.6157pt,282.6738pt) -- (318.6157pt,286.6738pt) -- (314.6157pt,282.6738pt) -- cycle;
\node at (314.6157pt,251.1953pt)[below right,color=black]{onAddButtonPressed(...)};
\draw[pstyle5] (55.6pt,314.1523pt) -- (45.6pt,318.1523pt) -- (55.6pt,322.1523pt) -- (51.6pt,318.1523pt) -- cycle;
\draw[color=plantucolor0001,line width=1.0pt,dash pattern=on 2.0pt off 2.0pt] (49.6pt,318.1523pt) -- (301.6157pt,318.1523pt);
\node at (61.6pt,299.6738pt)[below right,color=black]{mostra schermata di aggiunta contatto};
\draw[pstyle5] (290.6157pt,344.6309pt) -- (300.6157pt,348.6309pt) -- (290.6157pt,352.6309pt) -- (294.6157pt,348.6309pt) -- cycle;
\draw[pstyle6] (44.6pt,348.6309pt) -- (296.6157pt,348.6309pt);
\node at (51.6pt,330.1523pt)[below right,color=black]{inserisce i campi del contatto};
\draw[color=black,fill=plantucolor0004,line width=1.5pt] (10pt,363.6309pt) -- (71.8pt,363.6309pt) -- (71.8pt,372.1094pt) -- (61.8pt,382.1094pt) -- (10pt,382.1094pt) -- (10pt,363.6309pt);
\draw[pstyle1] (10pt,363.6309pt) rectangle (761.0057pt,559.7695pt);
\node at (25pt,364.6309pt)[below right,color=black]{\textbf{alt}};
\node at (86.8pt,365.6309pt)[below right,color=black]{\textbf{[utente clicca sull'immagine]}};
\draw[pstyle5] (285.6157pt,400.5879pt) -- (295.6157pt,404.5879pt) -- (285.6157pt,408.5879pt) -- (289.6157pt,404.5879pt) -- cycle;
\draw[pstyle6] (44.6pt,404.5879pt) -- (291.6157pt,404.5879pt);
\node at (51.6pt,386.1094pt)[below right,color=black]{clicca sull'immagine per cambiarla};
\draw[pstyle5] (531.9285pt,431.0664pt) -- (541.9285pt,435.0664pt) -- (531.9285pt,439.0664pt) -- (535.9285pt,435.0664pt) -- cycle;
\draw[pstyle6] (302.6157pt,435.0664pt) -- (537.9285pt,435.0664pt);
\node at (309.6157pt,416.5879pt)[below right,color=black]{showImagePopup(...)};
\draw[pstyle3] (543.9285pt,417.5879pt) arc (180:270:5pt) -- (548.9285pt,412.5879pt) -- (702.6828pt,412.5879pt) arc (270:360:5pt) -- (707.6828pt,417.5879pt) -- (707.6828pt,439.334pt) arc (0:90:5pt) -- (702.6828pt,444.334pt) -- (548.9285pt,444.334pt) arc (90:180:5pt) -- (543.9285pt,439.334pt) -- cycle;
\node at (550.9285pt,419.5879pt)[below right,color=black]{ImagePopupController};
\draw[pstyle5] (613.8057pt,473.8125pt) -- (623.8057pt,477.8125pt) -- (613.8057pt,481.8125pt) -- (617.8057pt,477.8125pt) -- cycle;
\draw[pstyle6] (44.6pt,477.8125pt) -- (619.8057pt,477.8125pt);
\node at (51.6pt,459.334pt)[below right,color=black]{sceglie un'immagine};
\draw[pstyle6] (625.8057pt,503.291pt) -- (672.8057pt,503.291pt);
\draw[pstyle6] (672.8057pt,503.291pt) -- (672.8057pt,516.291pt);
\draw[pstyle6] (631.8057pt,516.291pt) -- (672.8057pt,516.291pt);
\draw[pstyle5] (641.8057pt,512.291pt) -- (631.8057pt,516.291pt) -- (641.8057pt,520.291pt) -- (637.8057pt,516.291pt) -- cycle;
\node at (637.8057pt,484.8125pt)[below right,color=black]{carica un'immagine};
\draw[pstyle5] (613.8057pt,547.7695pt) -- (623.8057pt,551.7695pt) -- (613.8057pt,555.7695pt) -- (617.8057pt,551.7695pt) -- cycle;
\draw[pstyle6] (302.6157pt,551.7695pt) -- (619.8057pt,551.7695pt);
\node at (309.6157pt,533.291pt)[below right,color=black]{getSelectedImage()};
\draw[pstyle9] (616.8057pt,542.7695pt) -- (634.8057pt,560.7695pt);
\draw[pstyle9] (616.8057pt,560.7695pt) -- (634.8057pt,542.7695pt);
\draw[pstyle5] (290.6157pt,585.248pt) -- (300.6157pt,589.248pt) -- (290.6157pt,593.248pt) -- (294.6157pt,589.248pt) -- cycle;
\draw[pstyle6] (44.6pt,589.248pt) -- (296.6157pt,589.248pt);
\node at (51.6pt,570.7695pt)[below right,color=black]{clicca "Salva"};
\draw[pstyle6] (302.6157pt,614.7266pt) -- (349.6157pt,614.7266pt);
\draw[pstyle6] (349.6157pt,614.7266pt) -- (349.6157pt,627.7266pt);
\draw[pstyle6] (308.6157pt,627.7266pt) -- (349.6157pt,627.7266pt);
\draw[pstyle5] (318.6157pt,623.7266pt) -- (308.6157pt,627.7266pt) -- (318.6157pt,631.7266pt) -- (314.6157pt,627.7266pt) -- cycle;
\node at (314.6157pt,596.248pt)[below right,color=black]{onSaveContact(...)};
\draw[pstyle5] (705.6828pt,659.2051pt) -- (715.6828pt,663.2051pt) -- (705.6828pt,667.2051pt) -- (709.6828pt,663.2051pt) -- cycle;
\draw[pstyle6] (307.6157pt,663.2051pt) -- (711.6828pt,663.2051pt);
\node at (314.6157pt,644.7266pt)[below right,color=black]{crea il contatto};
\draw[pstyle3] (717.6828pt,645.7266pt) arc (180:270:5pt) -- (722.6828pt,640.7266pt) -- (790.1495pt,640.7266pt) arc (270:360:5pt) -- (795.1495pt,645.7266pt) -- (795.1495pt,667.4727pt) arc (0:90:5pt) -- (790.1495pt,672.4727pt) -- (722.6828pt,672.4727pt) arc (90:180:5pt) -- (717.6828pt,667.4727pt) -- cycle;
\node at (724.6828pt,647.7266pt)[below right,color=black]{a Contact};
\draw[pstyle5] (455.0763pt,701.9512pt) -- (465.0763pt,705.9512pt) -- (455.0763pt,709.9512pt) -- (459.0763pt,705.9512pt) -- cycle;
\draw[pstyle6] (307.6157pt,705.9512pt) -- (461.0763pt,705.9512pt);
\node at (314.6157pt,687.4727pt)[below right,color=black]{addContact(Contact c)};
\end{tikzpicture}
}
\end{adjustbox}

\begin{figure}[h]
\caption{Diagramma sequenza C1 - Aggiungere Contatto}
\label{fig:Diagramma sequenza C1 - Aggiungere Contatto}
\end{figure}

\newpage
\subsubsection{C2 C3 - Eliminare Modificare Contatto}
Il seguente diagramma di sequenza illustra l'esecuzione di entrambi i casi d'uso C2 e C3, tramite la sintassi \texttt{loop}, \texttt{break} e \texttt{alt}:
% generated by Plantuml 1.2024.3       
\definecolor{plantucolor0000}{RGB}{255,255,255}
\definecolor{plantucolor0001}{RGB}{24,24,24}
\definecolor{plantucolor0002}{RGB}{0,0,0}
\definecolor{plantucolor0003}{RGB}{226,226,240}
\definecolor{plantucolor0004}{RGB}{238,238,238}

\begin{adjustbox}{width=.84\paperwidth, center}
	\resizebox{\textwidth}{!}{
\begin{tikzpicture}[yscale=-1
,pstyle0/.style={color=plantucolor0001,fill=white,line width=1.0pt}
,pstyle1/.style={color=black,line width=1.5pt}
,pstyle2/.style={color=plantucolor0001,line width=0.5pt,dash pattern=on 5.0pt off 5.0pt}
,pstyle3/.style={color=plantucolor0001,fill=plantucolor0003,line width=0.5pt}
,pstyle4/.style={color=plantucolor0001,line width=0.5pt}
,pstyle5/.style={color=plantucolor0001,fill=plantucolor0001,line width=1.0pt}
,pstyle6/.style={color=plantucolor0001,line width=1.0pt}
,pstyle7/.style={color=plantucolor0001,line width=1.0pt,dash pattern=on 2.0pt off 2.0pt}
,pstyle8/.style={color=black,fill=plantucolor0004,line width=1.5pt}
]
\draw[pstyle0] (409.8219pt,165.9707pt) rectangle (419.8219pt,274.6738pt);
\draw[pstyle0] (414.8219pt,211.4492pt) rectangle (424.8219pt,244.1953pt);
\draw[pstyle0] (409.8219pt,369.1094pt) rectangle (419.8219pt,404.5879pt);
\draw[pstyle0] (409.8219pt,499.0234pt) rectangle (419.8219pt,527.0234pt);
\draw[pstyle0] (409.8219pt,642.459pt) rectangle (419.8219pt,738.8945pt);
\draw[pstyle0] (409.8219pt,823.7949pt) rectangle (419.8219pt,950.709pt);
\draw[pstyle0] (655.0081pt,211.4492pt) rectangle (665.0081pt,244.1953pt);
\draw[pstyle1] (10pt,289.6738pt) rectangle (680.0081pt,549.0234pt);
\draw[pstyle1] (20pt,419.5879pt) rectangle (670.0081pt,542.0234pt);
\draw[pstyle1] (20pt,563.0234pt) rectangle (819.327pt,970.9766pt);
\draw[pstyle2] (54pt,82.7461pt) -- (54pt,987.9766pt);
\draw[pstyle2] (414.1879pt,124.1191pt) -- (414.1879pt,987.9766pt);
\draw[pstyle2] (659.156pt,210.3438pt) -- (659.156pt,987.9766pt);
\draw[pstyle2] (769.8603pt,959.6035pt) -- (769.8603pt,987.9766pt);
\node at (30pt,65pt)[below right,color=black]{Utente};
\draw[pstyle3] (54.6pt,13.5pt) ellipse (8pt and 8pt);
\draw[pstyle4] (54.6pt,21.5pt) -- (54.6pt,48.5pt)(41.6pt,29.5pt) -- (67.6pt,29.5pt)(54.6pt,48.5pt) -- (41.6pt,63.5pt)(54.6pt,48.5pt) -- (67.6pt,63.5pt);
\node at (30pt,986.9766pt)[below right,color=black]{Utente};
\draw[pstyle3] (54.6pt,1013.2227pt) ellipse (8pt and 8pt);
\draw[pstyle4] (54.6pt,1021.2227pt) -- (54.6pt,1048.2227pt)(41.6pt,1029.2227pt) -- (67.6pt,1029.2227pt)(54.6pt,1048.2227pt) -- (41.6pt,1063.2227pt)(54.6pt,1048.2227pt) -- (67.6pt,1063.2227pt);
\draw[pstyle3] (352.1879pt,991.9766pt) arc (180:270:5pt) -- (357.1879pt,986.9766pt) -- (472.456pt,986.9766pt) arc (270:360:5pt) -- (477.456pt,991.9766pt) -- (477.456pt,1013.7227pt) arc (0:90:5pt) -- (472.456pt,1018.7227pt) -- (357.1879pt,1018.7227pt) arc (90:180:5pt) -- (352.1879pt,1013.7227pt) -- cycle;
\node at (359.1879pt,993.9766pt)[below right,color=black]{a MainController};
\draw[pstyle3] (598.156pt,991.9766pt) arc (180:270:5pt) -- (603.156pt,986.9766pt) -- (716.8603pt,986.9766pt) arc (270:360:5pt) -- (721.8603pt,991.9766pt) -- (721.8603pt,1013.7227pt) arc (0:90:5pt) -- (716.8603pt,1018.7227pt) -- (603.156pt,1018.7227pt) arc (90:180:5pt) -- (598.156pt,1013.7227pt) -- cycle;
\node at (605.156pt,993.9766pt)[below right,color=black]{an AddressBook};
\draw[pstyle3] (731.8603pt,991.9766pt) arc (180:270:5pt) -- (736.8603pt,986.9766pt) -- (804.327pt,986.9766pt) arc (270:360:5pt) -- (809.327pt,991.9766pt) -- (809.327pt,1013.7227pt) arc (0:90:5pt) -- (804.327pt,1018.7227pt) -- (736.8603pt,1018.7227pt) arc (90:180:5pt) -- (731.8603pt,1013.7227pt) -- cycle;
\node at (738.8603pt,993.9766pt)[below right,color=black]{a Contact};
\draw[pstyle0] (409.8219pt,165.9707pt) rectangle (419.8219pt,274.6738pt);
\draw[pstyle0] (414.8219pt,211.4492pt) rectangle (424.8219pt,244.1953pt);
\draw[pstyle0] (409.8219pt,369.1094pt) rectangle (419.8219pt,404.5879pt);
\draw[pstyle0] (409.8219pt,499.0234pt) rectangle (419.8219pt,527.0234pt);
\draw[pstyle0] (409.8219pt,642.459pt) rectangle (419.8219pt,738.8945pt);
\draw[pstyle0] (409.8219pt,823.7949pt) rectangle (419.8219pt,950.709pt);
\draw[pstyle0] (655.0081pt,211.4492pt) rectangle (665.0081pt,244.1953pt);
\draw[pstyle5] (340.1879pt,111.2246pt) -- (350.1879pt,115.2246pt) -- (340.1879pt,119.2246pt) -- (344.1879pt,115.2246pt) -- cycle;
\draw[pstyle6] (54.6pt,115.2246pt) -- (346.1879pt,115.2246pt);
\node at (61.6pt,96.7461pt)[below right,color=black]{Apre rubrica};
\draw[pstyle3] (352.1879pt,97.7461pt) arc (180:270:5pt) -- (357.1879pt,92.7461pt) -- (472.456pt,92.7461pt) arc (270:360:5pt) -- (477.456pt,97.7461pt) -- (477.456pt,119.4922pt) arc (0:90:5pt) -- (472.456pt,124.4922pt) -- (357.1879pt,124.4922pt) arc (90:180:5pt) -- (352.1879pt,119.4922pt) -- cycle;
\node at (359.1879pt,99.7461pt)[below right,color=black]{a MainController};
\draw[pstyle6] (414.8219pt,152.9707pt) -- (461.8219pt,152.9707pt);
\draw[pstyle6] (461.8219pt,152.9707pt) -- (461.8219pt,165.9707pt);
\draw[pstyle6] (420.8219pt,165.9707pt) -- (461.8219pt,165.9707pt);
\draw[pstyle5] (430.8219pt,161.9707pt) -- (420.8219pt,165.9707pt) -- (430.8219pt,169.9707pt) -- (426.8219pt,165.9707pt) -- cycle;
\node at (426.8219pt,134.4922pt)[below right,color=black]{initialize(...)};
\draw[pstyle5] (586.156pt,197.4492pt) -- (596.156pt,201.4492pt) -- (586.156pt,205.4492pt) -- (590.156pt,201.4492pt) -- cycle;
\draw[pstyle6] (419.8219pt,201.4492pt) -- (592.156pt,201.4492pt);
\node at (426.8219pt,182.9707pt)[below right,color=black]{getInstance()};
\draw[pstyle3] (598.156pt,183.9707pt) arc (180:270:5pt) -- (603.156pt,178.9707pt) -- (716.8603pt,178.9707pt) arc (270:360:5pt) -- (721.8603pt,183.9707pt) -- (721.8603pt,205.7168pt) arc (0:90:5pt) -- (716.8603pt,210.7168pt) -- (603.156pt,210.7168pt) arc (90:180:5pt) -- (598.156pt,205.7168pt) -- cycle;
\node at (605.156pt,185.9707pt)[below right,color=black]{an AddressBook};
\draw[pstyle5] (430.8219pt,240.1953pt) -- (420.8219pt,244.1953pt) -- (430.8219pt,248.1953pt) -- (426.8219pt,244.1953pt) -- cycle;
\draw[pstyle7] (424.8219pt,244.1953pt) -- (659.0081pt,244.1953pt);
\node at (436.8219pt,225.7168pt)[below right,color=black]{lista dei contatti};
\draw[pstyle5] (65.6pt,270.6738pt) -- (55.6pt,274.6738pt) -- (65.6pt,278.6738pt) -- (61.6pt,274.6738pt) -- cycle;
\draw[pstyle7] (59.6pt,274.6738pt) -- (413.8219pt,274.6738pt);
\node at (71.6pt,256.1953pt)[below right,color=black]{mostra la lista dei contatti};
\draw[pstyle8] (10pt,289.6738pt) -- (81.4pt,289.6738pt) -- (81.4pt,298.1523pt) -- (71.4pt,308.1523pt) -- (10pt,308.1523pt) -- (10pt,289.6738pt);
\draw[pstyle1] (10pt,289.6738pt) rectangle (680.0081pt,549.0234pt);
\node at (25pt,290.6738pt)[below right,color=black]{\textbf{loop}};
\node at (96.4pt,291.6738pt)[below right,color=black]{\textbf{[Utente digita nel TextField searchField una sottostringa o sceglie un tag]}};
\draw[pstyle5] (402.8219pt,326.6309pt) -- (412.8219pt,330.6309pt) -- (402.8219pt,334.6309pt) -- (406.8219pt,330.6309pt) -- cycle;
\draw[pstyle6] (54.6pt,330.6309pt) -- (408.8219pt,330.6309pt);
\node at (61.6pt,312.1523pt)[below right,color=black]{Digita una sottostringa nella barra di ricerca};
\draw[pstyle6] (414.8219pt,356.1094pt) -- (461.8219pt,356.1094pt);
\draw[pstyle6] (461.8219pt,356.1094pt) -- (461.8219pt,369.1094pt);
\draw[pstyle6] (420.8219pt,369.1094pt) -- (461.8219pt,369.1094pt);
\draw[pstyle5] (430.8219pt,365.1094pt) -- (420.8219pt,369.1094pt) -- (430.8219pt,373.1094pt) -- (426.8219pt,369.1094pt) -- cycle;
\node at (426.8219pt,337.6309pt)[below right,color=black]{searchFieldBinding()};
\draw[pstyle5] (65.6pt,400.5879pt) -- (55.6pt,404.5879pt) -- (65.6pt,408.5879pt) -- (61.6pt,404.5879pt) -- cycle;
\draw[pstyle7] (59.6pt,404.5879pt) -- (413.8219pt,404.5879pt);
\node at (71.6pt,386.1094pt)[below right,color=black]{Restituisce i contatti che rispecchiano i criteri di ricerca};
\draw[pstyle8] (20pt,419.5879pt) -- (101.9pt,419.5879pt) -- (101.9pt,428.0664pt) -- (91.9pt,438.0664pt) -- (20pt,438.0664pt) -- (20pt,419.5879pt);
\draw[pstyle1] (20pt,419.5879pt) rectangle (670.0081pt,542.0234pt);
\node at (35pt,420.5879pt)[below right,color=black]{\textbf{break}};
\node at (116.9pt,421.5879pt)[below right,color=black]{\textbf{[Quando l'utente clicca su un contatto]}};
\draw[pstyle5] (402.8219pt,456.5449pt) -- (412.8219pt,460.5449pt) -- (402.8219pt,464.5449pt) -- (406.8219pt,460.5449pt) -- cycle;
\draw[pstyle6] (54.6pt,460.5449pt) -- (408.8219pt,460.5449pt);
\node at (61.6pt,442.0664pt)[below right,color=black]{clicca su un contatto};
\draw[pstyle6] (419.8219pt,491.0234pt) -- (461.8219pt,491.0234pt);
\draw[pstyle6] (461.8219pt,491.0234pt) -- (461.8219pt,504.0234pt);
\draw[pstyle6] (420.8219pt,504.0234pt) -- (461.8219pt,504.0234pt);
\draw[pstyle5] (430.8219pt,500.0234pt) -- (420.8219pt,504.0234pt) -- (430.8219pt,508.0234pt) -- (426.8219pt,504.0234pt) -- cycle;
\node at (426.8219pt,472.5449pt)[below right,color=black]{onContactClicked(MouseEvent event)};
\draw[pstyle8] (20pt,563.0234pt) -- (81.8pt,563.0234pt) -- (81.8pt,571.502pt) -- (71.8pt,581.502pt) -- (20pt,581.502pt) -- (20pt,563.0234pt);
\draw[pstyle1] (20pt,563.0234pt) rectangle (819.327pt,970.9766pt);
\node at (35pt,564.0234pt)[below right,color=black]{\textbf{alt}};
\node at (96.8pt,565.0234pt)[below right,color=black]{\textbf{[utente clicca su "Elimina"]}};
\draw[pstyle5] (402.8219pt,599.9805pt) -- (412.8219pt,603.9805pt) -- (402.8219pt,607.9805pt) -- (406.8219pt,603.9805pt) -- cycle;
\draw[pstyle6] (54.6pt,603.9805pt) -- (408.8219pt,603.9805pt);
\node at (61.6pt,585.502pt)[below right,color=black]{utente clicca su "Elimina"};
\draw[pstyle6] (414.8219pt,629.459pt) -- (461.8219pt,629.459pt);
\draw[pstyle6] (461.8219pt,629.459pt) -- (461.8219pt,642.459pt);
\draw[pstyle6] (420.8219pt,642.459pt) -- (461.8219pt,642.459pt);
\draw[pstyle5] (430.8219pt,638.459pt) -- (420.8219pt,642.459pt) -- (430.8219pt,646.459pt) -- (426.8219pt,642.459pt) -- cycle;
\node at (426.8219pt,610.9805pt)[below right,color=black]{onRemoveContact(...)};
\draw[pstyle5] (65.6pt,673.9375pt) -- (55.6pt,677.9375pt) -- (65.6pt,681.9375pt) -- (61.6pt,677.9375pt) -- cycle;
\draw[pstyle6] (59.6pt,677.9375pt) -- (408.8219pt,677.9375pt);
\node at (71.6pt,659.459pt)[below right,color=black]{mostra la schermata di conferma};
\draw[pstyle5] (397.8219pt,704.416pt) -- (407.8219pt,708.416pt) -- (397.8219pt,712.416pt) -- (401.8219pt,708.416pt) -- cycle;
\draw[pstyle6] (54.6pt,708.416pt) -- (403.8219pt,708.416pt);
\node at (61.6pt,689.9375pt)[below right,color=black]{conferma l'operazione};
\draw[pstyle5] (648.0081pt,734.8945pt) -- (658.0081pt,738.8945pt) -- (648.0081pt,742.8945pt) -- (652.0081pt,738.8945pt) -- cycle;
\draw[pstyle6] (414.8219pt,738.8945pt) -- (654.0081pt,738.8945pt);
\node at (421.8219pt,720.416pt)[below right,color=black]{removeContact(Contact c)};
\draw[color=black,line width=1.0pt,dash pattern=on 2.0pt off 2.0pt] (20pt,747.8945pt) -- (819.327pt,747.8945pt);
\node at (25pt,747.8945pt)[below right,color=black]{\textbf{[utente clicca su "Modifica"]}};
\draw[pstyle5] (402.8219pt,781.3164pt) -- (412.8219pt,785.3164pt) -- (402.8219pt,789.3164pt) -- (406.8219pt,785.3164pt) -- cycle;
\draw[pstyle6] (54.6pt,785.3164pt) -- (408.8219pt,785.3164pt);
\node at (61.6pt,766.8379pt)[below right,color=black]{utente clicca su "Modifica"};
\draw[pstyle6] (414.8219pt,810.7949pt) -- (461.8219pt,810.7949pt);
\draw[pstyle6] (461.8219pt,810.7949pt) -- (461.8219pt,823.7949pt);
\draw[pstyle6] (420.8219pt,823.7949pt) -- (461.8219pt,823.7949pt);
\draw[pstyle5] (430.8219pt,819.7949pt) -- (420.8219pt,823.7949pt) -- (430.8219pt,827.7949pt) -- (426.8219pt,823.7949pt) -- cycle;
\node at (426.8219pt,792.3164pt)[below right,color=black]{onModifyContact(...)};
\draw[pstyle5] (65.6pt,855.2734pt) -- (55.6pt,859.2734pt) -- (65.6pt,863.2734pt) -- (61.6pt,859.2734pt) -- cycle;
\draw[pstyle6] (59.6pt,859.2734pt) -- (408.8219pt,859.2734pt);
\node at (71.6pt,840.7949pt)[below right,color=black]{rende modificabili i campi};
\draw[pstyle5] (397.8219pt,885.752pt) -- (407.8219pt,889.752pt) -- (397.8219pt,893.752pt) -- (401.8219pt,889.752pt) -- cycle;
\draw[pstyle6] (54.6pt,889.752pt) -- (403.8219pt,889.752pt);
\node at (61.6pt,871.2734pt)[below right,color=black]{modifica qualsiasi campo};
\draw[pstyle5] (397.8219pt,916.2305pt) -- (407.8219pt,920.2305pt) -- (397.8219pt,924.2305pt) -- (401.8219pt,920.2305pt) -- cycle;
\draw[pstyle6] (54.6pt,920.2305pt) -- (403.8219pt,920.2305pt);
\node at (61.6pt,901.752pt)[below right,color=black]{clicca "Salva"};
\draw[pstyle5] (719.8603pt,946.709pt) -- (729.8603pt,950.709pt) -- (719.8603pt,954.709pt) -- (723.8603pt,950.709pt) -- cycle;
\draw[pstyle6] (414.8219pt,950.709pt) -- (725.8603pt,950.709pt);
\node at (421.8219pt,932.2305pt)[below right,color=black]{modifica il contatto};
\draw[pstyle3] (731.8603pt,933.2305pt) arc (180:270:5pt) -- (736.8603pt,928.2305pt) -- (804.327pt,928.2305pt) arc (270:360:5pt) -- (809.327pt,933.2305pt) -- (809.327pt,954.9766pt) arc (0:90:5pt) -- (804.327pt,959.9766pt) -- (736.8603pt,959.9766pt) arc (90:180:5pt) -- (731.8603pt,954.9766pt) -- cycle;
\node at (738.8603pt,935.2305pt)[below right,color=black]{a Contact};
\end{tikzpicture}
}
\end{adjustbox}
\begin{figure}[h]
\caption{Diagramma sequenza C2 C3 - Eliminare Modificare Contatto}
\label{fig:Diagramma sequenza C2 C3 - Eliminare Modificare Contatto}
\end{figure}

\newpage
\subsubsection{C5 - Importare rubrica}
Il seguente diagramma di sequenza illustra l'esecuzione del caso d'uso C5:
% generated by Plantuml 1.2024.3       
\definecolor{plantucolor0000}{RGB}{255,255,255}
\definecolor{plantucolor0001}{RGB}{24,24,24}
\definecolor{plantucolor0002}{RGB}{0,0,0}
\definecolor{plantucolor0003}{RGB}{226,226,240}
\definecolor{plantucolor0004}{RGB}{238,238,238}
\definecolor{plantucolor0005}{RGB}{168,0,54}

\begin{adjustbox}{width=.9\paperwidth, center}
	\resizebox{\textwidth}{!}{
\begin{tikzpicture}[yscale=-1
,pstyle0/.style={color=plantucolor0001,fill=white,line width=1.0pt}
,pstyle1/.style={color=black,line width=1.5pt}
,pstyle2/.style={color=plantucolor0001,line width=0.5pt,dash pattern=on 5.0pt off 5.0pt}
,pstyle3/.style={color=plantucolor0001,fill=plantucolor0003,line width=0.5pt}
,pstyle4/.style={color=plantucolor0001,line width=0.5pt}
,pstyle5/.style={color=plantucolor0001,fill=plantucolor0001,line width=1.0pt}
,pstyle6/.style={color=plantucolor0001,line width=1.0pt}
,pstyle7/.style={color=plantucolor0001,line width=1.0pt,dash pattern=on 2.0pt off 2.0pt}
,pstyle10/.style={color=plantucolor0005,line width=2.0pt}
]
\draw[pstyle0] (266.4911pt,165.9707pt) rectangle (276.4911pt,244.1953pt);
\draw[pstyle0] (271.4911pt,211.4492pt) rectangle (281.4911pt,244.1953pt);
\draw[pstyle0] (266.4911pt,274.6738pt) rectangle (276.4911pt,305.1523pt);
\draw[pstyle0] (522.4243pt,355.8984pt) rectangle (532.4243pt,383.8984pt);
\draw[pstyle0] (522.4243pt,459.8555pt) rectangle (532.4243pt,487.8555pt);
\draw[pstyle0] (522.4243pt,558.8125pt) rectangle (532.4243pt,586.8125pt);
\draw[pstyle0] (522.4243pt,648.2344pt) rectangle (532.4243pt,668.2344pt);
\draw[pstyle0] (522.4243pt,729.1914pt) rectangle (532.4243pt,795.1484pt);
\draw[pstyle0] (685.7338pt,211.4492pt) rectangle (695.7338pt,244.1953pt);
\draw[pstyle1] (425.967pt,509.8555pt) rectangle (680.334pt,661.2344pt);
\draw[pstyle2] (29pt,82.7461pt) -- (29pt,813.1484pt);
\draw[pstyle2] (270.8571pt,124.1191pt) -- (270.8571pt,813.1484pt);
\draw[pstyle2] (526.967pt,314.0469pt) -- (526.967pt,813.1484pt);
\draw[pstyle2] (689.8816pt,210.3438pt) -- (689.8816pt,813.1484pt);
\node at (5pt,65pt)[below right,color=black]{Utente};
\draw[pstyle3] (29.6pt,13.5pt) ellipse (8pt and 8pt);
\draw[pstyle4] (29.6pt,21.5pt) -- (29.6pt,48.5pt)(16.6pt,29.5pt) -- (42.6pt,29.5pt)(29.6pt,48.5pt) -- (16.6pt,63.5pt)(29.6pt,48.5pt) -- (42.6pt,63.5pt);
\node at (5pt,812.1484pt)[below right,color=black]{Utente};
\draw[pstyle3] (29.6pt,838.3945pt) ellipse (8pt and 8pt);
\draw[pstyle4] (29.6pt,846.3945pt) -- (29.6pt,873.3945pt)(16.6pt,854.3945pt) -- (42.6pt,854.3945pt)(29.6pt,873.3945pt) -- (16.6pt,888.3945pt)(29.6pt,873.3945pt) -- (42.6pt,888.3945pt);
\draw[pstyle3] (208.8571pt,817.1484pt) arc (180:270:5pt) -- (213.8571pt,812.1484pt) -- (329.1251pt,812.1484pt) arc (270:360:5pt) -- (334.1251pt,817.1484pt) -- (334.1251pt,838.8945pt) arc (0:90:5pt) -- (329.1251pt,843.8945pt) -- (213.8571pt,843.8945pt) arc (90:180:5pt) -- (208.8571pt,838.8945pt) -- cycle;
\node at (215.8571pt,819.1484pt)[below right,color=black]{a MainController};
\draw[pstyle3] (435.967pt,817.1484pt) arc (180:270:5pt) -- (440.967pt,812.1484pt) -- (613.8816pt,812.1484pt) arc (270:360:5pt) -- (618.8816pt,817.1484pt) -- (618.8816pt,838.8945pt) arc (0:90:5pt) -- (613.8816pt,843.8945pt) -- (440.967pt,843.8945pt) arc (90:180:5pt) -- (435.967pt,838.8945pt) -- cycle;
\node at (442.967pt,819.1484pt)[below right,color=black]{an ImportPopupContoller};
\draw[pstyle3] (628.8816pt,817.1484pt) arc (180:270:5pt) -- (633.8816pt,812.1484pt) -- (747.586pt,812.1484pt) arc (270:360:5pt) -- (752.586pt,817.1484pt) -- (752.586pt,838.8945pt) arc (0:90:5pt) -- (747.586pt,843.8945pt) -- (633.8816pt,843.8945pt) arc (90:180:5pt) -- (628.8816pt,838.8945pt) -- cycle;
\node at (635.8816pt,819.1484pt)[below right,color=black]{an AddressBook};
\draw[pstyle0] (266.4911pt,165.9707pt) rectangle (276.4911pt,244.1953pt);
\draw[pstyle0] (271.4911pt,211.4492pt) rectangle (281.4911pt,244.1953pt);
\draw[pstyle0] (266.4911pt,274.6738pt) rectangle (276.4911pt,305.1523pt);
\draw[pstyle0] (522.4243pt,355.8984pt) rectangle (532.4243pt,383.8984pt);
\draw[pstyle0] (522.4243pt,459.8555pt) rectangle (532.4243pt,487.8555pt);
\draw[pstyle0] (522.4243pt,558.8125pt) rectangle (532.4243pt,586.8125pt);
\draw[pstyle0] (522.4243pt,648.2344pt) rectangle (532.4243pt,668.2344pt);
\draw[pstyle0] (522.4243pt,729.1914pt) rectangle (532.4243pt,795.1484pt);
\draw[pstyle0] (685.7338pt,211.4492pt) rectangle (695.7338pt,244.1953pt);
\draw[pstyle5] (196.8571pt,111.2246pt) -- (206.8571pt,115.2246pt) -- (196.8571pt,119.2246pt) -- (200.8571pt,115.2246pt) -- cycle;
\draw[pstyle6] (29.6pt,115.2246pt) -- (202.8571pt,115.2246pt);
\node at (36.6pt,96.7461pt)[below right,color=black]{Apre rubrica};
\draw[pstyle3] (208.8571pt,97.7461pt) arc (180:270:5pt) -- (213.8571pt,92.7461pt) -- (329.1251pt,92.7461pt) arc (270:360:5pt) -- (334.1251pt,97.7461pt) -- (334.1251pt,119.4922pt) arc (0:90:5pt) -- (329.1251pt,124.4922pt) -- (213.8571pt,124.4922pt) arc (90:180:5pt) -- (208.8571pt,119.4922pt) -- cycle;
\node at (215.8571pt,99.7461pt)[below right,color=black]{a MainController};
\draw[pstyle6] (271.4911pt,152.9707pt) -- (318.4911pt,152.9707pt);
\draw[pstyle6] (318.4911pt,152.9707pt) -- (318.4911pt,165.9707pt);
\draw[pstyle6] (277.4911pt,165.9707pt) -- (318.4911pt,165.9707pt);
\draw[pstyle5] (287.4911pt,161.9707pt) -- (277.4911pt,165.9707pt) -- (287.4911pt,169.9707pt) -- (283.4911pt,165.9707pt) -- cycle;
\node at (283.4911pt,134.4922pt)[below right,color=black]{initialize(...)};
\draw[pstyle5] (616.8816pt,197.4492pt) -- (626.8816pt,201.4492pt) -- (616.8816pt,205.4492pt) -- (620.8816pt,201.4492pt) -- cycle;
\draw[pstyle6] (276.4911pt,201.4492pt) -- (622.8816pt,201.4492pt);
\node at (283.4911pt,182.9707pt)[below right,color=black]{getInstance()};
\draw[pstyle3] (628.8816pt,183.9707pt) arc (180:270:5pt) -- (633.8816pt,178.9707pt) -- (747.586pt,178.9707pt) arc (270:360:5pt) -- (752.586pt,183.9707pt) -- (752.586pt,205.7168pt) arc (0:90:5pt) -- (747.586pt,210.7168pt) -- (633.8816pt,210.7168pt) arc (90:180:5pt) -- (628.8816pt,205.7168pt) -- cycle;
\node at (635.8816pt,185.9707pt)[below right,color=black]{an AddressBook};
\draw[pstyle5] (282.4911pt,240.1953pt) -- (272.4911pt,244.1953pt) -- (282.4911pt,248.1953pt) -- (278.4911pt,244.1953pt) -- cycle;
\draw[pstyle7] (276.4911pt,244.1953pt) -- (689.7338pt,244.1953pt);
\node at (288.4911pt,225.7168pt)[below right,color=black]{lista dei contatti};
\draw[pstyle5] (254.4911pt,270.6738pt) -- (264.4911pt,274.6738pt) -- (254.4911pt,278.6738pt) -- (258.4911pt,274.6738pt) -- cycle;
\draw[pstyle6] (29.6pt,274.6738pt) -- (260.4911pt,274.6738pt);
\node at (36.6pt,256.1953pt)[below right,color=black]{Clicca "Importa" dal menù a tendina};
\draw[pstyle5] (423.967pt,301.1523pt) -- (433.967pt,305.1523pt) -- (423.967pt,309.1523pt) -- (427.967pt,305.1523pt) -- cycle;
\draw[pstyle6] (271.4911pt,305.1523pt) -- (429.967pt,305.1523pt);
\node at (278.4911pt,286.6738pt)[below right,color=black]{showImportPopup(...)};
\draw[pstyle3] (435.967pt,287.6738pt) arc (180:270:5pt) -- (440.967pt,282.6738pt) -- (613.8816pt,282.6738pt) arc (270:360:5pt) -- (618.8816pt,287.6738pt) -- (618.8816pt,309.4199pt) arc (0:90:5pt) -- (613.8816pt,314.4199pt) -- (440.967pt,314.4199pt) arc (90:180:5pt) -- (435.967pt,309.4199pt) -- cycle;
\node at (442.967pt,289.6738pt)[below right,color=black]{an ImportPopupContoller};
\draw[pstyle6] (532.4243pt,347.8984pt) -- (574.4243pt,347.8984pt);
\draw[pstyle6] (574.4243pt,347.8984pt) -- (574.4243pt,360.8984pt);
\draw[pstyle6] (533.4243pt,360.8984pt) -- (574.4243pt,360.8984pt);
\draw[pstyle5] (543.4243pt,356.8984pt) -- (533.4243pt,360.8984pt) -- (543.4243pt,364.8984pt) -- (539.4243pt,360.8984pt) -- cycle;
\node at (539.4243pt,329.4199pt)[below right,color=black]{initialize(...)};
\draw[pstyle5] (515.4243pt,417.377pt) -- (525.4243pt,421.377pt) -- (515.4243pt,425.377pt) -- (519.4243pt,421.377pt) -- cycle;
\draw[pstyle6] (29.6pt,421.377pt) -- (521.4243pt,421.377pt);
\node at (36.6pt,402.8984pt)[below right,color=black]{Inserisce percorso file .csv o .vcf};
\draw[pstyle6] (532.4243pt,451.8555pt) -- (574.4243pt,451.8555pt);
\draw[pstyle6] (574.4243pt,451.8555pt) -- (574.4243pt,464.8555pt);
\draw[pstyle6] (533.4243pt,464.8555pt) -- (574.4243pt,464.8555pt);
\draw[pstyle5] (543.4243pt,460.8555pt) -- (533.4243pt,464.8555pt) -- (543.4243pt,468.8555pt) -- (539.4243pt,464.8555pt) -- cycle;
\node at (539.4243pt,433.377pt)[below right,color=black]{choosePath(...)};
\draw[color=black,fill=plantucolor0004,line width=1.5pt] (425.967pt,509.8555pt) -- (487.767pt,509.8555pt) -- (487.767pt,518.334pt) -- (477.767pt,528.334pt) -- (425.967pt,528.334pt) -- (425.967pt,509.8555pt);
\draw[pstyle1] (425.967pt,509.8555pt) rectangle (680.334pt,661.2344pt);
\node at (440.967pt,510.8555pt)[below right,color=black]{\textbf{alt}};
\node at (502.767pt,511.8555pt)[below right,color=black]{\textbf{[Il file scelto è in formato .csv]}};
\draw[pstyle6] (532.4243pt,550.8125pt) -- (574.4243pt,550.8125pt);
\draw[pstyle6] (574.4243pt,550.8125pt) -- (574.4243pt,563.8125pt);
\draw[pstyle6] (533.4243pt,563.8125pt) -- (574.4243pt,563.8125pt);
\draw[pstyle5] (543.4243pt,559.8125pt) -- (533.4243pt,563.8125pt) -- (543.4243pt,567.8125pt) -- (539.4243pt,563.8125pt) -- cycle;
\node at (539.4243pt,532.334pt)[below right,color=black]{checkCSVFormat();};
\draw[color=black,line width=1.0pt,dash pattern=on 2.0pt off 2.0pt] (425.967pt,602.8125pt) -- (680.334pt,602.8125pt);
\node at (430.967pt,602.8125pt)[below right,color=black]{\textbf{[Il file scelto è in formato .vcf]}};
\draw[pstyle6] (527.4243pt,635.2344pt) -- (574.4243pt,635.2344pt);
\draw[pstyle6] (574.4243pt,635.2344pt) -- (574.4243pt,648.2344pt);
\draw[pstyle6] (533.4243pt,648.2344pt) -- (574.4243pt,648.2344pt);
\draw[pstyle5] (543.4243pt,644.2344pt) -- (533.4243pt,648.2344pt) -- (543.4243pt,652.2344pt) -- (539.4243pt,648.2344pt) -- cycle;
\node at (539.4243pt,616.7559pt)[below right,color=black]{checkVCardFormat();};
\draw[pstyle5] (515.4243pt,686.7129pt) -- (525.4243pt,690.7129pt) -- (515.4243pt,694.7129pt) -- (519.4243pt,690.7129pt) -- cycle;
\draw[pstyle6] (29.6pt,690.7129pt) -- (521.4243pt,690.7129pt);
\node at (36.6pt,672.2344pt)[below right,color=black]{Utente clicca il pulsante "Importa"};
\draw[pstyle6] (527.4243pt,716.1914pt) -- (574.4243pt,716.1914pt);
\draw[pstyle6] (574.4243pt,716.1914pt) -- (574.4243pt,729.1914pt);
\draw[pstyle6] (533.4243pt,729.1914pt) -- (574.4243pt,729.1914pt);
\draw[pstyle5] (543.4243pt,725.1914pt) -- (533.4243pt,729.1914pt) -- (543.4243pt,733.1914pt) -- (539.4243pt,729.1914pt) -- cycle;
\node at (539.4243pt,697.7129pt)[below right,color=black]{onImport(...)};
\draw[pstyle5] (678.7338pt,760.6699pt) -- (688.7338pt,764.6699pt) -- (678.7338pt,768.6699pt) -- (682.7338pt,764.6699pt) -- cycle;
\draw[pstyle6] (532.4243pt,764.6699pt) -- (684.7338pt,764.6699pt);
\node at (539.4243pt,746.1914pt)[below right,color=black]{Importa rubrica};
\draw[pstyle5] (40.6pt,791.1484pt) -- (30.6pt,795.1484pt) -- (40.6pt,799.1484pt) -- (36.6pt,795.1484pt) -- cycle;
\draw[pstyle7] (34.6pt,795.1484pt) -- (526.4243pt,795.1484pt);
\node at (46.6pt,776.6699pt)[below right,color=black]{mostra la lista dei contatti aggiornata};
\draw[pstyle10] (518.4243pt,786.1484pt) -- (536.4243pt,804.1484pt);
\draw[pstyle10] (518.4243pt,804.1484pt) -- (536.4243pt,786.1484pt);
\end{tikzpicture}
}
\end{adjustbox}

\begin{figure}[h]
	\caption{Diagramma sequenza C5 - Importare rubrica}
	\label{fig:Diagramma sequenza C5 - Importare rubrica}
\end{figure}

\newpage
\subsubsection{C6 - Esportare rubrica}
Il seguente diagramma di sequenza illustra l'esecuzione del caso d'uso C6:
% generated by Plantuml 1.2024.3       
\definecolor{plantucolor0000}{RGB}{255,255,255}
\definecolor{plantucolor0001}{RGB}{24,24,24}
\definecolor{plantucolor0002}{RGB}{0,0,0}
\definecolor{plantucolor0003}{RGB}{226,226,240}
\definecolor{plantucolor0004}{RGB}{238,238,238}
\definecolor{plantucolor0005}{RGB}{168,0,54}

\begin{adjustbox}{width=.7\paperwidth, center}
	\resizebox{\textwidth}{!}{
\begin{tikzpicture}[yscale=-1
,pstyle0/.style={color=plantucolor0001,fill=white,line width=1.0pt}
,pstyle1/.style={color=black,line width=1.5pt}
,pstyle2/.style={color=plantucolor0001,line width=0.5pt,dash pattern=on 5.0pt off 5.0pt}
,pstyle3/.style={color=plantucolor0001,fill=plantucolor0003,line width=0.5pt}
,pstyle4/.style={color=plantucolor0001,line width=0.5pt}
,pstyle5/.style={color=plantucolor0001,fill=plantucolor0001,line width=1.0pt}
,pstyle6/.style={color=plantucolor0001,line width=1.0pt}
,pstyle9/.style={color=plantucolor0005,line width=2.0pt}
]
\draw[pstyle0] (187.6937pt,165.9707pt) rectangle (197.6937pt,193.9707pt);
\draw[pstyle0] (187.6937pt,231.4492pt) rectangle (197.6937pt,261.9277pt);
\draw[pstyle0] (444.4229pt,312.6738pt) rectangle (454.4229pt,340.6738pt);
\draw[pstyle0] (444.4229pt,386.1523pt) rectangle (454.4229pt,421.6309pt);
\draw[pstyle0] (444.4229pt,460.1094pt) rectangle (454.4229pt,488.1094pt);
\draw[pstyle0] (444.4229pt,564.0664pt) rectangle (454.4229pt,800.9238pt);
\draw[pstyle0] (449.4229pt,633.0234pt) rectangle (459.4229pt,661.0234pt);
\draw[pstyle0] (449.4229pt,722.4453pt) rectangle (459.4229pt,750.4453pt);
\draw[pstyle1] (345.9372pt,584.0664pt) rectangle (592.5535pt,765.4453pt);
\draw[pstyle2] (29pt,82.7461pt) -- (29pt,825.9238pt);
\draw[pstyle2] (192.0597pt,124.1191pt) -- (192.0597pt,825.9238pt);
\draw[pstyle2] (448.9372pt,270.8223pt) -- (448.9372pt,825.9238pt);
\node at (5pt,65pt)[below right,color=black]{Utente};
\draw[pstyle3] (29.6pt,13.5pt) ellipse (8pt and 8pt);
\draw[pstyle4] (29.6pt,21.5pt) -- (29.6pt,48.5pt)(16.6pt,29.5pt) -- (42.6pt,29.5pt)(29.6pt,48.5pt) -- (16.6pt,63.5pt)(29.6pt,48.5pt) -- (42.6pt,63.5pt);
\node at (5pt,824.9238pt)[below right,color=black]{Utente};
\draw[pstyle3] (29.6pt,851.1699pt) ellipse (8pt and 8pt);
\draw[pstyle4] (29.6pt,859.1699pt) -- (29.6pt,886.1699pt)(16.6pt,867.1699pt) -- (42.6pt,867.1699pt)(29.6pt,886.1699pt) -- (16.6pt,901.1699pt)(29.6pt,886.1699pt) -- (42.6pt,901.1699pt);
\draw[pstyle3] (130.0597pt,829.9238pt) arc (180:270:5pt) -- (135.0597pt,824.9238pt) -- (250.3278pt,824.9238pt) arc (270:360:5pt) -- (255.3278pt,829.9238pt) -- (255.3278pt,851.6699pt) arc (0:90:5pt) -- (250.3278pt,856.6699pt) -- (135.0597pt,856.6699pt) arc (90:180:5pt) -- (130.0597pt,851.6699pt) -- cycle;
\node at (137.0597pt,831.9238pt)[below right,color=black]{a MainController};
\draw[pstyle3] (355.9372pt,829.9238pt) arc (180:270:5pt) -- (360.9372pt,824.9238pt) -- (537.9087pt,824.9238pt) arc (270:360:5pt) -- (542.9087pt,829.9238pt) -- (542.9087pt,851.6699pt) arc (0:90:5pt) -- (537.9087pt,856.6699pt) -- (360.9372pt,856.6699pt) arc (90:180:5pt) -- (355.9372pt,851.6699pt) -- cycle;
\node at (362.9372pt,831.9238pt)[below right,color=black]{an ExportPopupController};
\draw[pstyle0] (187.6937pt,165.9707pt) rectangle (197.6937pt,193.9707pt);
\draw[pstyle0] (187.6937pt,231.4492pt) rectangle (197.6937pt,261.9277pt);
\draw[pstyle0] (444.4229pt,312.6738pt) rectangle (454.4229pt,340.6738pt);
\draw[pstyle0] (444.4229pt,386.1523pt) rectangle (454.4229pt,421.6309pt);
\draw[pstyle0] (444.4229pt,460.1094pt) rectangle (454.4229pt,488.1094pt);
\draw[pstyle0] (444.4229pt,564.0664pt) rectangle (454.4229pt,800.9238pt);
\draw[pstyle0] (449.4229pt,633.0234pt) rectangle (459.4229pt,661.0234pt);
\draw[pstyle0] (449.4229pt,722.4453pt) rectangle (459.4229pt,750.4453pt);
\draw[pstyle5] (118.0597pt,111.2246pt) -- (128.0597pt,115.2246pt) -- (118.0597pt,119.2246pt) -- (122.0597pt,115.2246pt) -- cycle;
\draw[pstyle6] (29.6pt,115.2246pt) -- (124.0597pt,115.2246pt);
\node at (36.6pt,96.7461pt)[below right,color=black]{apre Rubrica};
\draw[pstyle3] (130.0597pt,97.7461pt) arc (180:270:5pt) -- (135.0597pt,92.7461pt) -- (250.3278pt,92.7461pt) arc (270:360:5pt) -- (255.3278pt,97.7461pt) -- (255.3278pt,119.4922pt) arc (0:90:5pt) -- (250.3278pt,124.4922pt) -- (135.0597pt,124.4922pt) arc (90:180:5pt) -- (130.0597pt,119.4922pt) -- cycle;
\node at (137.0597pt,99.7461pt)[below right,color=black]{a MainController};
\draw[pstyle6] (197.6937pt,157.9707pt) -- (239.6937pt,157.9707pt);
\draw[pstyle6] (239.6937pt,157.9707pt) -- (239.6937pt,170.9707pt);
\draw[pstyle6] (198.6937pt,170.9707pt) -- (239.6937pt,170.9707pt);
\draw[pstyle5] (208.6937pt,166.9707pt) -- (198.6937pt,170.9707pt) -- (208.6937pt,174.9707pt) -- (204.6937pt,170.9707pt) -- cycle;
\node at (204.6937pt,139.4922pt)[below right,color=black]{initialize(...)};
\draw[pstyle5] (175.6937pt,227.4492pt) -- (185.6937pt,231.4492pt) -- (175.6937pt,235.4492pt) -- (179.6937pt,231.4492pt) -- cycle;
\draw[pstyle6] (29.6pt,231.4492pt) -- (181.6937pt,231.4492pt);
\node at (36.6pt,212.9707pt)[below right,color=black]{clicca "Esporta"};
\draw[pstyle5] (343.9372pt,257.9277pt) -- (353.9372pt,261.9277pt) -- (343.9372pt,265.9277pt) -- (347.9372pt,261.9277pt) -- cycle;
\draw[pstyle6] (192.6937pt,261.9277pt) -- (349.9372pt,261.9277pt);
\node at (199.6937pt,243.4492pt)[below right,color=black]{showExportPopup(...)};
\draw[pstyle3] (355.9372pt,244.4492pt) arc (180:270:5pt) -- (360.9372pt,239.4492pt) -- (537.9087pt,239.4492pt) arc (270:360:5pt) -- (542.9087pt,244.4492pt) -- (542.9087pt,266.1953pt) arc (0:90:5pt) -- (537.9087pt,271.1953pt) -- (360.9372pt,271.1953pt) arc (90:180:5pt) -- (355.9372pt,266.1953pt) -- cycle;
\node at (362.9372pt,246.4492pt)[below right,color=black]{an ExportPopupController};
\draw[pstyle6] (454.4229pt,304.6738pt) -- (496.4229pt,304.6738pt);
\draw[pstyle6] (496.4229pt,304.6738pt) -- (496.4229pt,317.6738pt);
\draw[pstyle6] (455.4229pt,317.6738pt) -- (496.4229pt,317.6738pt);
\draw[pstyle5] (465.4229pt,313.6738pt) -- (455.4229pt,317.6738pt) -- (465.4229pt,321.6738pt) -- (461.4229pt,317.6738pt) -- cycle;
\node at (461.4229pt,286.1953pt)[below right,color=black]{initialize(...)};
\draw[pstyle6] (449.4229pt,373.1523pt) -- (496.4229pt,373.1523pt);
\draw[pstyle6] (496.4229pt,373.1523pt) -- (496.4229pt,386.1523pt);
\draw[pstyle6] (455.4229pt,386.1523pt) -- (496.4229pt,386.1523pt);
\draw[pstyle5] (465.4229pt,382.1523pt) -- (455.4229pt,386.1523pt) -- (465.4229pt,390.1523pt) -- (461.4229pt,386.1523pt) -- cycle;
\node at (461.4229pt,354.6738pt)[below right,color=black]{choosePath(...)};
\draw[pstyle5] (437.4229pt,417.6309pt) -- (447.4229pt,421.6309pt) -- (437.4229pt,425.6309pt) -- (441.4229pt,421.6309pt) -- cycle;
\draw[pstyle6] (29.6pt,421.6309pt) -- (443.4229pt,421.6309pt);
\node at (36.6pt,403.1523pt)[below right,color=black]{sceglie percorso del file};
\draw[pstyle6] (454.4229pt,452.1094pt) -- (496.4229pt,452.1094pt);
\draw[pstyle6] (496.4229pt,452.1094pt) -- (496.4229pt,465.1094pt);
\draw[pstyle6] (455.4229pt,465.1094pt) -- (496.4229pt,465.1094pt);
\draw[pstyle5] (465.4229pt,461.1094pt) -- (455.4229pt,465.1094pt) -- (465.4229pt,469.1094pt) -- (461.4229pt,465.1094pt) -- cycle;
\node at (461.4229pt,433.6309pt)[below right,color=black]{genera file .csv o .vcf};
\draw[pstyle5] (437.4229pt,521.5879pt) -- (447.4229pt,525.5879pt) -- (437.4229pt,529.5879pt) -- (441.4229pt,525.5879pt) -- cycle;
\draw[pstyle6] (29.6pt,525.5879pt) -- (443.4229pt,525.5879pt);
\node at (36.6pt,507.1094pt)[below right,color=black]{clicca sul pulsante "Esporta"};
\draw[pstyle6] (449.4229pt,551.0664pt) -- (496.4229pt,551.0664pt);
\draw[pstyle6] (496.4229pt,551.0664pt) -- (496.4229pt,564.0664pt);
\draw[pstyle6] (455.4229pt,564.0664pt) -- (496.4229pt,564.0664pt);
\draw[pstyle5] (465.4229pt,560.0664pt) -- (455.4229pt,564.0664pt) -- (465.4229pt,568.0664pt) -- (461.4229pt,564.0664pt) -- cycle;
\node at (461.4229pt,532.5879pt)[below right,color=black]{onExport(...)};
\draw[color=black,fill=plantucolor0004,line width=1.5pt] (345.9372pt,584.0664pt) -- (407.7372pt,584.0664pt) -- (407.7372pt,592.5449pt) -- (397.7372pt,602.5449pt) -- (345.9372pt,602.5449pt) -- (345.9372pt,584.0664pt);
\draw[pstyle1] (345.9372pt,584.0664pt) rectangle (592.5535pt,765.4453pt);
\node at (360.9372pt,585.0664pt)[below right,color=black]{\textbf{alt}};
\node at (422.7372pt,586.0664pt)[below right,color=black]{\textbf{[utente ha scelto .csv]}};
\draw[pstyle6] (459.4229pt,625.0234pt) -- (501.4229pt,625.0234pt);
\draw[pstyle6] (501.4229pt,625.0234pt) -- (501.4229pt,638.0234pt);
\draw[pstyle6] (460.4229pt,638.0234pt) -- (501.4229pt,638.0234pt);
\draw[pstyle5] (470.4229pt,634.0234pt) -- (460.4229pt,638.0234pt) -- (470.4229pt,642.0234pt) -- (466.4229pt,638.0234pt) -- cycle;
\node at (466.4229pt,606.5449pt)[below right,color=black]{onExportCSV(...)};
\draw[color=black,line width=1.0pt,dash pattern=on 2.0pt off 2.0pt] (345.9372pt,677.0234pt) -- (592.5535pt,677.0234pt);
\node at (350.9372pt,677.0234pt)[below right,color=black]{\textbf{[utente ha scelto .vcf]}};
\draw[pstyle6] (459.4229pt,714.4453pt) -- (501.4229pt,714.4453pt);
\draw[pstyle6] (501.4229pt,714.4453pt) -- (501.4229pt,727.4453pt);
\draw[pstyle6] (460.4229pt,727.4453pt) -- (501.4229pt,727.4453pt);
\draw[pstyle5] (470.4229pt,723.4453pt) -- (460.4229pt,727.4453pt) -- (470.4229pt,731.4453pt) -- (466.4229pt,727.4453pt) -- cycle;
\node at (466.4229pt,695.9668pt)[below right,color=black]{onExportVCard(...)};
\draw[pstyle6] (454.4229pt,799.9238pt) -- (496.4229pt,799.9238pt);
\draw[pstyle6] (496.4229pt,799.9238pt) -- (496.4229pt,812.9238pt);
\draw[pstyle6] (449.4229pt,812.9238pt) -- (496.4229pt,812.9238pt);
\draw[pstyle5] (459.4229pt,808.9238pt) -- (449.4229pt,812.9238pt) -- (459.4229pt,816.9238pt) -- (455.4229pt,812.9238pt) -- cycle;
\node at (461.4229pt,781.4453pt)[below right,color=black]{salva file};
\draw[pstyle9] (440.4229pt,798.9238pt) -- (458.4229pt,816.9238pt);
\draw[pstyle9] (440.4229pt,816.9238pt) -- (458.4229pt,798.9238pt);
\end{tikzpicture}
}
\end{adjustbox}

\begin{figure}[h]
	\caption{Diagramma sequenza C6 - Esportare rubrica}
	\label{fig:Diagramma sequenza C6 - Esportare rubrica}
\end{figure}

\newpage
\subsubsection{C8 - Salvare rubrica}
Il seguente diagramma di sequenza illustra l'esecuzione del caso d'uso C8:
% generated by Plantuml 1.2024.3       
\definecolor{plantucolor0000}{RGB}{255,255,255}
\definecolor{plantucolor0001}{RGB}{24,24,24}
\definecolor{plantucolor0002}{RGB}{0,0,0}
\definecolor{plantucolor0003}{RGB}{226,226,240}
\definecolor{plantucolor0004}{RGB}{254,255,221}
\definecolor{plantucolor0005}{RGB}{238,238,238}

\begin{adjustbox}{width=.9\paperwidth, center}
	\resizebox{\textwidth}{!}{
\begin{tikzpicture}[yscale=-1
,pstyle0/.style={color=plantucolor0001,fill=white,line width=1.0pt}
,pstyle1/.style={color=black,line width=1.5pt}
,pstyle2/.style={color=plantucolor0001,line width=0.5pt,dash pattern=on 5.0pt off 5.0pt}
,pstyle3/.style={color=plantucolor0001,fill=plantucolor0003,line width=0.5pt}
,pstyle4/.style={color=plantucolor0001,line width=0.5pt}
,pstyle5/.style={color=plantucolor0001,fill=plantucolor0001,line width=1.0pt}
,pstyle6/.style={color=plantucolor0001,line width=1.0pt}
,pstyle7/.style={color=plantucolor0001,fill=plantucolor0004,line width=0.5pt}
]
\draw[pstyle0] (290.6771pt,166.6816pt) rectangle (300.6771pt,496.4746pt);
\draw[pstyle0] (295.6771pt,248.6172pt) rectangle (305.6771pt,393.5742pt);
\draw[pstyle0] (300.6771pt,292.0957pt) rectangle (310.6771pt,320.0957pt);
\draw[pstyle0] (300.6771pt,365.5742pt) rectangle (310.6771pt,393.5742pt);
\draw[pstyle0] (295.6771pt,466.4746pt) rectangle (305.6771pt,496.4746pt);
\draw[pstyle0] (538.6553pt,466.4746pt) rectangle (548.6553pt,496.4746pt);
\draw[pstyle1] (228.0535pt,199.6602pt) rectangle (686.7392pt,523.9531pt);
\draw[pstyle2] (29pt,82.7461pt) -- (29pt,540.9531pt);
\draw[pstyle2] (295.0535pt,82.7461pt) -- (295.0535pt,540.9531pt);
\draw[pstyle2] (543.5799pt,82.7461pt) -- (543.5799pt,540.9531pt);
\node at (5pt,65pt)[below right,color=black]{Utente};
\draw[pstyle3] (29.6pt,13.5pt) ellipse (8pt and 8pt);
\draw[pstyle4] (29.6pt,21.5pt) -- (29.6pt,48.5pt)(16.6pt,29.5pt) -- (42.6pt,29.5pt)(29.6pt,48.5pt) -- (16.6pt,63.5pt)(29.6pt,48.5pt) -- (42.6pt,63.5pt);
\node at (5pt,539.9531pt)[below right,color=black]{Utente};
\draw[pstyle3] (29.6pt,566.1992pt) ellipse (8pt and 8pt);
\draw[pstyle4] (29.6pt,574.1992pt) -- (29.6pt,601.1992pt)(16.6pt,582.1992pt) -- (42.6pt,582.1992pt)(29.6pt,601.1992pt) -- (16.6pt,616.1992pt)(29.6pt,601.1992pt) -- (42.6pt,616.1992pt);
\draw[pstyle3] (238.0535pt,55pt) arc (180:270:5pt) -- (243.0535pt,50pt) -- (348.3006pt,50pt) arc (270:360:5pt) -- (353.3006pt,55pt) -- (353.3006pt,76.7461pt) arc (0:90:5pt) -- (348.3006pt,81.7461pt) -- (243.0535pt,81.7461pt) arc (90:180:5pt) -- (238.0535pt,76.7461pt) -- cycle;
\node at (245.0535pt,57pt)[below right,color=black]{a AddressBook};
\draw[pstyle3] (238.0535pt,544.9531pt) arc (180:270:5pt) -- (243.0535pt,539.9531pt) -- (348.3006pt,539.9531pt) arc (270:360:5pt) -- (353.3006pt,544.9531pt) -- (353.3006pt,566.6992pt) arc (0:90:5pt) -- (348.3006pt,571.6992pt) -- (243.0535pt,571.6992pt) arc (90:180:5pt) -- (238.0535pt,566.6992pt) -- cycle;
\node at (245.0535pt,546.9531pt)[below right,color=black]{a AddressBook};
\draw[pstyle3] (500.5799pt,55pt) arc (180:270:5pt) -- (505.5799pt,50pt) -- (581.7307pt,50pt) arc (270:360:5pt) -- (586.7307pt,55pt) -- (586.7307pt,76.7461pt) arc (0:90:5pt) -- (581.7307pt,81.7461pt) -- (505.5799pt,81.7461pt) arc (90:180:5pt) -- (500.5799pt,76.7461pt) -- cycle;
\node at (507.5799pt,57pt)[below right,color=black]{a Database};
\draw[pstyle3] (500.5799pt,544.9531pt) arc (180:270:5pt) -- (505.5799pt,539.9531pt) -- (581.7307pt,539.9531pt) arc (270:360:5pt) -- (586.7307pt,544.9531pt) -- (586.7307pt,566.6992pt) arc (0:90:5pt) -- (581.7307pt,571.6992pt) -- (505.5799pt,571.6992pt) arc (90:180:5pt) -- (500.5799pt,566.6992pt) -- cycle;
\node at (507.5799pt,546.9531pt)[below right,color=black]{a Database};
\draw[pstyle0] (290.6771pt,166.6816pt) rectangle (300.6771pt,496.4746pt);
\draw[pstyle0] (295.6771pt,248.6172pt) rectangle (305.6771pt,393.5742pt);
\draw[pstyle0] (300.6771pt,292.0957pt) rectangle (310.6771pt,320.0957pt);
\draw[pstyle0] (300.6771pt,365.5742pt) rectangle (310.6771pt,393.5742pt);
\draw[pstyle0] (295.6771pt,466.4746pt) rectangle (305.6771pt,496.4746pt);
\draw[pstyle0] (538.6553pt,466.4746pt) rectangle (548.6553pt,496.4746pt);
\draw[pstyle5] (283.6771pt,111.2246pt) -- (293.6771pt,115.2246pt) -- (283.6771pt,119.2246pt) -- (287.6771pt,115.2246pt) -- cycle;
\draw[pstyle6] (29.6pt,115.2246pt) -- (289.6771pt,115.2246pt);
\node at (36.6pt,96.7461pt)[below right,color=black]{aggiunge/modifica/elimina contatto o tag};
\draw[pstyle6] (295.6771pt,153.6816pt) -- (342.6771pt,153.6816pt);
\draw[pstyle6] (342.6771pt,153.6816pt) -- (342.6771pt,166.6816pt);
\draw[pstyle6] (301.6771pt,166.6816pt) -- (342.6771pt,166.6816pt);
\draw[pstyle5] (311.6771pt,162.6816pt) -- (301.6771pt,166.6816pt) -- (311.6771pt,170.6816pt) -- (307.6771pt,166.6816pt) -- cycle;
\node at (307.6771pt,135.2031pt)[below right,color=black]{metodo di modifica rubrica};
\draw[pstyle7] (478.9801pt,128.2246pt) -- (478.9801pt,187.2246pt) -- (588.9801pt,187.2246pt) -- (588.9801pt,138.2246pt) -- (578.9801pt,128.2246pt) -- (478.9801pt,128.2246pt);
\draw[pstyle7] (578.9801pt,128.2246pt) -- (578.9801pt,138.2246pt) -- (588.9801pt,138.2246pt) -- (578.9801pt,128.2246pt);
\node at (484.9801pt,133.2246pt)[below right,color=black]{ad esempio};
\node at (484.9801pt,149.7031pt)[below right,color=black]{addContact(...)};
\node at (484.9801pt,166.1816pt)[below right,color=black]{removeTag(...)};
\draw[color=black,fill=plantucolor0005,line width=1.5pt] (228.0535pt,199.6602pt) -- (289.8535pt,199.6602pt) -- (289.8535pt,208.1387pt) -- (279.8535pt,218.1387pt) -- (228.0535pt,218.1387pt) -- (228.0535pt,199.6602pt);
\draw[pstyle1] (228.0535pt,199.6602pt) rectangle (686.7392pt,523.9531pt);
\node at (243.0535pt,200.6602pt)[below right,color=black]{\textbf{alt}};
\node at (304.8535pt,201.6602pt)[below right,color=black]{\textbf{[non esiste un DB per il salvataggio]}};
\draw[pstyle6] (300.6771pt,235.6172pt) -- (347.6771pt,235.6172pt);
\draw[pstyle6] (347.6771pt,235.6172pt) -- (347.6771pt,248.6172pt);
\draw[pstyle6] (306.6771pt,248.6172pt) -- (347.6771pt,248.6172pt);
\draw[pstyle5] (316.6771pt,244.6172pt) -- (306.6771pt,248.6172pt) -- (316.6771pt,252.6172pt) -- (312.6771pt,248.6172pt) -- cycle;
\node at (312.6771pt,217.1387pt)[below right,color=black]{saveOBJ()};
\draw[pstyle6] (310.6771pt,284.0957pt) -- (352.6771pt,284.0957pt);
\draw[pstyle6] (352.6771pt,284.0957pt) -- (352.6771pt,297.0957pt);
\draw[pstyle6] (311.6771pt,297.0957pt) -- (352.6771pt,297.0957pt);
\draw[pstyle5] (321.6771pt,293.0957pt) -- (311.6771pt,297.0957pt) -- (321.6771pt,301.0957pt) -- (317.6771pt,297.0957pt) -- cycle;
\node at (317.6771pt,265.6172pt)[below right,color=black]{Salvataggio lista contatti in Data.bin};
\draw[pstyle6] (310.6771pt,357.5742pt) -- (352.6771pt,357.5742pt);
\draw[pstyle6] (352.6771pt,357.5742pt) -- (352.6771pt,370.5742pt);
\draw[pstyle6] (311.6771pt,370.5742pt) -- (352.6771pt,370.5742pt);
\draw[pstyle5] (321.6771pt,366.5742pt) -- (311.6771pt,370.5742pt) -- (321.6771pt,374.5742pt) -- (317.6771pt,370.5742pt) -- cycle;
\node at (317.6771pt,339.0957pt)[below right,color=black]{Salvataggio lista tags in Data.bin};
\draw[color=black,line width=1.0pt,dash pattern=on 2.0pt off 2.0pt] (228.0535pt,409.5742pt) -- (686.7392pt,409.5742pt);
\node at (233.0535pt,409.5742pt)[below right,color=black]{\textbf{[esiste un DB per il salvataggio]}};
\draw[pstyle6] (536.6553pt,466.4746pt) -- (526.6553pt,462.4746pt);
\draw[pstyle6] (305.6771pt,466.4746pt) -- (537.6553pt,466.4746pt);
\node at (312.6771pt,447.9961pt)[below right,color=black]{metodo aggiornamento};
\draw[pstyle7] (553pt,429.5176pt) -- (553pt,488.5176pt) -- (681pt,488.5176pt) -- (681pt,439.5176pt) -- (671pt,429.5176pt) -- (553pt,429.5176pt);
\draw[pstyle7] (671pt,429.5176pt) -- (671pt,439.5176pt) -- (681pt,439.5176pt) -- (671pt,429.5176pt);
\node at (559pt,434.5176pt)[below right,color=black]{ad esempio};
\node at (559pt,450.9961pt)[below right,color=black]{upsertContact(...)};
\node at (559pt,467.4746pt)[below right,color=black]{removeTag(...)};
\end{tikzpicture}
}
\end{adjustbox}

\begin{figure}[h]
	\caption{Diagramma sequenza C8 - Salvare rubrica}
	\label{fig:Diagramma sequenza C8 - Salvare rubrica}
\end{figure}

\newpage
\subsubsection{Aggiunta DB}
Il seguente diagramma di sequenza illustra il flusso di operazioni attraverso cui l'utente aggiunge un link a un Database, consentendo il salvataggio dei dati nel DB anziché in locale.:
% generated by Plantuml 1.2024.3       
\definecolor{plantucolor0000}{RGB}{255,255,255}
\definecolor{plantucolor0001}{RGB}{24,24,24}
\definecolor{plantucolor0002}{RGB}{0,0,0}
\definecolor{plantucolor0003}{RGB}{226,226,240}
\definecolor{plantucolor0004}{RGB}{254,255,221}
\definecolor{plantucolor0005}{RGB}{238,238,238}
\definecolor{plantucolor0006}{RGB}{168,0,54}

\begin{adjustbox}{width=.9\paperwidth, center}
	\resizebox{\textwidth}{!}{
\begin{tikzpicture}[yscale=-1
,pstyle0/.style={color=plantucolor0001,fill=white,line width=1.0pt}
,pstyle1/.style={color=black,line width=1.5pt}
,pstyle2/.style={color=plantucolor0001,line width=0.5pt,dash pattern=on 5.0pt off 5.0pt}
,pstyle3/.style={color=plantucolor0001,fill=plantucolor0003,line width=0.5pt}
,pstyle4/.style={color=plantucolor0001,line width=0.5pt}
,pstyle5/.style={color=plantucolor0001,fill=plantucolor0001,line width=1.0pt}
,pstyle6/.style={color=plantucolor0001,line width=1.0pt}
,pstyle7/.style={color=plantucolor0001,fill=plantucolor0004,line width=0.5pt}
,pstyle10/.style={color=plantucolor0006,line width=2.0pt}
]
\draw[pstyle0] (233.7801pt,115.2246pt) rectangle (243.7801pt,145.7031pt);
\draw[pstyle0] (476.8468pt,226.9277pt) rectangle (486.8468pt,298.8848pt);
\draw[pstyle0] (481.8468pt,265.4063pt) rectangle (491.8468pt,298.8848pt);
\draw[pstyle0] (476.8468pt,393.3203pt) rectangle (486.8468pt,779.9375pt);
\draw[pstyle0] (634.2703pt,428.7988pt) rectangle (644.2703pt,458.7988pt);
\draw[pstyle0] (634.2703pt,489.2773pt) rectangle (644.2703pt,519.2773pt);
\draw[pstyle0] (634.2703pt,549.7559pt) rectangle (644.2703pt,779.9375pt);
\draw[pstyle0] (639.2703pt,751.9375pt) rectangle (649.2703pt,779.9375pt);
\draw[pstyle0] (739.2586pt,265.4063pt) rectangle (749.2586pt,298.8848pt);
\draw[pstyle0] (829.6987pt,622.9805pt) rectangle (839.6987pt,652.9805pt);
\draw[pstyle0] (829.6987pt,683.459pt) rectangle (839.6987pt,713.459pt);
\draw[pstyle1] (10pt,313.8848pt) rectangle (887.7741pt,794.9375pt);
\draw[pstyle2] (44pt,82.7461pt) -- (44pt,811.9375pt);
\draw[pstyle2] (238.1461pt,82.7461pt) -- (238.1461pt,811.9375pt);
\draw[pstyle2] (481.0468pt,154.5977pt) -- (481.0468pt,811.9375pt);
\draw[pstyle2] (638.6468pt,82.7461pt) -- (638.6468pt,811.9375pt);
\draw[pstyle2] (743.8939pt,82.7461pt) -- (743.8939pt,811.9375pt);
\draw[pstyle2] (834.6233pt,589.1289pt) -- (834.6233pt,811.9375pt);
\node at (20pt,65pt)[below right,color=black]{Utente};
\draw[pstyle3] (44.6pt,13.5pt) ellipse (8pt and 8pt);
\draw[pstyle4] (44.6pt,21.5pt) -- (44.6pt,48.5pt)(31.6pt,29.5pt) -- (57.6pt,29.5pt)(44.6pt,48.5pt) -- (31.6pt,63.5pt)(44.6pt,48.5pt) -- (57.6pt,63.5pt);
\node at (20pt,810.9375pt)[below right,color=black]{Utente};
\draw[pstyle3] (44.6pt,837.1836pt) ellipse (8pt and 8pt);
\draw[pstyle4] (44.6pt,845.1836pt) -- (44.6pt,872.1836pt)(31.6pt,853.1836pt) -- (57.6pt,853.1836pt)(44.6pt,872.1836pt) -- (31.6pt,887.1836pt)(44.6pt,872.1836pt) -- (57.6pt,887.1836pt);
\draw[pstyle3] (176.1461pt,55pt) arc (180:270:5pt) -- (181.1461pt,50pt) -- (296.4142pt,50pt) arc (270:360:5pt) -- (301.4142pt,55pt) -- (301.4142pt,76.7461pt) arc (0:90:5pt) -- (296.4142pt,81.7461pt) -- (181.1461pt,81.7461pt) arc (90:180:5pt) -- (176.1461pt,76.7461pt) -- cycle;
\node at (183.1461pt,57pt)[below right,color=black]{a MainController};
\draw[pstyle3] (176.1461pt,815.9375pt) arc (180:270:5pt) -- (181.1461pt,810.9375pt) -- (296.4142pt,810.9375pt) arc (270:360:5pt) -- (301.4142pt,815.9375pt) -- (301.4142pt,837.6836pt) arc (0:90:5pt) -- (296.4142pt,842.6836pt) -- (181.1461pt,842.6836pt) arc (90:180:5pt) -- (176.1461pt,837.6836pt) -- cycle;
\node at (183.1461pt,817.9375pt)[below right,color=black]{a MainController};
\draw[pstyle3] (392.0468pt,815.9375pt) arc (180:270:5pt) -- (397.0468pt,810.9375pt) -- (566.6468pt,810.9375pt) arc (270:360:5pt) -- (571.6468pt,815.9375pt) -- (571.6468pt,837.6836pt) arc (0:90:5pt) -- (566.6468pt,842.6836pt) -- (397.0468pt,842.6836pt) arc (90:180:5pt) -- (392.0468pt,837.6836pt) -- cycle;
\node at (399.0468pt,817.9375pt)[below right,color=black]{a ConfigPopupController};
\draw[pstyle3] (581.6468pt,55pt) arc (180:270:5pt) -- (586.6468pt,50pt) -- (691.8939pt,50pt) arc (270:360:5pt) -- (696.8939pt,55pt) -- (696.8939pt,76.7461pt) arc (0:90:5pt) -- (691.8939pt,81.7461pt) -- (586.6468pt,81.7461pt) arc (90:180:5pt) -- (581.6468pt,76.7461pt) -- cycle;
\node at (588.6468pt,57pt)[below right,color=black]{a AddressBook};
\draw[pstyle3] (581.6468pt,815.9375pt) arc (180:270:5pt) -- (586.6468pt,810.9375pt) -- (691.8939pt,810.9375pt) arc (270:360:5pt) -- (696.8939pt,815.9375pt) -- (696.8939pt,837.6836pt) arc (0:90:5pt) -- (691.8939pt,842.6836pt) -- (586.6468pt,842.6836pt) arc (90:180:5pt) -- (581.6468pt,837.6836pt) -- cycle;
\node at (588.6468pt,817.9375pt)[below right,color=black]{a AddressBook};
\draw[pstyle3] (706.8939pt,55pt) arc (180:270:5pt) -- (711.8939pt,50pt) -- (776.6233pt,50pt) arc (270:360:5pt) -- (781.6233pt,55pt) -- (781.6233pt,76.7461pt) arc (0:90:5pt) -- (776.6233pt,81.7461pt) -- (711.8939pt,81.7461pt) arc (90:180:5pt) -- (706.8939pt,76.7461pt) -- cycle;
\node at (713.8939pt,57pt)[below right,color=black]{Database};
\draw[pstyle3] (706.8939pt,815.9375pt) arc (180:270:5pt) -- (711.8939pt,810.9375pt) -- (776.6233pt,810.9375pt) arc (270:360:5pt) -- (781.6233pt,815.9375pt) -- (781.6233pt,837.6836pt) arc (0:90:5pt) -- (776.6233pt,842.6836pt) -- (711.8939pt,842.6836pt) arc (90:180:5pt) -- (706.8939pt,837.6836pt) -- cycle;
\node at (713.8939pt,817.9375pt)[below right,color=black]{Database};
\draw[pstyle3] (791.6233pt,815.9375pt) arc (180:270:5pt) -- (796.6233pt,810.9375pt) -- (872.7741pt,810.9375pt) arc (270:360:5pt) -- (877.7741pt,815.9375pt) -- (877.7741pt,837.6836pt) arc (0:90:5pt) -- (872.7741pt,842.6836pt) -- (796.6233pt,842.6836pt) arc (90:180:5pt) -- (791.6233pt,837.6836pt) -- cycle;
\node at (798.6233pt,817.9375pt)[below right,color=black]{a Database};
\draw[pstyle0] (233.7801pt,115.2246pt) rectangle (243.7801pt,145.7031pt);
\draw[pstyle0] (476.8468pt,226.9277pt) rectangle (486.8468pt,298.8848pt);
\draw[pstyle0] (481.8468pt,265.4063pt) rectangle (491.8468pt,298.8848pt);
\draw[pstyle0] (476.8468pt,393.3203pt) rectangle (486.8468pt,779.9375pt);
\draw[pstyle0] (634.2703pt,428.7988pt) rectangle (644.2703pt,458.7988pt);
\draw[pstyle0] (634.2703pt,489.2773pt) rectangle (644.2703pt,519.2773pt);
\draw[pstyle0] (634.2703pt,549.7559pt) rectangle (644.2703pt,779.9375pt);
\draw[pstyle0] (639.2703pt,751.9375pt) rectangle (649.2703pt,779.9375pt);
\draw[pstyle0] (739.2586pt,265.4063pt) rectangle (749.2586pt,298.8848pt);
\draw[pstyle0] (829.6987pt,622.9805pt) rectangle (839.6987pt,652.9805pt);
\draw[pstyle0] (829.6987pt,683.459pt) rectangle (839.6987pt,713.459pt);
\draw[pstyle5] (221.7801pt,111.2246pt) -- (231.7801pt,115.2246pt) -- (221.7801pt,119.2246pt) -- (225.7801pt,115.2246pt) -- cycle;
\draw[pstyle6] (44.6pt,115.2246pt) -- (227.7801pt,115.2246pt);
\node at (51.6pt,96.7461pt)[below right,color=black]{nel menù file clicca "Configurazione"};
\draw[pstyle5] (380.0468pt,141.7031pt) -- (390.0468pt,145.7031pt) -- (380.0468pt,149.7031pt) -- (384.0468pt,145.7031pt) -- cycle;
\draw[pstyle6] (238.7801pt,145.7031pt) -- (386.0468pt,145.7031pt);
\node at (245.7801pt,127.2246pt)[below right,color=black]{showConfigPopup()};
\draw[pstyle3] (392.0468pt,128.2246pt) arc (180:270:5pt) -- (397.0468pt,123.2246pt) -- (566.6468pt,123.2246pt) arc (270:360:5pt) -- (571.6468pt,128.2246pt) -- (571.6468pt,149.9707pt) arc (0:90:5pt) -- (566.6468pt,154.9707pt) -- (397.0468pt,154.9707pt) arc (90:180:5pt) -- (392.0468pt,149.9707pt) -- cycle;
\node at (399.0468pt,130.2246pt)[below right,color=black]{a ConfigPopupController};
\draw[pstyle5] (469.8468pt,184.4492pt) -- (479.8468pt,188.4492pt) -- (469.8468pt,192.4492pt) -- (473.8468pt,188.4492pt) -- cycle;
\draw[pstyle6] (44.6pt,188.4492pt) -- (475.8468pt,188.4492pt);
\node at (51.6pt,169.9707pt)[below right,color=black]{inserisce URL del database e clicca "Verifica"};
\draw[pstyle6] (481.8468pt,213.9277pt) -- (528.8468pt,213.9277pt);
\draw[pstyle6] (528.8468pt,213.9277pt) -- (528.8468pt,226.9277pt);
\draw[pstyle6] (487.8468pt,226.9277pt) -- (528.8468pt,226.9277pt);
\draw[pstyle5] (497.8468pt,222.9277pt) -- (487.8468pt,226.9277pt) -- (497.8468pt,230.9277pt) -- (493.8468pt,226.9277pt) -- cycle;
\node at (493.8468pt,195.4492pt)[below right,color=black]{onVerify()};
\draw[pstyle5] (727.2586pt,261.4063pt) -- (737.2586pt,265.4063pt) -- (727.2586pt,269.4063pt) -- (731.2586pt,265.4063pt) -- cycle;
\draw[pstyle6] (491.8468pt,265.4063pt) -- (733.2586pt,265.4063pt);
\node at (498.8468pt,246.9277pt)[below right,color=black]{verifyDBUrl(String url)};
\draw[pstyle7] (754pt,244.9277pt) -- (754pt,270.9277pt) -- (926pt,270.9277pt) -- (926pt,254.9277pt) -- (916pt,244.9277pt) -- (754pt,244.9277pt);
\draw[pstyle7] (916pt,244.9277pt) -- (916pt,254.9277pt) -- (926pt,254.9277pt) -- (916pt,244.9277pt);
\node at (760pt,249.9277pt)[below right,color=black]{Questo metodo è statico.};
\draw[pstyle5] (492.8468pt,294.8848pt) -- (482.8468pt,298.8848pt) -- (492.8468pt,302.8848pt) -- (488.8468pt,298.8848pt) -- cycle;
\draw[color=plantucolor0001,line width=1.0pt,dash pattern=on 2.0pt off 2.0pt] (486.8468pt,298.8848pt) -- (743.2586pt,298.8848pt);
\node at (498.8468pt,280.4063pt)[below right,color=black]{boolean};
\draw[color=black,fill=plantucolor0005,line width=1.5pt] (10pt,313.8848pt) -- (71.8pt,313.8848pt) -- (71.8pt,322.3633pt) -- (61.8pt,332.3633pt) -- (10pt,332.3633pt) -- (10pt,313.8848pt);
\draw[pstyle1] (10pt,313.8848pt) rectangle (887.7741pt,794.9375pt);
\node at (25pt,314.8848pt)[below right,color=black]{\textbf{alt}};
\node at (86.8pt,315.8848pt)[below right,color=black]{\textbf{["L'url è valido"]}};
\draw[pstyle5] (469.8468pt,350.8418pt) -- (479.8468pt,354.8418pt) -- (469.8468pt,358.8418pt) -- (473.8468pt,354.8418pt) -- cycle;
\draw[pstyle6] (44.6pt,354.8418pt) -- (475.8468pt,354.8418pt);
\node at (51.6pt,336.3633pt)[below right,color=black]{clicca "Conferma"};
\draw[pstyle6] (481.8468pt,380.3203pt) -- (528.8468pt,380.3203pt);
\draw[pstyle6] (528.8468pt,380.3203pt) -- (528.8468pt,393.3203pt);
\draw[pstyle6] (487.8468pt,393.3203pt) -- (528.8468pt,393.3203pt);
\draw[pstyle5] (497.8468pt,389.3203pt) -- (487.8468pt,393.3203pt) -- (497.8468pt,397.3203pt) -- (493.8468pt,393.3203pt) -- cycle;
\node at (493.8468pt,361.8418pt)[below right,color=black]{onConfirm()};
\draw[pstyle5] (622.2703pt,424.7988pt) -- (632.2703pt,428.7988pt) -- (622.2703pt,432.7988pt) -- (626.2703pt,428.7988pt) -- cycle;
\draw[pstyle6] (486.8468pt,428.7988pt) -- (628.2703pt,428.7988pt);
\node at (493.8468pt,410.3203pt)[below right,color=black]{setDBUrl(String url)};
\draw[pstyle5] (622.2703pt,485.2773pt) -- (632.2703pt,489.2773pt) -- (622.2703pt,493.2773pt) -- (626.2703pt,489.2773pt) -- cycle;
\draw[pstyle6] (486.8468pt,489.2773pt) -- (628.2703pt,489.2773pt);
\node at (493.8468pt,470.7988pt)[below right,color=black]{saveConfig()};
\draw[pstyle5] (622.2703pt,545.7559pt) -- (632.2703pt,549.7559pt) -- (622.2703pt,553.7559pt) -- (626.2703pt,549.7559pt) -- cycle;
\draw[pstyle6] (486.8468pt,549.7559pt) -- (628.2703pt,549.7559pt);
\node at (493.8468pt,531.2773pt)[below right,color=black]{initDB()};
\draw[pstyle5] (779.6233pt,576.2344pt) -- (789.6233pt,580.2344pt) -- (779.6233pt,584.2344pt) -- (783.6233pt,580.2344pt) -- cycle;
\draw[pstyle6] (644.2703pt,580.2344pt) -- (785.6233pt,580.2344pt);
\node at (651.2703pt,561.7559pt)[below right,color=black]{Database(String url)};
\draw[pstyle3] (791.6233pt,562.7559pt) arc (180:270:5pt) -- (796.6233pt,557.7559pt) -- (872.7741pt,557.7559pt) arc (270:360:5pt) -- (877.7741pt,562.7559pt) -- (877.7741pt,584.502pt) arc (0:90:5pt) -- (872.7741pt,589.502pt) -- (796.6233pt,589.502pt) arc (90:180:5pt) -- (791.6233pt,584.502pt) -- cycle;
\node at (798.6233pt,564.7559pt)[below right,color=black]{a Database};
\draw[pstyle6] (827.6987pt,622.9805pt) -- (817.6987pt,618.9805pt);
\draw[pstyle6] (644.2703pt,622.9805pt) -- (828.6987pt,622.9805pt);
\node at (651.2703pt,604.502pt)[below right,color=black]{insertManyContacts(...)};
\draw[pstyle6] (827.6987pt,683.459pt) -- (817.6987pt,679.459pt);
\draw[pstyle6] (644.2703pt,683.459pt) -- (828.6987pt,683.459pt);
\node at (651.2703pt,664.9805pt)[below right,color=black]{insertManyTags(...)};
\draw[pstyle6] (649.2703pt,743.9375pt) -- (691.2703pt,743.9375pt);
\draw[pstyle6] (691.2703pt,743.9375pt) -- (691.2703pt,756.9375pt);
\draw[pstyle6] (650.2703pt,756.9375pt) -- (691.2703pt,756.9375pt);
\draw[pstyle5] (660.2703pt,752.9375pt) -- (650.2703pt,756.9375pt) -- (660.2703pt,760.9375pt) -- (656.2703pt,756.9375pt) -- cycle;
\node at (656.2703pt,725.459pt)[below right,color=black]{removeOBJ()};
\draw[pstyle7] (498pt,729.959pt) -- (498pt,755.959pt) -- (628pt,755.959pt) -- (628pt,739.959pt) -- (618pt,729.959pt) -- (498pt,729.959pt);
\draw[pstyle7] (618pt,729.959pt) -- (618pt,739.959pt) -- (628pt,739.959pt) -- (618pt,729.959pt);
\node at (504pt,734.959pt)[below right,color=black]{Rimuove Data.bin};
\draw[pstyle10] (472.8468pt,777.9375pt) -- (490.8468pt,795.9375pt);
\draw[pstyle10] (472.8468pt,795.9375pt) -- (490.8468pt,777.9375pt);
\end{tikzpicture}
}
\end{adjustbox}

\begin{figure}[h]
	\caption{Diagramma sequenza Aggiunta DB}
	\label{fig:Diagramma sequenza Aggiunta DB}
\end{figure}

\newpage
\subsubsection{Inizializzazione rubrica}
Il seguente diagramma di sequenza illustra il flusso di operazioni eseguito quando l'utente apre l'applicazione, durante la fase di inizializzazione, permettendogli di recuperare tutti i contatti della sessione precedente, attraverso il DB o il file \texttt{Data.bin}:
% generated by Plantuml 1.2024.3       
\definecolor{plantucolor0000}{RGB}{255,255,255}
\definecolor{plantucolor0001}{RGB}{24,24,24}
\definecolor{plantucolor0002}{RGB}{0,0,0}
\definecolor{plantucolor0003}{RGB}{226,226,240}
\definecolor{plantucolor0004}{RGB}{238,238,238}

\begin{adjustbox}{width=.82\paperwidth, center}
	\resizebox{\textwidth}{!}{
\begin{tikzpicture}[yscale=-1
,pstyle0/.style={color=plantucolor0001,fill=white,line width=1.0pt}
,pstyle1/.style={color=black,line width=1.5pt}
,pstyle2/.style={color=plantucolor0001,line width=0.5pt,dash pattern=on 5.0pt off 5.0pt}
,pstyle3/.style={color=plantucolor0001,fill=plantucolor0003,line width=0.5pt}
,pstyle4/.style={color=plantucolor0001,line width=0.5pt}
,pstyle5/.style={color=plantucolor0001,fill=plantucolor0001,line width=1.0pt}
,pstyle6/.style={color=plantucolor0001,line width=1.0pt}
,pstyle9/.style={color=plantucolor0001,line width=1.0pt,dash pattern=on 2.0pt off 2.0pt}
]
\draw[pstyle0] (209.7412pt,165.9707pt) rectangle (219.7412pt,765.1484pt);
\draw[pstyle0] (214.7412pt,211.4492pt) rectangle (224.7412pt,734.6699pt);
\draw[pstyle0] (399.9201pt,252.1953pt) rectangle (409.9201pt,734.6699pt);
\draw[pstyle0] (404.9201pt,295.6738pt) rectangle (414.9201pt,323.6738pt);
\draw[pstyle0] (404.9201pt,394.6309pt) rectangle (414.9201pt,422.6309pt);
\draw[pstyle0] (404.9201pt,484.0527pt) rectangle (414.9201pt,519.5313pt);
\draw[pstyle0] (404.9201pt,570.2773pt) rectangle (414.9201pt,697.1914pt);
\draw[pstyle0] (409.9201pt,605.7559pt) rectangle (419.9201pt,636.2344pt);
\draw[pstyle0] (409.9201pt,666.7129pt) rectangle (419.9201pt,697.1914pt);
\draw[pstyle0] (599.1646pt,605.7559pt) rectangle (609.1646pt,636.2344pt);
\draw[pstyle0] (599.1646pt,666.7129pt) rectangle (609.1646pt,697.1914pt);
\draw[pstyle1] (337.2966pt,345.6738pt) rectangle (657.24pt,705.1914pt);
\draw[pstyle2] (29pt,82.7461pt) -- (29pt,783.1484pt);
\draw[pstyle2] (214.1071pt,124.1191pt) -- (214.1071pt,783.1484pt);
\draw[pstyle2] (404.2966pt,210.3438pt) -- (404.2966pt,783.1484pt);
\draw[pstyle2] (604.0892pt,528.4258pt) -- (604.0892pt,783.1484pt);
\node at (5pt,65pt)[below right,color=black]{Utente};
\draw[pstyle3] (29.6pt,13.5pt) ellipse (8pt and 8pt);
\draw[pstyle4] (29.6pt,21.5pt) -- (29.6pt,48.5pt)(16.6pt,29.5pt) -- (42.6pt,29.5pt)(29.6pt,48.5pt) -- (16.6pt,63.5pt)(29.6pt,48.5pt) -- (42.6pt,63.5pt);
\node at (5pt,782.1484pt)[below right,color=black]{Utente};
\draw[pstyle3] (29.6pt,808.3945pt) ellipse (8pt and 8pt);
\draw[pstyle4] (29.6pt,816.3945pt) -- (29.6pt,843.3945pt)(16.6pt,824.3945pt) -- (42.6pt,824.3945pt)(29.6pt,843.3945pt) -- (16.6pt,858.3945pt)(29.6pt,843.3945pt) -- (42.6pt,858.3945pt);
\draw[pstyle3] (152.1071pt,787.1484pt) arc (180:270:5pt) -- (157.1071pt,782.1484pt) -- (272.3752pt,782.1484pt) arc (270:360:5pt) -- (277.3752pt,787.1484pt) -- (277.3752pt,808.8945pt) arc (0:90:5pt) -- (272.3752pt,813.8945pt) -- (157.1071pt,813.8945pt) arc (90:180:5pt) -- (152.1071pt,808.8945pt) -- cycle;
\node at (159.1071pt,789.1484pt)[below right,color=black]{a MainController};
\draw[pstyle3] (347.2966pt,787.1484pt) arc (180:270:5pt) -- (352.2966pt,782.1484pt) -- (457.5437pt,782.1484pt) arc (270:360:5pt) -- (462.5437pt,787.1484pt) -- (462.5437pt,808.8945pt) arc (0:90:5pt) -- (457.5437pt,813.8945pt) -- (352.2966pt,813.8945pt) arc (90:180:5pt) -- (347.2966pt,808.8945pt) -- cycle;
\node at (354.2966pt,789.1484pt)[below right,color=black]{a AddressBook};
\draw[pstyle3] (561.0892pt,787.1484pt) arc (180:270:5pt) -- (566.0892pt,782.1484pt) -- (642.24pt,782.1484pt) arc (270:360:5pt) -- (647.24pt,787.1484pt) -- (647.24pt,808.8945pt) arc (0:90:5pt) -- (642.24pt,813.8945pt) -- (566.0892pt,813.8945pt) arc (90:180:5pt) -- (561.0892pt,808.8945pt) -- cycle;
\node at (568.0892pt,789.1484pt)[below right,color=black]{a Database};
\draw[pstyle0] (209.7412pt,165.9707pt) rectangle (219.7412pt,765.1484pt);
\draw[pstyle0] (214.7412pt,211.4492pt) rectangle (224.7412pt,734.6699pt);
\draw[pstyle0] (399.9201pt,252.1953pt) rectangle (409.9201pt,734.6699pt);
\draw[pstyle0] (404.9201pt,295.6738pt) rectangle (414.9201pt,323.6738pt);
\draw[pstyle0] (404.9201pt,394.6309pt) rectangle (414.9201pt,422.6309pt);
\draw[pstyle0] (404.9201pt,484.0527pt) rectangle (414.9201pt,519.5313pt);
\draw[pstyle0] (404.9201pt,570.2773pt) rectangle (414.9201pt,697.1914pt);
\draw[pstyle0] (409.9201pt,605.7559pt) rectangle (419.9201pt,636.2344pt);
\draw[pstyle0] (409.9201pt,666.7129pt) rectangle (419.9201pt,697.1914pt);
\draw[pstyle0] (599.1646pt,605.7559pt) rectangle (609.1646pt,636.2344pt);
\draw[pstyle0] (599.1646pt,666.7129pt) rectangle (609.1646pt,697.1914pt);
\draw[pstyle5] (140.1071pt,111.2246pt) -- (150.1071pt,115.2246pt) -- (140.1071pt,119.2246pt) -- (144.1071pt,115.2246pt) -- cycle;
\draw[pstyle6] (29.6pt,115.2246pt) -- (146.1071pt,115.2246pt);
\node at (36.6pt,96.7461pt)[below right,color=black]{apre rubrica};
\draw[pstyle3] (152.1071pt,97.7461pt) arc (180:270:5pt) -- (157.1071pt,92.7461pt) -- (272.3752pt,92.7461pt) arc (270:360:5pt) -- (277.3752pt,97.7461pt) -- (277.3752pt,119.4922pt) arc (0:90:5pt) -- (272.3752pt,124.4922pt) -- (157.1071pt,124.4922pt) arc (90:180:5pt) -- (152.1071pt,119.4922pt) -- cycle;
\node at (159.1071pt,99.7461pt)[below right,color=black]{a MainController};
\draw[pstyle6] (214.7412pt,152.9707pt) -- (261.7412pt,152.9707pt);
\draw[pstyle6] (261.7412pt,152.9707pt) -- (261.7412pt,165.9707pt);
\draw[pstyle6] (220.7412pt,165.9707pt) -- (261.7412pt,165.9707pt);
\draw[pstyle5] (230.7412pt,161.9707pt) -- (220.7412pt,165.9707pt) -- (230.7412pt,169.9707pt) -- (226.7412pt,165.9707pt) -- cycle;
\node at (226.7412pt,134.4922pt)[below right,color=black]{initialize(...)};
\draw[pstyle5] (335.2966pt,197.4492pt) -- (345.2966pt,201.4492pt) -- (335.2966pt,205.4492pt) -- (339.2966pt,201.4492pt) -- cycle;
\draw[pstyle6] (219.7412pt,201.4492pt) -- (341.2966pt,201.4492pt);
\node at (226.7412pt,182.9707pt)[below right,color=black]{getInstance()};
\draw[pstyle3] (347.2966pt,183.9707pt) arc (180:270:5pt) -- (352.2966pt,178.9707pt) -- (457.5437pt,178.9707pt) arc (270:360:5pt) -- (462.5437pt,183.9707pt) -- (462.5437pt,205.7168pt) arc (0:90:5pt) -- (457.5437pt,210.7168pt) -- (352.2966pt,210.7168pt) arc (90:180:5pt) -- (347.2966pt,205.7168pt) -- cycle;
\node at (354.2966pt,185.9707pt)[below right,color=black]{a AddressBook};
\draw[pstyle6] (404.9201pt,239.1953pt) -- (451.9201pt,239.1953pt);
\draw[pstyle6] (451.9201pt,239.1953pt) -- (451.9201pt,252.1953pt);
\draw[pstyle6] (410.9201pt,252.1953pt) -- (451.9201pt,252.1953pt);
\draw[pstyle5] (420.9201pt,248.1953pt) -- (410.9201pt,252.1953pt) -- (420.9201pt,256.1953pt) -- (416.9201pt,252.1953pt) -- cycle;
\node at (416.9201pt,220.7168pt)[below right,color=black]{AddressBook()};
\draw[pstyle6] (414.9201pt,287.6738pt) -- (456.9201pt,287.6738pt);
\draw[pstyle6] (456.9201pt,287.6738pt) -- (456.9201pt,300.6738pt);
\draw[pstyle6] (415.9201pt,300.6738pt) -- (456.9201pt,300.6738pt);
\draw[pstyle5] (425.9201pt,296.6738pt) -- (415.9201pt,300.6738pt) -- (425.9201pt,304.6738pt) -- (421.9201pt,300.6738pt) -- cycle;
\node at (421.9201pt,269.1953pt)[below right,color=black]{loadConfig()};
\draw[color=black,fill=plantucolor0004,line width=1.5pt] (337.2966pt,345.6738pt) -- (399.0966pt,345.6738pt) -- (399.0966pt,354.1523pt) -- (389.0966pt,364.1523pt) -- (337.2966pt,364.1523pt) -- (337.2966pt,345.6738pt);
\draw[pstyle1] (337.2966pt,345.6738pt) rectangle (657.24pt,705.1914pt);
\node at (352.2966pt,346.6738pt)[below right,color=black]{\textbf{alt}};
\node at (414.0966pt,347.6738pt)[below right,color=black]{\textbf{[utente non ha specificato il link per DB]}};
\draw[pstyle6] (414.9201pt,386.6309pt) -- (456.9201pt,386.6309pt);
\draw[pstyle6] (456.9201pt,386.6309pt) -- (456.9201pt,399.6309pt);
\draw[pstyle6] (415.9201pt,399.6309pt) -- (456.9201pt,399.6309pt);
\draw[pstyle5] (425.9201pt,395.6309pt) -- (415.9201pt,399.6309pt) -- (425.9201pt,403.6309pt) -- (421.9201pt,399.6309pt) -- cycle;
\node at (421.9201pt,368.1523pt)[below right,color=black]{loadOBJ()};
\draw[color=black,line width=1.0pt,dash pattern=on 2.0pt off 2.0pt] (337.2966pt,438.6309pt) -- (657.24pt,438.6309pt);
\node at (342.2966pt,438.6309pt)[below right,color=black]{\textbf{[utente ha specificato il link per DB]}};
\draw[pstyle6] (409.9201pt,471.0527pt) -- (456.9201pt,471.0527pt);
\draw[pstyle6] (456.9201pt,471.0527pt) -- (456.9201pt,484.0527pt);
\draw[pstyle6] (415.9201pt,484.0527pt) -- (456.9201pt,484.0527pt);
\draw[pstyle5] (425.9201pt,480.0527pt) -- (415.9201pt,484.0527pt) -- (425.9201pt,488.0527pt) -- (421.9201pt,484.0527pt) -- cycle;
\node at (421.9201pt,452.5742pt)[below right,color=black]{initDB()};
\draw[pstyle5] (549.0892pt,515.5313pt) -- (559.0892pt,519.5313pt) -- (549.0892pt,523.5313pt) -- (553.0892pt,519.5313pt) -- cycle;
\draw[pstyle6] (409.9201pt,519.5313pt) -- (555.0892pt,519.5313pt);
\node at (416.9201pt,501.0527pt)[below right,color=black]{Database(String url)};
\draw[pstyle3] (561.0892pt,502.0527pt) arc (180:270:5pt) -- (566.0892pt,497.0527pt) -- (642.24pt,497.0527pt) arc (270:360:5pt) -- (647.24pt,502.0527pt) -- (647.24pt,523.7988pt) arc (0:90:5pt) -- (642.24pt,528.7988pt) -- (566.0892pt,528.7988pt) arc (90:180:5pt) -- (561.0892pt,523.7988pt) -- cycle;
\node at (568.0892pt,504.0527pt)[below right,color=black]{a Database};
\draw[pstyle6] (409.9201pt,557.2773pt) -- (456.9201pt,557.2773pt);
\draw[pstyle6] (456.9201pt,557.2773pt) -- (456.9201pt,570.2773pt);
\draw[pstyle6] (415.9201pt,570.2773pt) -- (456.9201pt,570.2773pt);
\draw[pstyle5] (425.9201pt,566.2773pt) -- (415.9201pt,570.2773pt) -- (425.9201pt,574.2773pt) -- (421.9201pt,570.2773pt) -- cycle;
\node at (421.9201pt,538.7988pt)[below right,color=black]{loadFromDB()};
\draw[pstyle5] (587.1646pt,601.7559pt) -- (597.1646pt,605.7559pt) -- (587.1646pt,609.7559pt) -- (591.1646pt,605.7559pt) -- cycle;
\draw[pstyle6] (419.9201pt,605.7559pt) -- (593.1646pt,605.7559pt);
\node at (426.9201pt,587.2773pt)[below right,color=black]{getAllContacts()};
\draw[pstyle5] (425.9201pt,632.2344pt) -- (415.9201pt,636.2344pt) -- (425.9201pt,640.2344pt) -- (421.9201pt,636.2344pt) -- cycle;
\draw[pstyle9] (419.9201pt,636.2344pt) -- (603.1646pt,636.2344pt);
\node at (431.9201pt,617.7559pt)[below right,color=black]{restituisce contatti da DB};
\draw[pstyle5] (587.1646pt,662.7129pt) -- (597.1646pt,666.7129pt) -- (587.1646pt,670.7129pt) -- (591.1646pt,666.7129pt) -- cycle;
\draw[pstyle6] (419.9201pt,666.7129pt) -- (593.1646pt,666.7129pt);
\node at (426.9201pt,648.2344pt)[below right,color=black]{getAllTags()};
\draw[pstyle5] (420.9201pt,693.1914pt) -- (410.9201pt,697.1914pt) -- (420.9201pt,701.1914pt) -- (416.9201pt,697.1914pt) -- cycle;
\draw[pstyle9] (414.9201pt,697.1914pt) -- (603.1646pt,697.1914pt);
\node at (426.9201pt,678.7129pt)[below right,color=black]{restituisce tags da DB};
\draw[pstyle5] (230.7412pt,730.6699pt) -- (220.7412pt,734.6699pt) -- (230.7412pt,738.6699pt) -- (226.7412pt,734.6699pt) -- cycle;
\draw[pstyle9] (224.7412pt,734.6699pt) -- (403.9201pt,734.6699pt);
\node at (236.7412pt,716.1914pt)[below right,color=black]{lista dei contatti e dei tag};
\draw[pstyle5] (40.6pt,761.1484pt) -- (30.6pt,765.1484pt) -- (40.6pt,769.1484pt) -- (36.6pt,765.1484pt) -- cycle;
\draw[pstyle6] (34.6pt,765.1484pt) -- (213.7412pt,765.1484pt);
\node at (46.6pt,746.6699pt)[below right,color=black]{mostra la lista dei contatti};
\end{tikzpicture}
}
\end{adjustbox}

\begin{figure}[h]
	\caption{Diagramma sequenza Inizializzazione rubrica}
	\label{fig:Diagramma sequenza Inizializzazione rubrica}
\end{figure}


\newpage
\subsection{Principi di buona progettazione}
Si garantisce l'aderenza ai principi di buona progettazione del software, migliorando modularità, manutenibilità e chiarezza del sistema.
\subsubsection{Livelli di coesione}
\begin{figure}[h]
	\centering
	\includegraphics[width=.9\linewidth]{images/Coesione classi.jpg}
	\caption{Livelli di coesione classi}
	\label{fig:livelli di coesione classi}
\end{figure}
\newpage
\subsubsection{Livelli di accoppiamento}
\begin{figure}[h]
	\centering
	\includegraphics[width=.9\linewidth]{images/Accoppiamento_classi.jpg}
	\caption{Livelli di accoppiamento classi}
	\label{fig:livelli di accoppiamento classi}
\end{figure}

\subsubsection{Relazioni tra le Classi}
Nel progetto si è prediletta l’associazione rispetto alla specializzazione (ereditarietà). \\
In particolare:
\begin{itemize}[noitemsep, topsep=5pt]
	\item Esiste una relazione di aggregazione 1 a molti tra \textit{AddressBook} e \textit{Contact}, tra \textit{AddressBook} e \textit{Tag}.
	\item È presente una relazione di composizione 1 a 0..1 tra \textit{AddressBook} e \textit{Database}, poiché la rubrica può prevedere il salvataggio su database o in locale.
\end{itemize}

\subsubsection{Riduzione dell’Accoppiamento}
Per seguire il principio di riduzione dell’accoppiamento tra classi, sono state create le interfacce \textit{TagManager} e \textit{ContactManager}. \\
Questo consente alle classi \textit{Export}, \textit{Import} e \textit{ManageTagsPopupController} di accedere solo ai metodi strettamente necessari, favorendo anche il principio di segregazione delle interfacce.\\
Tuttavia:
\begin{itemize}[noitemsep, topsep=5pt]
	\item \textit{MainController} ha un riferimento diretto a \textit{AddressBook}, poiché dipende da esso per la maggior parte dei metodi, esclusi quelli relativi ai tag.
	\item \textit{ConfigPopupController} ha anch’esso un riferimento diretto a \textit{AddressBook}, ma con un accoppiamento per timbro, dato che utilizza solo alcuni metodi.
\end{itemize}
\subsubsection{Relazione con l’Interfaccia Initializable}
Tutti i controller implementano l’interfaccia \textit{Initializable}, ma nei diagrammi di classe tale relazione viene omessa per ridurre il livello di dettaglio e garantire una maggiore visibilità.

\subsubsection{Livelli di dettaglio}
Sono forniti due diagrammi di classi con livelli di dettaglio differenti:
\begin{itemize}[noitemsep, topsep=5pt]
	\item Diagramma con relazioni evidenziate: Mostra solo i metodi e attributi più rilevanti, evidenziando meglio le relazioni.
	\item Diagramma completo: Descrive in dettaglio tutti gli attributi e metodi delle classi, evidenziandone più concretamente il ruolo nella realizzazione del sistema.
\end{itemize}

\subsubsection{Diagrammi di Sequenza}
Sono forniti diagrammi di sequenza per descrivere i flussi di eventi relativi all’interazione tra l’utente e la rubrica, tra cui:
\begin{itemize}[noitemsep, topsep=5pt]
	\item Alcuni casi d’uso specifici (C1, C2, C3, C5, C6, C8).
	\item L’operazione di aggiunta del database tramite link inserito dall’utente.
	\item Lo scenario di apertura e inizializzazione della rubrica, che può avvenire attraverso il database o il file locale \texttt{Data.bin} in assenza del database.
\end{itemize}
	
\end{document}